\documentclass[oneside]{utmthesis}
%By default, print on two-side, can change to oneside printing 
%According to the new manual, should not mixed single-side with two-side printing

\usepackage{graphicx}
\usepackage{url} 
%\usepackage[pages=some]{background}
%\usepackage{lipsum}
\usepackage{pdflscape}
\usepackage{longtable}
\usepackage{siunitx}
\usepackage{subcaption}
\usepackage{multirow}

%%% You MUST load the natbib package if you want to use author-date bibliography style. Also remember to change the bibliography style at the bottom of the .tex file.
% Comment natbib for citation by number
\usepackage{natbib}
\let\cite\citep

%This is to make sure vertical spacing non-stretchable
\raggedbottom

\begin{document}

% Macros for commmon symbols <<<
\newcommand{\ypl}{y^{+}} %yPlus
\newcommand{\ured}{U^{*}} %reduced velocity
\newcommand{\yrms}{y^{*}_{\text{RMS}}} %root-mean-square of the normalised cylinder displacement
\newcommand{\ystr}{y^{*}} %the normalised cylinder displacement
\newcommand{\fstr}{f^{*}} %the normalised vibration frequency
\newcommand{\fn}{f_{n}} %system natural frequency
\newcommand{\fk}{f_{k}} %the coarsest grid in a grid independence study
\newcommand{\fvstr}{f^{*}_{v}} %normalise vortex shedding frequency
\newcommand{\fvk}{f_{v,\text{Karman}}} %Karman vortex shedding frequency
\newcommand{\fvkstr}{f^{*}_{v,\text{Karman}}} %normalised Karman vortex shedding frequency
\newcommand{\fcyl}{f_{\text{cyl.}}} %frequency of cylinder vibration
\newcommand{\fosc}{f_{\text{osc.}}} %frequency of cylinder oscillation
\newcommand{\fclstr}{f_{\text{Cl}}^{*}} %normalised frequency of lift coefficient
\newcommand{\flrms}{F_{\text{L,RMS}}} %root-mean-square of the lift force
\newcommand{\fl}{F_{\text{L}}} %the lift force
\newcommand{\clrms}{\text{Cl}_{\text{RMS}}} %root-mean-square of the lift coefficient
\newcommand{\cflyt}{C_{F_{L},y(t)}} %IMF component of lift that is most similar to the displacement signal in terms of temporal evolution of amplitude and frequency, differing only perhaps in phase OR the component of lift with the highest correlation to the displacement signal
\newcommand{\cflkrms}{C_{F_{L},\text{Karman},\text{RMS}}} %the Karman component of lift
\newcommand{\cflsrms}{C_{F_{L},\text{streamwise},\text{RMS}}} %the streamwise component of lift
\newcommand{\ccli}{C_{\text{Cl},i}} %the ith component of lift coefficient
\newcommand{\cclystr}{C_{\text{Cl},\ystr}} %the ith component of lift coefficient
\newcommand{\cflm}{C_{F_{L},\text{max}}} %IMF component of lift that has maximum RMS amplitude in the IMF set
\newcommand{\cyrms}{C_{y,\text{RMS}}} %the RMS of the component of lift that is most correlated with the cylinder displacement signal
\newcommand{\cclrms}{C_{\text{Cl},\text{RMS}}} %the RMS of the component of lift that is most correlated with the cylinder displacement signal (new symbol)
\newcommand{\cysys}{C_{\ystr,\ystr}} %the characteristic IMF representing the normalised cylinder displacement
\newcommand{\cclys}{C_{\text{Cl},\ystr}} %the characteristic IMF representing the lift coefficient
\newcommand{\afl}{\alpha_{F_{L}}} %ratio between two dominant IMF components of the lift
\newcommand{\cd}{\text{C}_{\text{d}}} %drag coefficient

\newcommand{\angfi}{\SI{90}{\degree}} %90 deg. angle
\newcommand{\angfo}{\SI{67.5}{\degree}} %67.5 deg. angle
\newcommand{\angth}{\SI{45}{\degree}} %45 deg. angle
\newcommand{\angtw}{\SI{22.5}{\degree}} %22.5 deg. angle
\newcommand{\angon}{\SI{0}{\degree}} %0 deg. angle

\newcommand{\pfrms}{P_{\text{Fluid,RMS}}} %estimated root-mean-square of fluid power
\newcommand{\pmrms}{P_{\text{Mech.,RMS}}} %estimated root-mean-square of mechanical power
\newcommand{\etamech}{\eta_{\text{Mech.}}} %mechanical power efficiency
\newcommand{\re}{\text{Re}} %Reynolds number
\newcommand{\st}{\text{St}} %Strouhal number
\newcommand{\plag}{\theta_{y-\text{Cl}}} %Characteristic phase lag
\newcommand{\phim}{\phi_{\text{mean}}} %mean phase lag
\newcommand{\wcl}{W_{\text{cyl.}}} %mean work done by cylinder over one cycle of vibration
\newcommand{\tosc}{T_{\text{osc.}}} %mean period of cylinder oscillation
\newcommand{\meff}{m_{\text{eff.}}} %effective mass
\newcommand{\zetatot}{\zeta_{tot.}} %total damping of the system

%Macros that are shorthands in writing
\newcommand{\rms}{root-mean-square} %shorthand for root-mean-square

%Macros used in writing section on GCI study
\newcommand{\rp}{r^{p}} %refinement ratio, used in GCI study
\newcommand{\fre}{f_{\text{RE}}} %Richardson extrapolation of quantity of interest, used in GCI study

%The macros for freestream velocities
\newcommand{\uon}{\SI[per-mode=symbol]{0.1}{\metre\per\second}}
\newcommand{\utw}{\SI[per-mode=symbol]{0.2}{\metre\per\second}}
\newcommand{\uth}{\SI[per-mode=symbol]{0.3}{\metre\per\second}}
\newcommand{\ufo}{\SI[per-mode=symbol]{0.4}{\metre\per\second}}
\newcommand{\ufi}{\SI[per-mode=symbol]{0.5}{\metre\per\second}}
\newcommand{\usi}{\SI[per-mode=symbol]{0.6}{\metre\per\second}}
\newcommand{\use}{\SI[per-mode=symbol]{0.7}{\metre\per\second}}
\newcommand{\uei}{\SI[per-mode=symbol]{0.8}{\metre\per\second}}
\newcommand{\uni}{\SI[per-mode=symbol]{0.9}{\metre\per\second}}
\newcommand{\ute}{\SI[per-mode=symbol]{1.0}{\metre\per\second}}
\newcommand{\uel}{\SI[per-mode=symbol]{1.1}{\metre\per\second}}
\newcommand{\utv}{\SI[per-mode=symbol]{1.2}{\metre\per\second}}
\newcommand{\utt}{\SI[per-mode=symbol]{1.3}{\metre\per\second}}

\newcommand{\uron}{2.3}
\newcommand{\urtw}{4.5}
\newcommand{\urth}{6.8}
\newcommand{\urfo}{9.1}
\newcommand{\urfi}{11.4}
\newcommand{\ursi}{13.6}
\newcommand{\urse}{15.9}
\newcommand{\urei}{18.2}
\newcommand{\urni}{20.5}
\newcommand{\urte}{22.7}
\newcommand{\urel}{25.0}
\newcommand{\urtv}{27.3}
\newcommand{\urtt}{29.5}
% >>>

% Required information
\title{Energy Harvesting from an Elastically Supported}
\titletwo{Cruciform Structure}
% \titlethree{Third Line (Optional)}
% \titlefour{Fourth Line (Optional)}
\author{Ahmad Adzlan Fadzli Bin Khairi}
\degree{Doctor of Philosophy}
\specialization{Mechanical Engineering}
\intakeyear{2017}
% \school{Malaysia-Japan International Institute of Technology}
\faculty{Malaysia-Japan International Institute of Technology}
\titledate{December 2020}
\award{5}
% Options for Award 
% 1. Bachelor Degree Project Report
% 2. Master's Project Report (By course work)
% 3. Master's Dissertation (By course work and research)
% 4. Master's Thesis (By research)
% 5. Doctor of Philosophy Thesis
% 6. Other PhD Thesis
% 7. Generic PhD Thesis
% 8. Thesis Proposal
\superone{Mohamed Sukri Bin Mat Ali}
% \supertwo{M.Y. Other Supervisor}
% \superthree{M.Y. Superlong named supervisor}
% \superfour{Fourth SV}
% \superfive{Fifth SV}

% Mandatory pages
\coverpage
\superpage
\certification
\frontmatter
\maketitle
\declaration


\begin{dedication}
  My dearest wife and children. Respected supervisor and colleagues at the WEE iKohza. This is for all of you.\
\end{dedication}

\begin{acknowledgement}
I would like to first express my highest sense of gratitude to my supervisor Dr. Mohamed Sukri Mat Ali. His knowledge and proficiency in computational fluid dynamics have made my entry into this field of study a little bit more bearable and has since been an excellent tool in making novel discoveries and conclusions. His open stance in receiving ideas and suggestions to strengthen the foundations of this research has been fundamental to the preservation of the originality and timeliness of this work. I have learned a lot from him about the value of logical continuity and how to achieve it throughout the course of completing this work. The relentless effort for logical continuity throughout the thesis has been, in my opinion, crucial towards a manuscript that not only is easily accessible but helps target audiences to build upon this work in the future, hence advancing the field as a whole.

As for my beloved family, especially my wife and two babies, I really cannot think of any way to repay your hardships and understanding throughout the years of my study. Even a lifetime of unconditional love and devotion towards all of you squares only but a small fraction of what you have given up making my studies a reality. To that end I can only trust the Most Merciful to fully even the debt and make all of you among those who are most beloved by the Most Gracious. I pray to the Almighty to accept the fruit of our jihad in finding knowledge and may all the trials and tribulations we have encountered together, physically, psychologically and financially, will stand witness in the hereafter that we have answered His call towards submission, and His call towards success. Aamin ya Rabbal `alamin.
\end{acknowledgement}


\begin{abstract}
  From off-grid charging of electronic devices to energising independent wireless sensor networks, the demand for stand-alone, low-power generators from renewable energy sources is becoming ever more prevalent.
  This study aims to address this need, by numerically investigating a cruciform energy harvester that comprises of an elastically supported circular cylinder, and a downstream strip plate at right angles in the Reynolds number range $1.1 \times 10^{3} \leq \text{Re} \leq 14.6 \times 10^{3}$ and Scruton number 9.94. 
  The continuity and three dimensional, unsteady Reynolds-averaged Navier-Stokes equations are solved on the numerical domain using a free and open-source C++ library called OpenFOAM. The Spalart-Allmaras turbulence model is used to provide closure to the governing equations. Previous studies on the power output from a 10 mm diameter cylinder show that meaningful power generation only begins when the reduced velocity $\ured$, exceeds 15 and produces a consistent output in the order of 1 mW over the whole observation window.
  To improve upon this, a more fundamental understanding of why this observation takes place is indispensable. This is done by investigating the temporal evolution of the lift and displacement signals using the Hilbert-Huang transform, leading to the discovery of a route through which a significant quantity of energy is lost during energy harvesting. To eliminate said route, this work examines energy harvesting of a generalised cruciform structure, with varying cruciform angles, and discovered the following. For steep-angled cruciforms ($45 \le \alpha (\si{\degree}) \le 67.5$) this study found asymmetries in the vortical structures that prevents lock-in and thus high-amplitude vibrations from taking place. However, for shallow-angled cruciforms ($0 \le \alpha (\si{\degree}) \le 22.5$), this work discovered a high degree of symmetry in the distribution of vortical structures, leading to the onset of meaningful power generation as early as $\ured = \urfo$ up to the upper limit of observation, with a maximum mechanical power that is one order of magnitude larger than the highest reported by similar studies in the literature.
  Finally, the mechanical power and efficiency of the generalised cruciform energy harvester are presented as a map in cruciform angle-reduced velocity ($\alpha(\si{\degree})-\ured$) parameter space, thus making it possible for future engineers to tailor the design of their cruciform energy harvester to their specific power and efficiency needs.
\end{abstract}

\begin{abstrak}
  Daripada pengecasan peranti elektronik di luar grid sehingga pentenagaan rangkaian sensor tanpa wayar, permintaan untuk sistem penjana berkuasa rendah dari sumber tenaga boleh diperbaharui adalah semakin meningkat. Kajian ini bertujuan untuk memenuhi keperluan tersebut, dengan cara menyelidik sebuah sistem pemungut tenaga jenis krusiform yang terdiri daripada sebuah silinder bulat yang disokong secara elastik, dan sekeping plat jalur yang diletakkan pada sudut tegak di hilirnya dalam julat nombor Reynolds $1.1 \times 10^{3} \leq \text{Re} \leq 14.6 \times 10^{3}$ dan nombor Scruton 9.94.
  Persamaan keselanjaran dan Navier-Stokes purata Reynolds tidak stabil tiga dimensi diselesaikan dalam domain numerikal menggunakan OpenFOAM, sebuah pustaka C++ yang percuma dan bersumber terbuka. Model gelora Spalart-Allmaras digunakan untuk melengkapkan persamaan tersebut. Kajian terdahulu tentang kuasa dari silinder berdiameter 10 mm menunjukkan penjanaan kuasa yang signifikan bermula apabila halaju terturun $\ured$ melebihi 15, seraya menghasilkan kuasa dalam julat 1 mW secara konsisten.
  Kefahaman yang lebih mendalam tentang pencetus respon mekanikal tersebut adalah diperlukan untuk membolehkan penambahbaikan terhadap keputusan ini. Ini dilakukan dengan mengkaji evolusi temporal bagi daya angkat dan sesaran menggunakan penjelmaan Hilbert-Huang, yang membawa kepada penemuan proses yang membazirkan jumlah tenaga yang signifikan dalam aktiviti penjanaan kuasa. Bagi merencat proses tersebut, tinjauan dilakukan terhadap pemungut tenaga daripada bentuk krusiform yang lebih umum, sekaligus membawa kepada penemuan-penemuan berikut.
  Bagi krusiform bersudut tinggi ($45 \le \alpha (\si{\degree}) \le 67.5$), kajian ini mendapati asimetri dalam struktur vorteks yang menghalang kejadian frekuensi terkunci, serta penghasilan amplitud getaran yang tinggi, daripada berlaku. Di sudut yang lain, bagi krusiform bersudut rendah ($0 \le \alpha (\si{\degree}) \le 22.5$), kajian ini mendapati darjah kesimetrian yang tinggi dalam taburan struktur vorteks di sekeliling krusiform tersebut, yang membenarkan pemungutan tenaga yang signifikan bermula seawal $\ured = \urfo$ sehingga batasan tertinggi pemerhatian dalam penyelidikan ini, dengan penghasilan kuasa maksimum satu peringkat magnitud lebih tinggi daripada yang tertinggi pernah dilaporkan dalam kajian yang setara.
  Akhir sekali, tulisan ini memetakan kuasa dan kecekapan mekanikal bagi krusiform terubahsuai tersebut dalam ruang parameter sudut krusiform-halaju terturun ($\alpha (\si{\degree})-\ured$), sekaligus membolehkan penyesuaian rekabentuk pemungut tenaga mengikut perincian kuasa dan kecekapan yang diperlukan.
\end{abstrak}


\tableofcontents
\listoftables
\listoffigures


%List of abbreviation 
\listofabbre
\addabbre{ANN}{Artificial Neural Network}
\addabbre{PC}{Personal Computer}
\addabbre{SVM}{Support Vector Machine}
\addabbre{UTM}{Universiti Teknologi Malaysia}
\addabbre{XML}{Extensible Markup Language}


%List of symbols 
\listofsymbols
\addsymbol{Re}{Reynolds number}
\addsymbol{St}{Strouhal number}
\addsymbol{$\ured$}{Reduced velocity}
\addsymbol{Cl}{Lift coefficient}
% \addsymbol{$\sigma$}{Whatever}
% \addsymbol{$\varepsilon$}{Whatever}


%Uncomment if have appendices
\listofappendices


\mainmatter


\chapter{Introduction} \label{chap:introduction}
Vortex-induced vibration (VIV) is a type of vibration that grows from instabilities in fluid flows moving past a solid object, i.e. bluff body. When the flow exceeds a critical velocity, the flow develops vortices that are shed alternately downstream the bluff body. This triggers the onset of unsteady lift and drag forces that initiate and sustain its vibration \citep{Bukka2020}. Numerous occurrences of VIV are readily observable in the field of engineering. In the ocean, marine currents give rise to the vibration of risers and offshore drilling platforms \citep{Liu2020,Zhang2020,Meng2020}. Up in the sky, aeroplane wings vibrate, and high-rise buildings experience sway as strong gusts blow around the mighty structures \citep{Arul2020,Hao2020,Gao2020}. Closer to the ground, power transmission lines vibrate as the result of wind blowing over them \citep{Wang2019,Gomez-Ortega2019}.

The common denominator for all these examples is the potential damage to the engineering construct experiencing it. Thus, methods are devised and implemented to mitigate the effects of the vibrations by dissipating the vibrational energy or delaying/aborting its onset in the first place.

However, the past decade has seen efforts to do exactly the opposite: purposely maximising the vibration induced by the vortices, with the aim to generate electrical power. The simplicity of design and scalability attracts many to contribute to this multidisciplinary field of study, along with the prospect of successful development and subsequent commercialization of a new generation of energy harvester.

\section{Background}
The term “flow-induced vibration” refers to a wide range of phenomena: flutter \citep{Doare2011,Xia2015a,Xia2015b}, galloping \citep{Kluger2013,Barrero-Gil2010,Luo2003,Chen2012}, turbulence-induced vibration \citep{Nakamura2013}, wake-induced vibration \citep{Ogink2010,Bearman2011,Derakhshandeh2014}, and VIV of various kinds, which are the main object of study in this proposal. The study of VIV is traditionally motivated by the potential failure of engineering structures resulting from the fluid moving around them \citep{Shiraishi1988,Nakamura1986,Larsen1997,Khalak1999}. Nevertheless, technical publications since the 2000s saw a surge in contributions toward the subject from the perspective of energy harvesting. A simple search in SCOPUS shown in Fig. \ref{fig:scopusTrend} reveals this trend for keywords [“vortex induced vibration” energy] for the last 4 decades.

\begin{figure}
  \centering
  \includegraphics[width=0.9\textwidth]{figs/scopusTrend}
  \caption{Number of publications with keywords ["vortex induced vibration" energy]. Retrieved from SCOPUS.}
  \label{fig:scopusTrend}
\end{figure}

At the cutting edge of this field of research is a group at The University of Michigan, that has already built prototypes of the energy harvester, named VIVACE. They compared the cost of power production in USD/kWh between VIVACE and a wide selection of common (pulverised coal, integrated gasification combined cycle, natural gas combined cycle, etc.) and new power generation technologies (anaerobic digester, landfill gas, solar, etc.). The result of this comparison demonstrated how VIVACE is on par in terms of power production cost with most of the technologies it was contrasted to \citet{Bernitsas2008a}.

The VIV phenomenon utilised by the team at the University of Michigan is of the Karman VIV type (KVIV), capable of producing power in the order of MW when installed as a large-scale energy farm \citep{Raghavan2007}. However, as pointed out by \citet{Koide2013} the reduced velocity ($\ured$) range within which KVIV can be relied upon for power generation is about one order of magnitude smaller than what can be expected from another form of VIV namely the streamwise VIV (SVIV). Reduced velocity $\ured$ is a nondimensional form of characteristic velocity that allows comparison of results between similar systems of differing dimensions. Since SVIV power generation is possible for a large range of $\ured$, it is better suited for deployment in flows with large velocity changes.

\begin{figure}
  \centering
  \includegraphics[width=0.9\textwidth]{figs/apparentPowerKoide}
  \caption{Apparent power $P_a $ (W) versus reduced velocity $\ured$ for cases of KVIV and SVIV. Adapted from \citet{Koide2013}.}
  \label{fig:apparentPowerKoide}
\end{figure}

Despite this, the main shortcoming of SVIV is its maximum power output which is demonstrated at this stage of development to cap at a mW scale for a single-cylinder setup. An isolated cylinder setup for KVIV produces a maximum power in the order of 10 W \citep{Bernitsas2009}. The apparent power $P_a$ (W) for both KVIV and SVIV is shown in \citet{Koide2013}. Following this present limitation of the unoptimized SVIV energy harvesters, their application is currently limited to mW electronics e.g., sensors and signal transmitters.

% One of the most immediate uses of such sensors in Malaysia is part of a flood early warning system. Flood forecasting is possible without much difficulty using conventional methods if there is a sufficient number of hydrological observatories along the body of water \citep{DIDMalaysia2010}. For this to take place, electricity must be available to run the observatories, in addition to favourable terrain near the body of water to house the equipment.

% In urban areas, even though sourcing electricity may not be a hurdle, placement of the observatory itself can be, due to land ownership issues. Nevertheless, these issues do not in any way lessen the need to have an adequate number of observatories in urban areas. After all, urban areas are known to bear a greater risk of flash floods compared to rural areas owing to disrupted natural systems of runoff production as a result of poorly planned development \citep{Abdullah2004,Takaijudin2010,Abdullah2011}. Therefore, the prospect of devising a simple river monitoring system that consists of only a few sensors and electronics totalling to a maximum combined number of three (3), powered by a TVIV energy harvester hence becomes worthy of further inquiry.

\section{Problem Statement}

The preceding section has established the viability of harnessing energy from a flow by exploiting the VIV phenomenon. Multiple modes of VIV have been observed, and SVIV stands out as better oriented for deployment in fluid flows that vary greatly in terms of free-stream velocity. Even with very rudimentary optimisations, SVIV has demonstrated its ability to generate power in the order of mW consistently over a large range of free-stream velocities. This can be harnessed to develop a self-contained power source for off-grid electronics charging, or to power independent sensor networks.

% This can be harnessed to develop a self-contained river monitoring device that deploys with ease, especially in urban neighbourhoods to facilitate early warning of floods.

To achieve this, the problems outlined below must be addressed to close relevant gaps in the current body of knowledge.

\begin{enumerate}
  \item A lack of understanding on the transition mechanism from Karman to streamwise vortex-induced vibration.
  \item A paucity in the knowledge on what contributes to the magnitude of the alternating lift force acting on the cylinder, and its vibrational frequency components.
  \item A deficiency of new methods to control the flow perturbation which gives rise to a strong, stable and periodic forcing of the cylinder vibration, sustainable over the desired operating range of $\ured$.
\end{enumerate}

Addressing the above problems will provide a better understanding on the origins of the streamwise vortex pairs, uncover new perspectives on variables that affect the alternating lift force in the context of streamwise vortex pairs, and generate novel insight on new flow regimes, which can enlighten us to a better configuration of the energy harvester, i.e. producing more power than ever before.

\section{Research Questions}

The answer to several questions is sought in this proposed study. These questions are meant to drive the study towards its objectives.
\begin{enumerate}
  \item How does the lift signal evolve as the flow transitions from being driven primarily by Karman vortex to streamwise vortex?
  \item How does the ratio of energy transferred from the flow to the lift components evolve with respect to $\ured$?
  \item What deviations do the modified cruciform configuration impose on the vortical structure, against the vortical structure observed around a pure cruciform? \label{enum:deviation}
  \item How do the deviations mentioned in \ref{enum:deviation} affect the lift magnitude, and by extension the frequency-amplitude response?
  \item Where in the power envelope can we obtain maximum (minimum) power with the largest (narrowest) operability range, and how does this translate into a new mode of flow control to suit the operating conditions of the cruciform energy harvester?
\end{enumerate}

\section{Thesis Objectives}
Following the problems outlined in the previous section, the objectives that define the scope of work in this proposal are listed below.

\begin{enumerate}
  \item To investigate what takes place when the dominant vortical structure forcing the vibration changes from Karman to streamwise vortex-induced vibration (SVIV) in terms of lift and vibration signals. \label{enum:whatHappens}
  \item To characterise the lift signal in terms of its components and how the components interact to modify the frequency-amplitude response of the cylinder. \label{enum:characteriseLift}
  \item To propose a new passive control method for an SVIV-oriented energy harvester that modifies the vortical structures and their distribution around the oscillator to control its region of operability. \label{enum:passiveControl}
\end{enumerate}

The aim of objective item \ref{enum:whatHappens} is to get a better understanding of how the advent of streamwise vortical structures perturb the lift acting on the cylinder, which directly modifies the vibration signal of the cylinder. Objective \ref{enum:characteriseLift} is an attempt to identify the footprint of dominant vortical structures in the flow in the lift signal and relate those to the resulting vibration signal. Finally, objective \ref{enum:passiveControl} seeks to recommend and evaluate a modified version of the cruciform structure that alters the vortical structure in the flow, thus modulating the lift signal acting on the cylinder and its frequency-amplitude response. The power envelope will give us a more generalised operability condition for the energy harvester, and how we can vary the cruciform configuration to cater to a particular flow environment.

\section{Significance of Study}

% As a whole, Malaysia receives an average rainfall of about 2990 mm annually (Harris et al., 2014). East Malaysia registers an approximate 3250 mm - nearly 1000 mm in excess of the average rainfall figures for west Malaysia which is about 2500 mm \citep{DIDMalaysia}. Contribution to these values is greater during the monsoon seasons which occur during the months of November to March (north-west monsoon) (Kuala Lumpur Monsoon Activity Centre, 2015) and May to September (south-west monsoon) \citep{Billa2004}. The increased rainfall during the monsoon season inevitably saturates catchments, producing several times the usual amount of runoff that rivers can drain to the sea within sufficient time \citep{Chia2004}. Consequently, water lever rises past the river banks and progresses into the surrounding terrain. This is how floods commonly come into being during the monsoon season.

% Outside the monsoon season, the inception of floods is due to convective rainfall. The flood frequency and extent due to convective rain are worse in an urban setting compared to rural areas. The underlying cause of this is none other than the major disruption of pre-urbanization mechanisms that govern the original rainfall-runoff processes of an area. Thus, crippling floods can manifest within a matter of hours from the start of rainfall, i.e. flash floods.

Structural efforts to put flash floods in check are almost always very costly - RM443 million allocated in the 2020 Malaysia Budget alone \citep{KementerianKewanganMalaysia2019}. However, the rationale for structural solutions to the floods becomes increasingly suspect considering the worsening climate change in recent decades. However, the allocation for non-structural methods of reducing flood damages such as flood forecasting and warning protocols, was not highlighted in this budget, despite them being part of the 2016-2020 Civil Service ICT Strategic Plan \citep{MAMPU2016} and 2016-2020 Department of Irrigation and Drainage Strategic Plan \citep{DIDMalaysia2016}. This fact, along with the decreasing effectiveness of structural efforts highlights the need to develop an inexpensive, simple, yet reliable non-structural system to tackle the flood problem.

Against this background, this study is able contribute significantly towards the development of an in-situ device to power water level and flow velocity/discharge sensors for a given river/drainage system using vibration energy harnessed from the flow itself. The minimalist and self-powered feature of the device allows installation of as many of the devices as necessary along the river/drainage system for adequate monitoring.

% Finally, pursuant to further scaling studies, the upscaled version of the energy harvester may one day be on par with diesel generators for use at off-grid areas.

Off-grid charging of electronics is also another area to which the results of this study is able to improve. Both civilian and military activities in remote areas can benefit from the power generated to charge essential electronic devices necessary for a successful operation in said area. Improvement in terms of power output means that usage of more complex and resource intensive electronic equipments will gradually be possible in off-grid locations. This gives the deployed personnel an added advantage over their peers in performing his or her civilian or military duties in such locations.

The most significant contribution is however, to facilitate the widespread adoption of independent, wireless sensor networks. Sensor networks have wide-ranging applications in the field of monitoring; from wildlife \citep{Gazis2020,Pathak2020} to forests \citep{Kadir2019,Zellweger2019} to civil structures \citep{Ni2020,SadeghiEshkevari2020,Mao2020}, these networks are low-power by design and is expected to operate around the clock. To increase the coverage and resolution of the monitoring task, the obvious solution is to increase the overall quantity and concentration of sensors in a particular area of interest. Nevertheless, the impracticality of connecting such a network to the national grid thrusts them in a unique position to benefit from VIV-based energy harvesters that this study seeks to improve upon.

\section{Thesis Scope}
This work is a mainly a computational fluid dynamics (CFD) study of a particular version of VIV-based energy harvester that comprise of an elastically supported, horizontally constrained smooth circular cylinder of diameter 1 cm and a passive flow control mechanism that is a strip of rectangular plate at a right angle downstream the cylinder, forming a cruciform. The range of Reynolds number investigated in this thesis is between $1.1 \times 10^{3} \leq \text{Re} \leq 14.6 \times 10^{3}$ and the mass-damping parameter, expressed by the nondimensional Scruton number Sc, is 9.94. This work limits itself to examining a cruciform where the width of the strip plate is equal to the diameter of the cylinder $D$, and the primary data collected from the simulation runs are the time evolution of the cylinder displacement and the corresponding lift coefficient Cl.

The baseline numerical results, i.e. results from a pure cruciform (a cruciform where the cylinder and strip plate are \SI{90}{\degree} to each other) are validated against experimental results of a similar system in a custom-made recirculating open flow channel. Although analogous systems have been studied experimentally in the past \citep{Koide2017,Zhao2018a}, in the author's opinion it is best to test the extent of reproducibility of the results in these studies first before attempting any sort of comparison with the numerical results of this work. Furthermore, having a system similar to the numerical model studied in this thesis physically available allows for rapid consistency checks to be made between the two, thus increasing the credibility of the results.

The first part of this study looks to elucidate the relationship between the vortical structures present in the flow, especially when the ambient fluid power is high (high flow velocity region), and how they modulate the resulting lift acting on the cylinder. This is done by conducting a time-series analysis of both cylinder displacement and lift coefficient signals using the Hilbert-Huang transform (HHT). The motivation behind this is to understand why the amplitude of cylinder displacement is limited to the order of magnitude observed not only by the author, but also in numerous studies within the last ten years. This part of the study concludes with the discovery of a particular route through which a significant amount of energy from the freestream is lost during the energy harvesting process, and the amount by which the power output can be improved if this loss is eliminated.

The second part of this thesis is the author's attempt to eliminate the loss mentioned previously. This is done by generalising the cruciform system, through the variation of the relative angle between the cylinder and the strip plate. The study then proceeds to investigate the generalised cruciform system by examining the vortical structures present in the flow, how they affect the resulting lift acting on the cylinder, the amplitude of cylinder displacement itself, and ultimately the power output. The dynamics between the lift and cylinder displacement are explained through the computation of instantaneous phase lag between the two, which in turn is made possible by HHT.

The thesis concludes with the unveiling of a mechanical power and efficiency map, within a parameter space consisting of the cruciform angle and flow velocity. Useful recommendations can be deduced from the map, which highlights regions of high and low power output, and also regions of high and low efficiency, in order to obtain the desired power output and efficiency for any given power consumption requirement.

\section{Thesis Organisation}
This thesis is organised into eight chapters. The author introduces the study and gives a general overview of the research in Chapter 1. In Chapter 1, gaps in the research are identified and thesis objectives are formulated based on those gaps. Chapter 1 also outlines the questions the author seeks to address, details the scope of this study and provide concrete examples as to the significance and merit of this work. Chapter 2 reviews relevant literature that gives an overview of the progress made up to the present day, on the subject of VIV energy harvesting, by exploiting an isolated circular cylinder as the oscillator. The chapter then introduces the cruciform oscillator and the studies on the vibration characteristics of a number of variations of the cruciform oscillator.

Chapter 3 discusses the methodology taken by the author to attain the objectives listed in Chapter 1. In Chapter 3, the author details the numerical model implemented in the CFD undertaking and this includes the domain size, critical dimensions of the cruciform, boundary conditions and solution method to the unsteady, three-dimensional (3D) Reynolds-averaged Navier-Stokes equation governing the flow. Apart from that, the author also discusses the turbulence modelling adhered to in the numerical studies. The author also introduces the Hilbert-Huang transform (HHT) and explains the ensemble empirical mode decomposition (EEMD) algorithm that drives the decomposition of a time series signal into a finite number of orthogonal components. Finally, the author explains the Hilbert transform and how the transform is able to compute instantaneous phase or frequencies of a decomposed component of the signal.

Chapter 4 takes into account the validation of the numerical setup in two ways: by way of a grid independency study, and by way of experimental comparison. The grid independency study utilises the Richardson extrapolation and grid convergence index (GCI) as the primary tool to ensure spatial convergence of the numerical results. In the experimental validation, this work showcases a simple contactless method of measuring the cylinder displacement using a camera and an open-source image tracking software. After the processing of the experimental data to compute the uncertainty and present them as error bars, the author concluded that the numerical results of the pure cruciform (\SI{90}{\degree} cruciform) is in fair agreement with the experimentally obtained values, providing an added layer of confidence in the numerical results.

Chapter 5 deals with the vibration characteristics of a pure cruciform. In this chapter, the author studies in detail the lift-displacement dynamics that results from the kind of vortical structures that appear in this setup. This chapter concludes with the discovery of a path to energy loss that has never been considered before in the literature and estimated the amount of improvement possible for the power output if said loss is eliminated. Chapter 6 discourses about the vortical structures and lift-displacement dynamics of a steep-angled cruciform ($45 \le \alpha (\si{\degree}) \le 67.5$), while Chapter 7 covers the vortical structures and lift-displacement dynamics of a shallow-angled cruciform ($0 \le \alpha (\si{\degree}) \le 22.5$). Here, the study found out that for shallow-angled cruciforms, the onset of meaningful power generation is brought down significantly to from $\ured = \urei$ in the pure cruciform, to $\ured = \urfo$ when the cruciform angle is $\angon$. At $\alpha = \angon$, the maximum power also improves by approximately a power of two.

Finally, in Chapter 8, the author computes the mechanical power and efficiency of each of the cruciform variants for all flow velocities studied. From it, this work is able to produce a mechanical power and efficiency map, in essentially a cruciform angle-flow velocity parameter space.

\chapter{Literature Review} \label{chap:literatureReview}
The alternate shedding of vortices from opposing sides of a cylinder introduces an alternating lift and drag which, unless constrained, will induce the vibration of the cylinder. This vibration energy can be converted into electrical power via the piezoelectric effect or electromagnetic induction. This is the basis of vortex-induced vibration (VIV) energy harvesting, and in this chapter, the author goes through the body of work done on the subject of vortex shedding, vortex-induced vibration and finally energy harvesting from the many variants of cylindrical oscillators, to determine the limits of our current knowledge and pinpoint where this work resides within the web of research done by the community.

\section{Vortex Shedding from a Cylinder}
In this work, a cylinder is defined as an elongated three-dimensional (3D) object with a well-defined axis, and whose cross-sectional shapes are the same at any arbitrary location along the axis. Flow around a cylinder is one of the phenomenon that, although abundantly occurring in nature and in man-made settings, remains impervious to rigorous and analytical description of the phenomenon. Key to this elusiveness is the fact that for a flow pattern to manifest itself in the physical world, the solution to the governing equations of fluid mechanics must not only exist, but must also be stable to ambient random excitations for it to be sustainable. Thus far, the Navier-Stokes equation remains unsolved on the list of Millennium Problems by the Clay Mathematics Institute, prompting researchers to rely heavily on experimental and numerical techniques in the push for breakthroughs in this field of study. Vortex shedding from a cylinder is one such interest in this field - and as the reader will see in the following subsections - whose advances are greatly driven by experimental and numerical research.

\subsection{Karman Vortex Shedding} \label{ssec:kvShedding}
In the study of Karman vortex shedding, seldom is the case where one does not find reference being made to a chart on lift and drag coefficients for a stationary circular cylinder between $10^{0} < \text{Re} < 10^{8}$ in \citet{Zdravkovich1997}. The observation that both lift and drag coefficients (Cl and $\cd$) have their high and low regions within $10^{0} < \text{Re} < 10^{8}$ is of extreme utility to engineers. From the preservation of risers to the improvement of power output for VIV-based energy harvesters, the chart gives a rough outline to the range of Reynolds number suitable for a particular engineering operation. That being said, the chart remains silent on why Cl and $\cd$ evolve the way they do with respect to Re, leaving the gap to be filled by later researchers.

For example, \citet{Desai2020} tackled the question why is there a reduction in Cl and $\cd$ as the cylinder goes through the upper boundary of subcritical flow ($1.49 \times 10^{5} < \text{Re} < 3.55 \times 10^{5}$). Their study is experimental, using tools such as particle image velocimetry (PIV) for a quantitative visualisation of the two-dimensional (2D) flow field, and proper orthogonal decomposition (POD) for identification of dominant flow structures. In doing so, they have identified two modes of shedding - the antisymmetric and symmetric modes - which takes place one after another, albeit in no well-defined cycle. The antisymmetric mode corresponds to the ``normal'' shedding of Karman vortices while the symmetric mode a weaker form of shedding with a higher energy level in its low-frequency components. The symmetric mode gains strength with increasing Re, which entails a weaker form of shedding taking over the flow. This ultimately reduces the magnitude of Cl and $\cd$ of the cylinder. Depending on which side of the divide one is, an engineer may opt to amplify the symmetric mode to protect the integrity of a structure, or augment the antisymmetric mode to improve the conversion efficiency of energy from the freestream into cylinder vibration. In other words, a deeper understanding into the mechanics of Karman vortex shedding invariably leads to better strategies in cylinder motion control.

Expanding on the topic of flow control, \citet{Durhasan2019} conducted an experimental study of the shedding process under the following setup: the circular cylinder is placed in a larger hollow cylinder with varying pore spacing - quantified by their porosity - $\beta$ - and submerged in a water flow at Reynolds number $\text{Re} = 5000$. What they found out is the stunting of the wake from the inner cylinder when $\beta \leq 0.5$, while for porosities $\beta \geq 0.6$, a reduction of 21\% to 87\% in drag is observed.

Another exploitation of porous material in the control of Karman vortex shedding is in the work of \citet{Geyer2020}. In \citet{Geyer2020}, they work to test a method of noise reduction by wrapping certain parts of the cylinder surface with a porous material. This is because, at velocities in the order of $O(10^{1})$ m/s, Karman vortex shedding also exacerbates the level of noise generated by the cyinder. They discovered that the higher the air resistance of the material, and the higher the porosity, the bigger the improvement will be in terms of noise reduction. In addition, the porous wrapping reduces turbulence intensity in the wake near-field, hence making it more uniform.

In studying a flow around a solid object, one quickly finds that experimental methods employed in the investigation is never absolutely independent of the phenomenon being studied. The use of probes, for example, means that the researcher is introducing additional elements into the flow that cannot be fully isolated from affecting the flow, however careful the design of the experiments are. To compensate this shortcoming, studies have employed numerical methods to model the flow under consideration, and track the evolution of quantities of interest in the flow to a degree of resolution limited only by the computational power available. Taking note of the expediency of porous media buffering the cylinder to temper flow behaviour, \citet{Ledda2018} made use of numerical methods to simulate Karman vortex shedding from a square cylinder to gain a deeper insight on how the flow interacts with the porous media to produce the flow responses discussed previously. In this study, the authors observed the efficacy of a porous cylinder surface in not only detaching the recirculation region in the near-wake, but in purging it altogether, at higher porosity values. If a technique can be found to vary the the porosity of the surface, perhaps through the use of smart materials \citep{Arsh2020}, an engineer can basically switch Karman vortex shedding on and off at will.

Other than the usual methods of numerical investigation that work to solve the continuity and Navier-Stokes equations directly (DNS) \citep{Xiong2018,AlvesPortela2020}, or those that employ some form of simplification \citep{Deloze2016,Shao2016}, methods that model molecular dynamics (MD) have also found their way in the arsenal of tools available to researchers in the field of fluid-structure interaction. One such example is the study by \citet{Asano2020}. In it, the authors employed the Lennard-Jones potential to model represent the molecular dynamics of a flow around one or two circular cylinders. The Lennard-Jones potential models the fluid particle distribution, and all physical quantities are then expressed in terms of energy, length and time. What they discovered was, cavitation, although undesirable in normal operation of turbomachineries, can be a determining factor in the onset of Karman vortex shedding. As they decreased the fluid temperature, the cavitation bubbles interact with the shear layer of the cylinder in a manner that pushes the wake further downstream than usual. Eventually, wake where the vortex shedding resides is pushed far enough downstream that the alternating lift that normally acts on a cylinder during Karman vortex shedding vanishes.

Another method for Karman vortex shedding control that imposes some form of modification to the shear layer of the cylinder is through the affixing of fairings on the surface of the cylinder. A recent example of this can be found in the work of \citet{Kang2020}. In \citet{Kang2020}, the fairings affixed to the cylinder surface caused a reduction in both lift and drag acting on the cylinder, although the strength of this effect varies with respect to the actual flow velocity being imposed on the cylinder.

Apart from tempering the shear layer of the cylinder, one can also - similar to \citet{Yokoi2016} - alter the downstream region of the cylinder to achieve a desired response. In their study, \citet{Yokoi2016} introduced a splitter plate on the trailing edge of a circular cylinder and discovered a rich collection of flow patterns that one can achieve simply by varying the following variables: splitter plate length, cylinder forcing amplitude and cylinder forcing frequency. By doing this, one can even achieve a symmetric shedding of Karman vortices from the top and bottom surfaces of the cylinder. To appreciate the significance of this result, one simply needs to recall that alternating Cl and $\cd$ acting on the cylinder is the consequence of Karman vortices shedding alternately from the top and bottom of the cylinder. The fact that we can force the shedding to take place simultaneously from the top and bottom of the cylinder means that we can significantly diminish the amplitude of both Cl and $\cd$, if not altogether.

To be clear, the splitter plate method is not particularly new in near-wake control; splitter plates have been studied in the past under a similar setup \citep{MatAli2011,MatAli2012}, although under completely different flow conditions and cylinder geometry. However, one may find it difficult not to rely to some extent on the studies by \citet{MatAli2011,MatAli2012}, since unlike \citet{Yokoi2016}, they study flow patterns that are self-induced and does not rely on external forcing. This is particularly pertinent as in nature and man-made settings, self-induced vibrations is the norm rather than the exception.

\subsection{Streamwise Vortex Shedding} \label{ssec:svShedding}
In the real world, Karman vortex shedding is almost never a two-dimensional phenomenon. Three-dimensionality dominates the flow field and streamwise vortices appear alongside the Karman vortices once Re exceeds 150. Among the seminal studies on Karman vortex three-dimensionality are the works of \citet{Williamson1996a,Williamson1996b}. The author discusses two main modes of three-dimensionality: modes A and B. Mode A has a spanwise wavelength of three to four diameters and appears at $\re = 200$. However, as Re is increased, another mode with a smaller wavelength appears, designated as mode B. The modes consist of vorticities that point in the streamwise direction. The shedding of these modes are therefore a simple representative case of streamwise vortex shedding.

% The Mathematica file used to compute the 20\% value below is in the file liftCoefficients67.50V2HHT.nb
Noticeably, a lot of studies on streamwise vortex shedding discuss the process as a secondary process that is the offspring of the vortical structure that actually dominates the flow field - the spanwise Karman vortex. This trend in viewing streamwise vortex shedding as a secondary process is also found in \citet{Agbaglah2019}. In their numerical study of an isolated square and circular cylinder, they considered two Reynolds number, $\re = 205$ and $\re = 225$, and noted that the shedding frequencies of the two cylinders are quite similar. The Strouhal number St for the square cylinder is 0.152 and is 0.149 for the circular cylinder. It is interesting to note that not only do they differ from each other only by two percent, but also that they differ by less than 20\% from the empirical equation (see, Eq. \ref{eq:karmanStrouhalNumber}) describing the shedding frequency of Karman vortices from a smooth circular cylinder \citep{Blevins1990}.

\begin{equation}
  \text{St}_{\text{Karman}} = 0.198 \left( 1 - \frac{19.7}{\text{Re}} \right)
  \label{eq:karmanStrouhalNumber}
\end{equation}

The fact that St for the square and circular cylinders of \citet{Agbaglah2019} differs less than 20\% of the value estimated using Eq. \ref{eq:karmanStrouhalNumber} indicates that the shedding process of streamwise vortex is overshadowed by Karman vortices, which are larger in scale. Perhaps this is an artefact of using an isolated cylinder, and as long as one uses a similar system, Karman vortex shedding will remain the primary process governing the flow.

More recent works on streamwise vortex shedding have shifted away from a simple isolated circular cylinder configuration to slightly more complex layouts, perhaps to recreate models that are more faithful to real engineering situations. For example, in \citet{Gibeau2018}, an elongated body is studied in the range $3.5 \times 10^{3} \leq \re \leq 7.0 \times 10^{3}$. Through this study, \citet{Gibeau2018} found that the average strength of the streamwise vortices in the upstream boundary layer is ten times smaller than in the wake. They also observed that the wavelength reaches between 0.7 and 0.8 of the elongated plate thickness, $h$.

 In another study, \citet{Gibeau2019} swapped the isolated circular cylinder for a blunt trailing edge body. This blunt trailing edge (BTE) body is made from a leading edge that is a 5:1 ellipse connected to a plate with thickness $h$. This experimental study in the range $2.6 \times 10^{2} \leq \re \leq 2.58 \times 10^{4}$, reveals that the rotational direction of the streamwise vortices are conserved during the shedding cycle, and their wavelength is close to mode B of \citet{Williamson1996a}.

Knowledge on the spatial distribution and key dimensions of streamwise vortices may be insufficient to highlight how one can exploit them to one's advantage. To exploit streamwise vortex shedding to fulfil engineering objectives, one needs insight on the effects of streamwise vortex shedding on the flow. This is demonstrated in the work by \citet{Rai2018}, in which the author published numerical results of flow around a flat plate with a circular trailing edge. The Reynolds number varies from $250 \leq \re \leq 10000$. \citet{Rai2018} discovered that streamwise vortices are responsible for generating steep spanwise velocity gradients that affect the strength of the dominant vortical structure of the flow, i.e. the Karman vortex. The author also found that fluctuations in the streamwise velocity within the boundary layer leads to variation in the shedding period of the dominant Karman vortex. Finally, streamwise vortex shedding is observed to affect the shedding frequency of the spanwise Karman vortices through the speed of their streaks in the boundary layer.

In an evaluation of a passive method to control flow around a circular cylinder, \citet{Lin2018} inserted the cylinder inside a conical shroud and assessed its effect on the vortical structures shed around it. They chose a very low $\re = 100$, to keep the flow as two-dimensional as possible. \citet{Lin2018} tested two types of conical shroud. One shroud is with perforations of varying diameter and placement angle along the peak of the cone, and the other without. The authors also varied the wavelength and outer diameter of the conical shroud. By doing so, the study essentially discovered a method to prevent the energising of Karman vortex when the disturbance brought about by the conical shroud is large enough. Energy from the freestream is redirected away from Karman vortices, to form strong $\Omega$ vortices in the neighbourhood of the conical valley. This results in partial to full suppression of Karman vortex shedding. The ability to refocus energy to a vortical structure of our choice is indispensable, especially to engineers who may either want to minimise cyclic force acting on a structure, or maximise it to improve vibration energy harvesting potential of the structure.

Other examples of energy redirection can be found in \citet{Zhang2016} and \citet{Zhang2019}. In these studies, the authors examined two methods to control Karman vortex shedding from a model bridge. The first method in \citet{Zhang2016} employs suction on the bottom of the bridge to disrupt the homogeneity of the Karman vortex along the span of the bridge. When suction is applied, streamwise vortex - which is considered a secondary process by the authors - becomes energised. The energised streamwise vortex then destabilises the primary vortical structure - the Karman vortex - hence reducing the net force resulting from it.

Providing suction on the bottom of the bridge is an active flow control method. This is because an active supply of energy is required to power the suction from the designated holes. As an alternative, \citet{Zhang2019} proposed using vortex generators affixed similarly to the bottom of the bridge, and evaluates the degree to which a similar result can be achieved. In the final analysis, they demonstrated that the disruption of Karman vortex homogeneity is possible through the use of vortex generators. However, they did not specify quantitatively how much weaker does the Karman vortices get in expense of stronger streamwise vortices.

Attachment of a plate \citep{Gibeau2019} or placement of the cylindrical object inside a conical casing \citep{Lin2018}, even the deployment of vortex generators on the surface of the cylinder \citep{Zhang2019} are all methods of flow control that add another layer of complexity to the cylindrical object. In situations where an engineer seeks to retrofit a flow control method to existing systems with cylindrical objects, direct modification of the cylinder can be difficult and downright undesirable. What if there exists an accessible flow control method that requires only a minimum amount of effort from the end used to install, but is still powerful enough to significantly alter the dominant vortical structures of the flow? The deployment of another cylindrical object downstream the main cylinder seems to be a promising technique.

Flow control using a secondary cylinder placed downstream the main cylinder dates back nearly three decades. Originally, the authors investigating it \citep{Shirakashi1989} is focussed on reducing the magnitude of cyclic forces acting on an isolated circular cylinder system. They placed a secondary circular cylinder downstream the original circular cylinder and explored how, among others, \rms{} lift coefficient of the original cylinder varies as the gap between the original and secondary cylinders are varied. The secondary cylinder is placed downstream the original in a way such that the axes are perpendicular to each other, creating a cruciform. In doing so, they discovered that while the \rms{} of lift coefficient can be reduced as the downstream cylinder is brought closer to the original one, when the normalised gap between two cylinders $g/D$ is $< 0.25$ the lift coefficient rises again and peaks close to $\approx 0.7$ before dropping off again as $g/D \to 0$. This is a significant finding in that it demonstrates how the flow around an existing isolated cylinder system can be altered in hindsight by simply forming a cruciform structure with another cylinder.

The dominant vortical structure in such a cruciform when $0 < g/D < 0.5$ is dominated by a pair of streamwise vortices, shed alternately from the top and bottom of the primary cylinder \citep{Koide2017}. Two types of streamwise vortices are observed, the necklace vortex when $0.25 < g/D < 0.5$, and the trailing vortex when $0 < g/D < 0.25$. Although both types of vortices form close to the cruciform juncture, the necklace vortex dominates a region that is closer to the juncture than the trailing vortex \citep{Takahashi1999}, which forms a little further away from the juncture. The streamwise vortex pair that forms due to the cruciform does not seem to be secondary to the Karman vortex that is also shed from the upstream cylinder. Instead, the streamwise vortices seem to have its own mechanics which is independent from the cycle of Karman vortex shedding.

One obvious example is the deference to Eq. \ref{eq:karmanStrouhalNumber}. Streamwise vortices such as those discussed earlier \citep{Zhang2016,Rai2018,Agbaglah2019,Zhang2019} develop as three-dimensional instabilities of Karman vortex shedding, and as such, have a base shedding frequency that loosely obeys Eq. \ref{eq:karmanStrouhalNumber}. However, streamwise vortices in \citet{Takahashi1999}, and even in \citet{Shirakashi2001} and \citet{Bae2001} do not. Streamwise vortices from these studies are shed at an increasing St as Re is increased within the region $\re < 10000$. When $\re > 10000$, St for the streamwise vortex becomes more or less constant with $\st \approx 0.04$ for the necklace vortex and nearly twice that value for the trailing vortex.

Domination of the cruciform flow field by the streamwise vortex pair does not imply that the Karman vortex completely disappears from the experimental domain. This is demonstrated in \citet{Koide2006} and \citet{Kato2006}. According to the frequency spectra of the upstream cylinder vibration, the Karman shedding frequency still exist in the system, only far smaller in magnitude compared to the amplitude attained by the streamwise vortex structures. The flow field is thus dominated by the streamwise vortex pair close to the cruciform juncture, while the influence of the Karman vortex only grows as one moves towards the ends of the upstream cylinder. This begs the question whether the streamwise vortex-dominant flow is simply a consequence of two circular cylinders at right angles to each other - placed some distance away from each other - or is it something universal to cruciforms structures?

To answer this, \citet{Kato2007} and \citet{Nguyen2010} tested several variations of the cruciform, to see whether a similar occurrance of streamwise vortex shedding can be observed. For example, \citet{Kato2007} swapped the downstream circular cylinder with a strip plate, and observed that similar streamwise vortices do appear in the flow. They are shed at very similar values of St as the two-cylinder cruciform, but the frequency spectra of velocity fluctuations revealed a noisier pattern with less discernable peaks compared to the frequency spectra computed from a two-cylinder cruciform. More recent flow visualisations suggest that the noisier pattern is due to an increased turbulence level in the flow as streamwise vortices are produced from the circular cylinder-strip plate system \citep{Koide2017}. Streamwise vortices produced from the two-cylinder system however, is less turbulent and thus one finds less distortion in the vortical structures compared to the circular cylinder-strip plate cruciform.

Distortion of the vortical structures discussed in \citet{Kato2007} and \citet{Koide2017} are self-induced: the result of an increasing level of turbulence in the flow. An increased level of turbulence is something that is internal to the flow being considered. It remains to be seen whether external factors such as the type of medium used as the ambient fluid, or configuration of the experimental rig can affect the shedding behaviour of the streamwise vortices. The work by \citet{Nguyen2010} partially answers this question. In their study, the two-circular cylinder cruciform is tested under various settings. They tested the system in wind and water tunnels of various sizes and ascribing different dimensions to the cruciform. The dimensions \citet{Nguyen2010} vary in their experiments are the circular cylinder diameter, cylinder length and aspect ratio. At the end of their study, \citet{Nguyen2010} concluded that nearly all major characteristics of streamwise vortex shedding are observed under the various test conditions. There are, however, some discrepancies namely reduction by 30\% to 50\% of streamwise vortex shedding St in the largest wind tunnel, and the vanishing of the necklace vortex in the water tunnel as $\re > 22400$.

Clearly, streamwise vortex shedding that results from the deployment of a cruciform structure is robust and induces as much force on the upstream cylinder as Karman vortex shedding on an isolated cylinder, if not more. This makes the cruciform structure a good oscillator candidate for the purpose of converting energy from the free stream in to vibration energy. In the next section, the author reviews the recent developments in VIV and discusses how a cruciform oscillator ties into the overall work done to induce a larger, but predictable vibration from fluid flows, including that which depends on Karman vortex.

\section{Vortex-induced Vibration of a Cylinder} \label{sec:cylinderVIV}
\subsection{Single Cylinder Oscillator Unit} \label{ssec:singleCylinderOscillator}
Since Karman vortex shedding has been demonstrated to produce alternating lift and drag on an isolated cylinder system, replacing motion constraints with elastic supports allows oscillatory motion to take place. Typically, the oscillation is constrained to a one-dimensional transverse motion against the direction of the free stream. Following the success of \citet{Raghavan2007} and  \citet{Bernitsas2008a,Bernitsas2008b} in achieving a high amplitude response from an isolated circular cylinder oscillator, experimental studies for such an oscillator are increasingly done in the TrSL3 regime of \citet{Zdravkovich1997}. The TrSL3 regime implies working in the $2 - 4 \times 10^{4} \leq \re \leq 1 - 2 \times 10^{5}$ range.

Numerical studies provide a non-intrusive method to collect high-resolution data on flow quantities. Following the trend in experimental studies of VIV, an increasing number of works have been pushing through the $\re \sim O(10^{5})$ barrier, and collect time-resolved flow quantities that lay the groundwork for advancing VIV flow theory. Additionally, if one is able to develop a working numerical model of VIV at a particular range of Re that is in good agreement with experimental findings, scholars can then use that model to automate the variation of parameters. This automation enables efficient exploration of the parameter space of interest and rapid product prototyping.

However, for numerical works to accurately depict what is going on in the flow at $\re \sim O(10^{5})$, investigators need to abandon one of the most commonly made assumption to simplify the KVIV phenomenon: that the flow is two-dimensionally reducible. Reducing KVIV to a two-dimensional domain greatly reduces computational requirements and allows the researcher to focus the computational effort on resolving the shear layer of the cylinder and the near-wake region. However, three-dimensional effects exists throughout the flow field as soon as $\re > 150$, and an accurate representation of KVIV necessitates taking these effects into account.

\citet{Kinaci2016} proposed a tip-flow correction factor to reduce deviation of a two-dimensional CFD simulation from experimental results. The tip-flow correction factor is computed from three-dimensional simulation of flow around a cylinder within the Re range of interest, in essence providing us with a new ``effective length'' of the cylinder. The success of this method is however quite limited as it only scales the lift magnitude from a two-dimensional simulation, but not the phase lag between the lift and cylinder displacement. This means that to conduct a meaningful study of KVIV in the TrSL3 regime ($2 - 4 \times 10^{4} \leq \re \leq 1 - 2 \times 10^{5}$) one must rely on experiments and three-dimensional simulations to explore the bulk of parameter space.

One of the recent developments in the control of KVIV is by affixing passive turbulent stimulations on the surface of the cylinder. For example, \citet{Park2016} deploys surface roughness on the cylinder and observed their effect on KVIV. They identified several location on the cylinder that produces weak to high reduction in the amplitude response. They reasoned that this is due to the reduction of vorticity in the shear layer that forestalls vortex roll-up, hence trim the vibration amplitude by at least a factor of three. This forestalling of vortex shows that surface roughness on the scale of the shear layer thickness can alter the vorticity and thus the lift production process of the cylinder.

The altering of vorticity in the shear layer is also what enables the bridging of KVIV and galloping in single cylinder systems with surface roughness. At low angles of attachment of the roughness strip \citep{Chang2011,Park2013}, the flow separates at the strip upstream but reattaches further downstream the cylinder surface. At high angles of attachement, there is not enough distance along the cylinder surface for the separated flow to reattach, hence vorticity cancellation between Karman vortices shed from the top and bottom of the cylinder becomes comparatively more aggressive. The absence of this aggressive cancellation of spanwise vorticity is the reason why placement of the roughness strip is able to, in contrast to \citet{Bernitsas2008c} and \citet{Park2016}, enhance KVIV and improve the amplitude response of the single cylinder oscillator.

In \S\ref{ssec:svShedding}, this work discussed how the cruciform system has its origin as a simple method that can be retrofitted to an existing single cylinder oscillator unit to control Karman vortex shedding. As such, the effect of the downstream cylinder/strip plate is targeted towards the upstream cylinder and not itself. However, within the last five years, the VIV research community has increasingly realised that there is no reason for the conversion of energy from the free stream into vibration to strictly be a one-stage process. Excess energy that is not eludes capture by a single cylinder should simply be recovered with another cylinder downstream.

Bearing this in mind, \citet{Ding2017} investigated the system response of a two-circular cylinder system, one in front of the other and with their axes parallel to each other. Then, they measured the amplitude/frequency responses of the system and observed the following, compared against a single cylinder system with passive turbulence control (PTC). The amplitude response of the upstream cylinder seems to improve in this serial arrangement of cylinders, and they experience only a small influence from centre-to-centre distance variation. On the other hand, the downstream cylinder experiences a reduction in the amplitude response compared to a single cylinder system and is more sensitive to the centre-to-centre distance between the two cylinders. The reduced amplitude response of the downstream cylinder is the result of the acquisition of downstream cylinder vorticity by the upstream cylinder in proximity. Proximity between the two cylinders determine how much downstream cylinder vorticity is carried away by the upstream vortex, and this reduces the net lift acting on the downstream cylinder. The reduction of lift finally results in the reduction of downstream vibration amplitude.

There is however, a countermeasure to improve the amplitude response of the downstream cylinder. The improvement is achieved by imposing a smaller damping on the downstream cylinder \citep{Xu2017}. When \citet{Xu2017} reduced damping for the downstream cylinder, the amplitude response of the downstream cylinder becomes more similar to the amplitude response of the upsteram cylinder. Nevertheless, because harnessed power is proportional to damping - specifically damping in the mechanical to electrical power converter - a reduced total damping just to achieve a higher amplitude vibration is simply contrary to the engineer's main objective of energy harvesting \citet{Bernitsas2008a,Bernitsas2009}.

Spacing between the two cylinders play an important role in obtaining maximum vibration amplitude from both the upstream and downstream cylinders. This much is already known from the works of \citet{Park2016} and \citet{Ding2017}. However, recently, \citet{Yuan2020} utilised the Hilbert transform on the cylinder displacement time series of both upstream and downstream cylinders. In doing so, they identified five major responses of the two-cylinder oscillator that depends primarily on the centre-to-centre distance between the two cylinders, and secondarily on other parameters such as spring coefficient and total damping.

The use of Hilbert transform enabled \citet{Yuan2020} to compute instantaneous frequency and phase lag between the upstream and downstream cylinders and identified the five major response regimes of the system. Although \citet{Yuan2020} have essentially demonstrated the utility of Hilbert transform in analysing complex and interdependent systems of vibration, the signals they analysed are for the most part monocomponent signals with more or less a symmtetrical signal envelope. For signals that lack these features, the pre-processing step of empirical mode decomposition (EMD) or ensemble empirircal mode decomposition (EEMD) is necessary to produce meaningful physical results \citep{Huang2014,Chen2019}.

\subsection{Cruciform Oscillator Unit}
Streamwise vortex-induced vibration (SVIV) is a type of vortex-induced vibration (VIV) driven by vortical structures whose vorticity vector points in the direction of the free stream. In its early days, the cruciform oscillator is nothing but two circular cylinders perpendicular to each other, overlapping at their midpoints \citep{Zdravkovich1981,Zdravkovich1983,Zdravkovich1985}.

A cruciform oscillator unit derives its vibration from the streamwise vortex pairs discussed in \S\ref{ssec:svShedding}. Investigators observed the upstream cylinder acquiring significant vibrations in the order of $O(10^{-1}D)$, as the reduced velocity of the flow $\ured$ exceeds 14. Here $D$ denotes the diameter of the upstream cylinder and the reduced velocity is defined in Eq. \ref{eq:reducedVelocity} as

\begin{equation}
  \ured = \frac{U_{\infty} \fn}{D},
  \label{eq:reducedVelocity}
\end{equation}

\noindent where $U_{\infty}$ and $\fn$ refers to the freestream velocity and the natural frequency of the system, respectively.

In their study of the two-circular cylinder cruciform, \citet{Shirakashi2001} observed the vibration of the upstream cylinder hitting the top boundary of the wind tunnel between $20.5 \leq \ured \leq 35.9$. The height of the wind tunnel in \citet{Shirakashi2001} is 320 mm and the diamter of both upstream and downstream cylinders are 26 mm. The fact that the reported vibration amplitude reaches the top extent of the wind tunnel means that it is able to achieve at least $(\SI{320}{\milli\metre}/2)/\SI{26}{\milli\metre} = 6D$ within $20.5 \leq \ured \leq 35.9$. This alone is far bigger than any of the amplitude attained by the single cylinder oscillator units reviewed in \S\ref{ssec:singleCylinderOscillator}. However, there seems to be a finite range within which the cruciform oscillator is able to sustain this high amplitude response, which is 15.4 units of $\ured$ long. Past this, the amplitude of the cylinder drops to about $0.1D$ and stayed there up to the maximum $\ured$ tested. In contrast, single cylinder oscillators of \S\ref{ssec:singleCylinderOscillator} is able to sustain its vibration through galloping for seemingly an infinite range of $\ured$.

More recent iterations of the system \citep{Koide2007,Kato2007} employs a thin, rectangular plate of width $w = D$ in place of the fixed downstream cylinder, and showed that the region of high-amplitude vibrations of the sytem is enlarged from 15 units of $\ured$ for a two-circular cylinder cruciform to approximately 40 units of $\ured$ for the circular cylinder-strip plate oscillator studied in \citet{Koide2007}. Nevertheless, the shortcoming of this configuration is a lower amplitude of vibration, which attains only a maximum of close to $\yrms = 0.35$. The vibration amplitude in \citep{Koide2007,Kato2007} is given as the \rms{} of nondimensionalised vibration amplitude $\yrms = y_{\text{RMS}}/D$. Here, $y$ denotes the the coordinate pointing in the transverse direction of the flow, i.e. the direction of cylinder vibration. The effect of circular cylinder-strip plate gap $g/D$ on $\yrms$ was also examined, and the results were consistent with the trend of high lift generation discussed in \S\ref{ssec:svShedding}: $\yrms$ jumps to a high-amplitude regime as $g/D$ is decreased to $< 0.5$, especially when $g/D < 0.25$. 

However, at $G = g/D = 0.16$, researchers have found that the circular cylinder-strip plate oscillator managed to sustain its vibration over a very large range of $\ured$. In relative terms, the vibration of this variant of the cruciform is sustained over a range of $\ured$ 15 times the width observed in an isolated, smooth circular cylinder oscillator of similar diameter \citep{Koide2009,Koide2013}. The reason behind this is the vortical structure driving the vibrations: vibration of the cruciform is driven by a pair of counter-rotating streamwise vortex shed a short distance from the cruciform juncture (refer Fig. \ref{fig:oscillatorSchematic}), while vibration of the isolated, smooth circular cylinder is driven by Karman vortices shed alternately from the top and bottom surface of the cylinder.

The upside of this discovery is the fact that this wide synchronisation range is due to vortex shedding i.e., vortex-induced vibration (VIV) instead of galloping, as in \citet{Sun2019b}, \citet{Xu2019} and \citet{Ding2019}, as VIV-based energy harvesters of similar scale operate at higher efficiencies compared to galloping-based ones \citep{Sun2016,Ma2016}.

\begin{figure}
  \centering
  \hspace{1cm} \includegraphics[width=0.7\textwidth]{figs/oscillatorSchematic}
  \caption{Schematic of the base configuration of the oscillator system used in this study, i.e. the pure cruciform. In this configuration, the axis of the cylinder and the strip plate are perpendicular to each other.}
  \label{fig:oscillatorSchematic}
\end{figure}

In their study, \citet{Deng2007} examined the flow field of a twin circular cylinder cruciform using computational fluid dynamics (CFD). Their domain stretches  $28D$  in the streamwise direction,  $16D$  in the transverse direction and  $12D$  in the spanwise direction. They studied an Re range yet another order of magnitude smaller than that studied by \citet{Koide2017}, possibly to get an even clearer visualisation of the vortical structures with less turbulence, and to ease computational requisites. At a fixed  $\re = 150$ , streamwise vortices form even at a gap ratio of $2$. This result differs quite strikingly from \citet{Koide2006,Koide2007}, conducted at an Re twice the order of magnitude of \citet{Deng2007}, an indication that the minimum gap ratio needed for the onset of streamwise varies with respect to Re.

They also observed that when the gap ratio $G$, which they denote as  $L/D$  in their paper, increases from 3 to 4, the maximum amplitude of the lift coefficient increases by almost threefold. This can be attributed quite easily to the current vortex pair shed by the upstream cylinder. The downstream cylinder immediately disturbs the pair shed from the upstream cylinder when  $G=3$. The lift coefficient increases by about a factor of 3 when this immediate disturbance diminishes at  $G=4$. The visualisation of three-dimensional (3D) vorticity isocontours enables the present work to quickly establish this link vis-\`{a}-vis the lift coefficient signal. The authors use of CFD made this possible.

A similar study in the order of magnitude $\re \sim O \left( 10^{2} \right)$ by \citet{Zhao2018a} particularly highlighted the immense utility of CFD as a tool to research SVIV or flow around a cruciform in general. They computed the sectional lift coefficient along the upstream cylinder, and the time history of this sectional lift coefficient revealed two different modes of vortex shedding, namely, parallel and K-shaped. They also paid attention to the local flow patterns that vary along the length of the upstream cylinder such as the trailing vortex flow, necklace vortex flow and flow in the small gap (denoted as SG flow). The discontinuities in the phase angle of the sectional lift coefficient along the upstream cylinder seems to suggest the inadequateness of attributing the lift coefficient to streamwise vortex shedding alone, particularly when Karman vortex streamlines were also observed some distance away from the junction of the cruciform.

\citet{Shirakashi1989} also made a similar observation in their experimental work. These observations lead the present author to hypothesise that the lift signal is more appropriately viewed as the streamwise and Karman vortex-induced composite lift signal. The present work, nevertheless, could neither find studies that took this viewpoint, nor work out its consequence on power generation.

\section{Energy Harvesting from a Vibrating Cylinder} \label{sec:energyHarvesting}
The field of energy harvesting has witnessed exponential growth in the past two decades, particularly those which are based on alternating lift mechanisms. Following the same method used in the previous \S\ref{sec:cylinderVIV}, the author reviews the progress in the single cylinder energy harvesting unit first, followed by the recent advances in the cruciform systems.

\subsection{Single Cylinder System} \label{ssec:singleCylinderHarvester}
Recent advances in energy harvesting from a single cylinder system builds upon the body of work done on single cylinder oscillators. One of the overarching themes in power output improvement from a single cylinder system is the operating range of Re. Ever since the work done by \citet{Bernitsas2008a} and \citet{Bernitsas2009}, prototyping studies have planted their foot firmly in the high lift regime of TrSL3. In doing so, studies such as \citet{Ding2019} and \citet{Park2017} managed to demonstrate an amplitude response with an upper branch sustained over a wider range of $\ured$ compared to amplitude responses of single cylinder systems in TrSL1 or TrSL2. A higher vibration amplitude will elicit a higher harnessed power from the harvester.

Here, TrSL1, TrSL2 or TrSL3 refers to the classification system in \citet{Zdravkovich1997} to characterise the flow around a smooth circular cylinder. The classification system uses abbreviations of the flow description as labels for different flow behaviours. For example, the flow around the circular cylinder undergoes a number of transitions in the shear layer between $350 - 500 \leq \re \leq 1 \times 10^{5} - 2 \times 10^{5}$. The \underline{\textbf{first}} \underline{\textbf{tr}}ansition in the \underline{\textbf{s}}hear \underline{\textbf{l}}ayer, occurring between $350 - 500 \leq \re \leq 1 \times 10^{3} \times 2 \times 10^{3}$, is the development of transition waves. \citet{Zdravkovich1997} labeled this regime as TrSL1, formed by joining the abbreviations of ``transition'', ``shear'', ``layer'', and the numerical equivalent of ``first'', which is the number one (1). Similarly, the second transition in the shear layer occurring between $1 \times 10^{3} - 2 \times 10^{3} \leq \re \leq 2 \times 10^{4} - 4 \times 10^{4}$, which is the formation of free vortices, is labeled as TrSL2. The shear layer becomes fully turbulent between $2 \times 10^{4} - 4 \times 10^{4} \leq \re \leq 1 \times 10^{5} - 2 \times 10^{5}$. This regime is thus labeled TrSL3. 

Now with the single cylinder system firmly planted in the TrSL3 regime, researchers worked to push the single cylinder harvesting system even further by tweaking its mechanical parameters. For example, \citet{Sun2018} and \citet{Ma2018} used as the restorative force, i.e. springs, of the nonlinear variant. So instead of modelling the spring linearly as

\begin{equation}
  F(t) = k x(t),
  \label{eq:linearSpring}
\end{equation}

\noindent where $F(t)$ is the restorative force, $k$ the spring coefficient and $x(t)$ the displacement of the object fixed to the end of the spring, they experimented with $F(t)$ that is given as a cubic function and an $F(t)$ that is given as a piecewise linear function. The cubic spring

\begin{equation}
  F(t) = k y(t) + k y(t)^{3},
  \label{eq:cubicSpring}
\end{equation}

\noindent is found to significantly improve energy harvesting in all branches of amplitude response. Most significantly, however, is the improvement in the transition from VIV lower branch to galloping branch. In systems with a linear restorative force, the VIV lower branch is out of sync with the lift and also vortex shedding. This causes the net power harnessed in this regime to drop up to a factor of three, compared to the power harnessed in the upper branch of VIV \citep{Sun2018,Ma2018}. In contrast, when a cubic restorative force is applied to the system, energy harvesting improves by up to 78\% during the transition from lower branch to galloping. Furthermore, once the system enters the galloping regime, the cubic restorative force improves energy harvesting up to 12\%. In fact, the only branch where the original linear restorative force outperforms the cubic spring system in terms of energy harvesting is the upper branch.

Differently, a piecewise linear spring such as in Eq. \ref{eq:piecewiseLinearSpring}, in essence enables switching of system stiffness and natural frequency according to the cylinder displacement.

\begin{equation}
  F(t)=\begin{cases}
    k_{1} y(t),                              & \text{if $|y(t)| \leq y_{0}$}\\
    k_{1} y(t) y_{0} + k_{2} (y(t) - y_{0}), & \text{if $|y(t)| > y_{0}$}.
  \end{cases}
  \label{eq:piecewiseLinearSpring}
\end{equation}

\noindent Through this modification, the system is able to automatically switch to a lower stiffness when the lift acting on the cylinder is small, and to a higher stiffness otherwise. A higher stiffness results in a higher system natural frequency. Consequently, when the system is in sync with the alternating lift and vortex shedding, thus producing a large vibration amplitude, the frequency of vibration also switches to a larger value. This switch to a larger value is valuable as the harnessed power from such a system is proportional to the square of its angular frequency \citep{Ma2018}.

Apart from this, the piecewise linear spring is able to harness energy even when $\re < 25000$, which is lower than the usual linear spring system of Eq. \ref{eq:linearSpring} \citep{Sun2016}. In terms of damping, although it is proportional to the net harnessed power \citep{Ma2018}, there is a trade-off with the size of the stall zone during transition from VIV to galloping. The width of the stall zone increases with increasing system damping, and this reduces the system's range of operability. Furthermore, in a single cylinder system with a linear spring, simply increasing the damping reduces the maximum vibration amplitude.

Interestingly, this suppression effect on vibration amplitude due to increased damping can be kept in check when one uses the in-tandem arrangement of cylinders. The in-tandem system is similar to that discussed in \S\ref{ssec:singleCylinderOscillator}, and \citet{Sun2019b} observed that the amplitude reduction is not as severe in the in-tandem system compared to the single cylinder unit. This highlights the potential of exploiting the synergy between the two cylinders to counter vibration amplitude reduction that is otherwise striking, provided that the two cylinders are optimally spaced. The results also brought attention to the concept of synergy between single cylinder oscillators in order to harness more energy from the flow.

\citet{Kim2016} took the idea of synergy and investigated a four-cylinder system for energy harvesting. Overall, what they found out was the net energy harvested by the four-cylinder system can exceed the total energy harvested by four isolated cylinder, provided that the four-cylinder system adhere to the optimised system parameters of centre-to-centre cylinder spacing, mass ratio, spring stiffness and damping. They also postulated an energy harvesting unit that consists of stacked arrays of single cylinder oscillators, thus maximising the footprint of the unit and increase power-to-volume density of the system. In summary, \citet{Kim2016} highlighted the need for the research community to widen their perspective and pay greater attention to multi-component energy harvesting systems as there is a high chance that one can discover practical arrangement of the components that increases the net harnessed energy. The next subsection explores one such system: the pure cruciform energy harvester.

\subsection{Pure Cruciform System} \label{ssec:pureCruciformHarvester}
A pure cruciform system is an example of a multicomponent energy harvester where the presence of the downstream plate perpendicular to the upstream cylinder alters the flow such that the vibration response of the upstream cylinder becomes different from the single cylinder oscillator. In such an energy harvesting system, there exists two vortical structures regulating the flow around the cruciform: the Karman and streamwise vortices. However, as the present study discussed in \S\ref{ssec:svShedding}, only the shedding of streamwise vortex is able to synchronise with the alternating lift and cylinder displacement, thus producing a cylinder response fit for energy harvesting.

An example the cruciform system is given in Fig. \ref{fig:oscillatorSchematic}. In broad terms, the literature on SVIV-based energy harvesting from pure cruciform oscillators are either one of these two categories: studies that investigate how the mass ratio, spring stiffness and damping affect the system response, and studies that put into the spotlight the details of flow that regulates the lift acting on the elastically supported cylinder, hence its amplitude/frequency response, which directly influences the system's energy harvesting capabilities.

For example, \citet{Koide2009} investigated such an energy harvesting system by varying a number of parameters defining the vibration-to-electrical energy converter. Damping, similar to single cylinder systems in \S\ref{ssec:singleCylinderHarvester}, decreases the magnitude of amplitude attained by the cylinder during vibration. They pointed out the possibility of reduced harnessed power due to this reduction in vibration amplitude. \citet{Koide2009} also measured the power and current generated by their pure cruciform system. From these measurements, they computed the phase angle between power and current to estimate the power factor of the system. The evolution of the power factor, however, seems to indicate that there is a lot of room for improvement for their vibration-to-electrical energy converter as the power factor varies significantly from reduced velocity to reduced velocity without any prominent trend.

In \citet{Nguyen2012}, the authors attempted to unravel the effect of mass ratio and damping on a pure cruciform oscillator. However, since they are unable to independently vary either of the two parameters, the conclusions one can draw from their study is fairly limited. Taking this handicap into account, they proposed a crude model of the system response with respect to $\ured$ by assuming a strong linearity underpinning the cylinder vibration. According to this model, the synchronisation range of an oscillator is larger when the total damping is higher. This seems to contradict the experimental observations of \citet{Sun2016}, whose data indicates that an increased total damping diminishes the range of $\ured$ capable of sustaining VIV and delayes the onset of galloping. Futhermore, only the experimental data from the single cylinder oscillator of \citet{Nguyen2012} showed some semblance of agreement with the model, but even this agreement is objectively poor as it incapable in capturing the actual trend and magnitude of the experimentally measured amplitude responses. The study did, however, confirm that resonance in VIVs are highly nonliner, while also suggesting the nonlinearity of SVIV to be qualitatively higher than KVIV. 

Such a shortcoming can be overcome using a host of techniques such as the sensor-based virtual spring and damper system \citep{Garcia2018,Sun2018}, application of machine learning to fill in the gaps missing from a full factorial study \citep{Wu2018,Ren2019,Raissi2019,Hu2020}, and quite conventionally through the use of computational fluid dynamics (CFD) \citep{Wu2011c,Zhang2018a}.

In the second focus area, investigators turned their attention to the details of the flow where streamwise vortex shedding occurs. One such study carefully shot motion pictures of the dye-injected flow \citep{Koide2017} at Reynolds number in the order of $\re \sim O\left( 10^{3} \right)$. A lower Reynolds number (Re) reduces the amount of turbulence in the flow, allowing a clearer shot of the vortex structures. Their study also highlights the higher level of turbulence produced by the circular cylinder-strip plate cruciform in contrast to the twin circular cylinder cruciform, which diminishes the periodicity of vortex shedding.

Although visually enlightening, this and other more qualitative studies contribute little towards improving our understanding of the relationship between vortex shedding and the resulting lift. \citet{Deng2007} demonstrated a way to overcome such a shortcoming. In their study at Reynolds number 150, they changed the natural frequency of the system at certain points in their transient simulation and observed how the cylinder amplitude and the force coefficients evolve with respect to the changes. In doing so, the authors found that a higher lift amplitude does not simply translate into a higher amplitude in cylinder displacement. Other factors such as the phase lag between the lift and cylinder displacement plays a significant role in predicting the outcome amplitude of the cylinder. They also verified the previous findings of \citet{Shirakashi1989}, \citet{Koide2009} and \citet{Koide2017}, that points to the coexistence of both Karman and streamwise vortex shedding in a cruciform system. The fact that only one of the two vortex shedding mechanisms actually govern the cylinder's vibration is viewed by this present work as hinting towards an untapped potential in improving the output and efficiency of the cruciform energy harvester.

Despite this, there is a marked lack in efforts to propose and evaluate new types of cruciform oscillators, particularly on the hidden potential of the strip plate as an effective vortical structure regulator. There are, however, some recent studies on the generalisation of the strip plate in the effort to produce steady lift over the surface on the cylinder \citep{Hemsuwan2018a,Hemsuwan2018b,Hemsuwan2018c,Hemsuwan2018d}. In these studies, there are not one, but two circular cylinders placed at the ends of a small rod. The two cylinders placed just upstream the ring plate, with the centre of the small rod coinciding with the centre of the ring plate. The result of the steady lift acting on the cylinders is the rotation of the rod with the two cylinders at its ends and the centre of the ring plate as its centre of rotation - creating essentially a turbine for fluid energy harvesting. Although this is a novel variation on the cruciform and seems very promising, the power output and efficiency is still far behind conventional wind turbines by several orders of magnitude. Apart from these studies, the present work is not aware other attempts to generalise the cruciform oscillator system and further unlock the vortical structure and cylinder vibration-regulating potential of the strip plate, which might hold the key to further improving its power output and efficiency.

\section{Chapter Summary}
This chapter reviewed studies that contribute towards the advancement of vibration energy harvesting from alternating lift mechanisms, in particular the advancement of energy harvesting from cruciform systems. Figure \ref{fig:pyramidOfKnowledge} illustrates how the status quo of knowledge concerning energy harvesting from a cruciform is akin to the tip of the pyramid. Improving the performance of the cruciform energy harvester is heavily dependent on advances in other fields that are more fundamental, such as vortex-induced vibration and vortex shedding. Figure \ref{fig:pyramidOfKnowledge} also exemplifies the relative number of studies between the subjects, with the most number of studies over the years investigating the phenomenon of vortex shedding, especially Karman vortex shedding, followed by studies on the vibration responses of circular cylinder and cruciform oscillators, and finally, studies on energy harvesting using cruciform oscillators and the circular cylinder oscillator which they are based upon.

\begin{figure}
  \centering
    \includegraphics[width=\textwidth]{figs/pyramidOfKnowledge}
    \caption{The pyramid of knowledge for the cruciform energy harvester.}
    \label{fig:pyramidOfKnowledge}
\end{figure}

The main takeaways from each literature reviewed in this chapter is summarised in

\begin{landscape}
  \begin{table}[p]
  \centering
  \caption{Summary of findings in literature and key takeaways.} \label{tab:litRevSummary}
  \vspace{\baselineskip}
  \begin{tabular}{p{4cm}>{\centering}p{2.5cm}>{\centering\arraybackslash}p{15cm}}
    \hline
    \hline

    Literature           & Type of study & Key takeaway \\
    \hline

    \citet{Asano2020} & Numerical & \multirow{5}{\hsize}{Karman vortex shedding can be controlled by modulating the shear layer of the circular cylinder. This is achieved through various methods, e.g., porous surface/outer layer and caviration bubbles.} \\
    \cline{1-2}
    \citet{Ledda2018}, \citet{Durhasan2019}, \citet{Geyer2020}, \citet{Kang2020}  & \multirow{5}{\hsize}{Experimental} & \\
    \hline
    \citet{Gibeau2018}, \citet{Gibeau2019}, \citet{Rai2018} & \multirow{3}{\hsize}{Experimental} & \multirow{3}{\hsize}{Small scale streamwise vortices originate from the three-dimensionality of Karman vortices, and the temporal dynamics of these streamwise vortices are dictated by the shedding dynamics of the Karman vortices.} \\
    \hline
    \citet{Lin2018}, \citet{Zhang2019}, \citet{Koide2017}  & \multirow{3}{\hsize}{Experimental} & \multirow{3}{\hsize}{Flow control due to placement of a downstream body is able to induce the shedding of large scale streamwise vortices, and the temporal dynamics of these streamwise vortices are independent of the shedding dynamics of coexisting Karman vortices.} \\
    \hline
    \citet{Ding2017}, \citet{Xu2017}, \citet{Yuan2020} & \multirow{3}{\hsize}{Experimental} & \multirow{3}{\hsize}{The amplitude/frequency response of a single cylinder oscillator unit can be modulated by tuning the mechanical parameters of the oscillator, e.g., adjusting the spacing between units, spring stiffness, system damping.} \\
    \hline
    \citet{Deng2007}   & Numerical & \multirow{4}{\hsize}{The amplitude/frequency response of a pure cruciform oscillator is heavily influenced by the spacing between the upstream and downstream cylinders. Optimising this parameter allows for very large range of synchronisation and a well-behaved upper limit to the vibration amplitude.} \\
    \cline{1-2}
    \citet{Kato2007}, \citet{Koide2013}, \citet{Zhao2018a}   & \multirow{3}{\hsize}{Experimental} &        \\
    \hline
    \citet{Ma2018}, \citet{Sun2019b} & \multirow{2}{\hsize}{Experimental} & \multirow{2}{\hsize}{In-tandem configuration of single cylinder harvester units are able to synergise and recover a large amount of power. The use of nonlinear springs can enhance this effect.} \\
    \hline
    \citet{Hemsuwan2018a,Hemsuwan2018b,Hemsuwan2018c}  & \multirow{2}{\hsize}{Numerical} &  \multirow{4}{\hsize}{Advances in cruciform energy harvesters are heavily focussed on recording the system responses due to cruciform parameter variation. Numerical studies are disrupting this trend.} \\
    \cline{1-2}
    \citet{Koide2009}, \citet{Nguyen2012} & \multirow{3}{\hsize}{Experimental} &  \\
    \hline
    \hline
\end{tabular}
  \end{table}
\end{landscape}

\chapter{Methodology} \label{chap:method}
\section{Problem geometry} \label{sec:probGeo}
This study bases itself on the work done by \citet{Maruai2017}, \citet{Maruai2018}, and \citet{Koide2013}. In these works, the investigators conducted both experimental and computational investigations of passive control of FSI of cylinders using a strip plate located at the cylinder downstream. Here, the term ``strip plate'' is used as a shorthand for the long, rectangular plate used to control the vibration of the cylinder -- since the plate resembles a strip due to its large aspect ratio. These studies demonstrated the feasibility of energy harvesting using the oscillator system described, in the Reynolds number range $3.6\times10^{3}<\text{Re}<12.5\times10^{3}$. Following this observation, we performed our numerical investigations within a similar \re{} range, albeit slightly widened, to check for variations in the cylinder response in as wide an \re{} range as possible, computational resources permitting.

Our oscillator system derives its geometry from the works of \citet{Nguyen2012}, \citet{Koide2013}, and \citet{Koide2017}. The basic layout of our oscillator system is the pure cruciform: an arrangement where the circular cylinder and the strip plate located downstream have their axes perpendicular to each other. We fixed the gap between the cylinder and the strip plate, $G$, to $0.16$. This value of $G$ was chosen because the cylinder response most suitable for energy harvesting is sustained over the largest range of reduced velocity $\ured$ when $G = 0.16$ \citep{Koide2013}.

\begin{figure}
  \centering
  \begin{subfigure}[h]{0.5\textwidth}
    \includegraphics[width=\textwidth]{figs/problemGeometrySide}
    \caption{}
    \label{fig:probGeoSide}
  \end{subfigure}

  \begin{subfigure}[h]{0.5\textwidth}
    \includegraphics[width=\textwidth]{figs/problemGeometryTop}
    \caption{}
    \label{fig:probGeoTop}
  \end{subfigure}

  \caption{Figure \ref{fig:probGeoSide} shows the cross-chapteral layout of the computational domain, along with key dimensions, when viewed from the side. Figure \ref{fig:probGeoTop} visualises the cross-chapter of the computational domain as viewed from the top. The arbitrarily coupled mesh interface (ACMI) used to connect the domain containing the cylinder with the domain containing the strip plate is placed halfway through the gap, i.e. $0.08D$ downstream the cylinder.} \label{fig:problemGeometry}
\end{figure}

Figure \ref{fig:probGeoSide} visualises our computational domain from its side. We chose these dimensions based on analogous works such as \cite{Maruai2017} and \cite{Maruai2018} which had produced results that agree well with experiments of their own and with \citet{Kawabata2013}. The streamwise coordinates of this domain extends from $-10.5D$ to $10.5D$, and the lateral coordinates from $-10.5D$ to $10.5D$. The coordinate origin $(0,0,0)$, is at the centre of the cylinder and the strip plate is $D/3$ thick.

In Fig. \ref{fig:probGeoTop}, the circular cylinder extends from $z/D=7.5$ to $z/D=-7.5$, giving the computational domain an overall spanwise length of $15D$. The computational domain of a similar study by \citet{Deng2007} has a length of $12D$, upon which the dimension of our domain is based upon. The extra $1.5D$ of spanwise length on either side of our domain is allocated to ensure the full expression of the three-dimensionality of flow structures that appear during the course of our numerical study.

\begin{figure}
  \centering
  \begin{subfigure}[h]{0.3\textwidth}
    \includegraphics[width=\textwidth]{figs/cruciform90}
    \caption{Cruciform layout for \angfi{}}
    \label{fig:cruciform90}
  \end{subfigure}
  \hfill
  \begin{subfigure}[h]{0.3\textwidth}
    \includegraphics[width=\textwidth]{figs/cruciform675}
    \caption{Cruciform layout for \angfo{}}
    \label{fig:cruciform675}
  \end{subfigure}
  \hfill
  \begin{subfigure}[h]{0.3\textwidth}
    \includegraphics[width=\textwidth]{figs/cruciform45}
    \caption{Cruciform layout for \angth{}}
    \label{fig:cruciform45}
  \end{subfigure}

  \raggedright
  \begin{subfigure}[h]{0.3\textwidth}
    \includegraphics[width=\textwidth]{figs/cruciform225}
    \caption{Cruciform layout for \angtw{}}
    \label{fig:cruciform22.5}
  \end{subfigure}
  \hspace{6mm}
  \begin{subfigure}[h]{0.3\textwidth}
    \includegraphics[width=\textwidth]{figs/cruciform00}
    \caption{Cruciform layout for \angon{}}
    \label{fig:cruciform00}
  \end{subfigure}

  \caption{Variation of cruciforms studied in this work. We vary the cruciform angle from the case of a pure cruciform (\angfi{}) to the case of cylinder - plate in tandem (\angon{}), in increments of \angtw{}.}\label{fig:cruciformLayouts}
\end{figure}

We then produce variants of the pure cruciform configuration by rotating the strip plate from \angfi{} to \angon{} in \angtw{} increments. In total, we constructed five different cruciforms shown in Fig. \ref{fig:cruciformLayouts}. The dimensions of the computational domain remain fixed for all cruciforms, including the gap between the cylinder and the strip plate.

\section{Numerical method} \label{sec:numMeth}
Our numerical study utilises OpenFOAM, an open-source computational fluid dynamics (CFD) platform written in C++. With OpenFOAM, we solved the 3D unsteady Reynolds averaged Navier-Stokes (3D URANS) equations that are the following.

\begin{equation}
  \frac{\partial U_{i}}{\partial x_{i}}=0,
  \label{eq:continuity}
\end{equation}

\begin{equation}
  \frac{\partial U_{i}}{\partial t}+U_{j}\frac{\partial U_{i}}{\partial x_{j}} = -\frac{1}{p}\frac{P}{x_{i}}+\frac{\partial}{\partial x_{j}} \left( 2\nu S_{ij}-\overline{u'_{j}u'_{i}} \right).
  \label{eq:navier-stokes}
\end{equation}

The symbols $U$, $x$, $t$, $\rho$, $P$, $\nu$, $S$, and $u'$ denote the mean component of velocity, spatial component, time, density, pressure, kinematic viscosity, mean strain rate and the fluctuating component of velocity, respectively. Equation \ref{eq:sij} gives the mean strain rate $S_{ij}$.

\begin{equation}
  S_{ij} = \frac{1}{2} \left( \frac{\partial U_{i}}{\partial x_{j}} + \frac{\partial U_{j}}{\partial x_{i}} \right).
  \label{eq:sij}
\end{equation}

The turbulence model employed to approximate the Reynolds stress tensor is the Spalart-Allmaras turbulence model. Previous numerical studies on energy harvesting from FIM of circular cylinders have shown reasonable agreement with experiments in the literature through the use of this turbulence model, and thus becomes the basis for the implementation of the same turbulence model in our study \citep{Ding2015a,Ding2015b}. The Boussinesq approximation relates the Reynolds stress tensor $\tau_{ij} = \overline{u'_{j}u'{i}}$ to the mean velocity gradient, exemplified by Eq. \ref{eq:tauij}.

\begin{equation}
  \tau_{ij} = 2 \nu_{T}S_{ij},
  \label{eq:tauij}
\end{equation}

\noindent where $\nu_{T}$ represents the kinetic eddy viscosity. This kinetic eddy viscosity is ultimately expressed as a function whose arguments consist of the molecular viscosity $\nu$, and an intermediate variable $\tilde{\nu}$ that is the solution of Eq. \ref{eq:kineticEddyTransport}. Equation \ref{eq:kineticEddyTransport} incorporates empirically obtained constants to provide closure to the equations governing our numerical investigation. We list the empirical constants that make up Eq. \ref{eq:kineticEddyTransport} in Table \ref{tab:spalart-Allmaras}.

\begin{equation}
  \label{eq:kineticEddyTransport}
  \frac{\partial \tilde{\nu}}{\partial t} + U_{j} \frac{\partial \tilde{\nu}}{\partial x_{j}} = c_{b1}\tilde{S}\tilde{\nu} - c_{w1} f_{w} \left( \frac{\tilde{\nu}}{D} \right)^{2} + \frac{1}{\sigma} \left\{ \frac{\partial}{\partial x_{j}} \left[ \left( \nu + \tilde{\nu} \right) \frac{\partial \tilde{\nu}}{\partial x_{j}} \right] c_{b2} \frac{\partial \tilde{\nu}}{\partial x_{i}} \frac{\partial \tilde{\nu}}{\partial x_{i}} \right\}
\end{equation}

\begin{table}[!ht]
\centering
\caption{Empirical constants used in the Spalart-Allmaras turbulence model.} \label{tab:spalart-Allmaras}
\vspace{\baselineskip}
\begin{tabular}{l c}
  \hline
  \hline

  Empirical constants & Value    \\
  \hline

  $c_{b1}$            & $0.01$   \\
  $c_{b2}$            & $0.09$   \\
  $c_{\nu1}$          & $0.01$   \\
  $\kappa$            & $0.1$    \\
  $\sigma$            & $0.162$  \\
  $c_{\omega3}$       & $0.178$  \\
  \hline
  \hline
\end{tabular}
\end{table}

\noindent We refer the interested reader to the original paper by \citet{Spalart1992} and more recent applications of the turbulence model in \citet{Ding2019} and \citet{Sun2019b}. With the turbulence model properly defined, we are finally able to solve Eqs. \ref{eq:continuity} and \ref{eq:navier-stokes} using the SIMPLE-stabilised PISO algorithm native to OpenFOAM, known as the PIMPLE algorithm. 

\section{Dynamic mesh motion} \label{sec:dynMesh}

Cylinder motion in the computational domain due to FIV introduces distortion to the mesh immediately surrounding the cylinder. The simplest way to keep the mesh distortion in check, thus keeping mesh quality within an acceptable level, is by diffusing the amount of warping to the surrounding space. In practice, the surrounding space is the rest of the mesh nodes, and Eq. \ref{eq:laplace} governs the diffusion.

\begin{equation}
  \nabla \cdot \left( \gamma \nabla u \right) = 0.
  \label{eq:laplace}
\end{equation}

In Eq. \ref{eq:laplace}, $u$ and $\gamma$ represents the mesh deformation velocity and displacement diffusion, respectively. In this work, we set the displacement to be diffused according to the inverse quadratic rule $\gamma = 1/l^{2}$. Here, $l$ denotes the distance from the cell centre to the nearest cylinder edge. Then, we solve Eq. \ref{eq:laplace} using the GAMG algorithm and the Gauss-Seidel smoother. Solution of Eq. \ref{eq:laplace} returns an updated value of $u$, and this updated value of $u$ is used to update the position of the mesh nodes according to Eq. \ref{eq:meshNodeUpdate}. The PIMPLE solver resumes the solution of the 3D URANS equations after we update the mesh node positions.

\begin{equation}
  x_{\text{new}} = x_{\text{old}} + u \Delta t
  \label{eq:meshNodeUpdate}
\end{equation}

For most numerical studies of FSI, the mesh warp diffusion method governed by Eq. \ref{eq:laplace} serves as an adequate workaround to conserve mesh quality. However, this requires ample number of ambient mesh nodes acting as the receiving end of the diffusion algorithm. In our case, the small gap between the cylinder and strip plate ($G = 0.16$) pose a serious limitation to our ability to diffuse the amount of warp introduces by the displacement of the cylinder, since a small space means that we can only allocate a proportionate number of mesh nodes in said gap. Sole reliance on the warp diffusion algorithm will hamper our effort to preserve mesh quality as a high concentration of warp remains within the gap. To overcome this problem, we implement the arbitrarily coupled mesh interface (ACMI) halfway through the gap (see Fig. \ref{fig:problemGeometry}). This technique allows adjacent cells to slide over each other precisely at the $x = 0.13$ plane, ridding us of the requirement for mesh warp diffusion. In the literature, ACMI is also known as the generalised grid interface, or GGI \citep{Zhang2018,Sun2019b}.

\section{Open flow channel experiment} \label{sec:openFlowExp}

As part of the validation process for our numerical setup, we constructed a closed loop open flow channel, with a test chapter \SI{100}{\milli\metre} wide, \SI{200}{\milli\metre} high and \SI{1500}{\milli\metre} long. The design of this open flow channel is heavily inspired by the water tunnel of \citet{Nguyen2012} and \citet{Koide2013}. Considering the application of this research in the far future is in open flows such as natural drainage systems or the ocean - and not within pipes - prompts us to make this distinction.

We benchmark the open flow channel by setting up a pure cruciform oscillator (\angfi{}) experiment, whose data from similar studies are readily available in published works. Following this, we dimensioned the rig to follow the parameters used in \citet{Koide2013}. A summary of our parameters and those used in \citet{Koide2013} are provided in Table \ref{tab:expParameter}. We tune the parameters governing the amplitude/frequency response of the oscillator using simple length-based mechanism as follows (see Fig. \ref{fig:rigSketch}). To tune the spring coefficient $k$, we simply adjust the active length of the twin spring plate. In practice, we obtained the calibration curve of the twin spring plate by performing a weight - displacement measurement \citep{Sun2016} at several active lengths of the plate. Once the spring coefficient versus spring plate active length calibration curve is obtained, we can just adjust the length of the spring plate to achieve the desired value of $k$.

\begin{figure}
  \centering
  \begin{subfigure}[h]{0.5\textwidth}
    \includegraphics[width=\textwidth]{figs/rigSketch}
    \caption{}
    \label{fig:rigSketch}
  \end{subfigure}

  \begin{subfigure}[h]{0.35\textwidth}
    \includegraphics[width=\textwidth]{figs/damperSketch}
    \caption{}
    \label{fig:damperSketch}
  \end{subfigure}

  \caption{Our experimental system used to validate our numerical study. Figure \ref{fig:rigSketch} presents a 3D schematic of the open channel test chapter with a pure cruciform oscillator setup, while Fig. \ref{fig:damperSketch} shows a magnified schematic of the damping system.} \label{fig:experimentalSetup}
\end{figure}

Tuning the total damping of the system and consequently the multiple expressions of damping such as the logarithmic damping $\delta$, Scruton number Sc, or the damping coefficient is done by attaching, as shown in Fig. \ref{fig:rigSketch}, a T-shaped plate made from aluminium into a claw-shaped casing that houses neodymium magnets at its ends. As presented in Fig. \ref{fig:damperSketch}, the method we use to control the strength of the magnetic field exposed to the T-shaped plate is by fixing the insertion depth of the T-shaped plate into the casing. The magnetic field serves to dissipate the kinetic energy of the T-shaped plate that moves with the cylinder during FIM, providing system damping.

\begin{table}[!ht]
\centering
\caption{Summary of experimental parameters in contrast to those used in the experimental work of \citet{Koide2013}.} \label{tab:expParameter}
\vspace{\baselineskip}
\begin{tabular}{l c c}
  \hline
  \hline
                                           & Current study & \citet{Koide2013}\\
  \hline
Cylinder diameter, $D$ (m)                 & $0.01$        & $0.01$           \\
Cylinder length, $l_{\text{cylinder}}$ (m) & $0.09$        & $0.098$          \\
Strip-plate width (m)                      & $0.01$        & $0.01$           \\
Strip-plate length (m)                     & $0.1$         & $0.1$            \\
Effective mass, $m_{\text{eff.}}$ (kg)     & $0.162$       & $0.174$          \\
Logarithmic damping, $\delta$              & $0.178$       & $0.24$           \\
Scruton number, Sc                         & $9.94$        & $7.74$           \\
System natural frequency, $f_{n}$ (Hz)     & $4.42$        & $4.4$ to $4.79$  \\
  \hline
  \hline
\end{tabular}
\end{table}

A voltage controller regulates the power driving the $3.728$ kW (5 hp) centrifugal pump. To set the freestream velocity in the open flow channel, we placed an acoustic Doppler velocimeter (ADV) sampling at in an empty test chapter, filled with plain tap water to a height of \SI{100}{\milli\metre}, on the centreline of the channel, as pictured in Fig. \ref{fig:channelSchematic}. The height of \SI{100}{\milli\metre} is also the water level we conduct our experiments in. We keep the water level at this height of \SI{100}{\milli\metre} during all data collections to achieve a flow ambience analogous to our benchmark study of \citet{Koide2013}, facilitating comparison between the two. Then, we sampled the velocity of the flow at different input voltages by the voltage controller, the final product being an input voltage $V_{\text{in}}$ (V) versus centreline velocity $U_{\text{cent.}}$ calibration curve. This calibration curve allows us to set the freestream velocity of the open flow channel by specifying the input voltage to the pump. The finished product gave an operability range between \uth{} and \uel{}, which translates to $\urth \leq \ured \leq \urel$ for an circular cylinder of diameter \SI{10}{\milli\metre}. The turbulence level ranges between $5\%$ to $8\%$ when the freestream velocity $U_{\infty} \geq \uei$.

\begin{figure}
  \centering
  \includegraphics[width=0.6\textwidth]{figs/channelSchematic}
  \caption{The side view of our test chapter. For a more valid benchmarking of our open channel flow with a similar system in \citet{Koide2013}, we keep the water level to \SI{100}{\milli\metre}.}
  \label{fig:channelSchematic}
\end{figure}

We measured the cylinder displacement $y$ as a function of time by placing a visual marker on the support plate of the cylinder (see Fig. \ref{fig:rigSketch}) and capturing the motion of the marker using a video camera positioned perpendicular to the support plate. The motion of the marker is then analysed using \textit{Tracker}: a motion analysis tool built on the Open Source Physics Java framework (for recent implementation examples, see \citet{Wen2020}  or \citet{Krishnendu2020}).

For the benchmarking, we chose the reduced velocity $\ured = \urte$, as the cylinder at that $\ured$ produces a large and stable displacement that simplifies on our part, the measurement and comparison process between our experimental system and \citet{Koide2013}. A sample of the normalised displacement -- $\ystr = y/D$ -- measured as a function of time is illustrated in Fig. \ref{fig:sampTimeHist}. This time series allows us to also compute the normalised cylinder vibration frequency, $\fstr = \fcyl/\fn$ ($\fcyl$ being the vibration frequency of the cylinder). The $\ystr$ data presented in Fig. \ref{fig:sampTimeHist} returns $\ystr = 0.33 \pm 0.03$ and $\fstr 1.03 \pm 0.04$, after computing the uncertainty from multiple experimental runs. In their work, \citet{Koide2013} obtained $\ystr = 0.32$ and $\fstr = 1.09$ at a similar $\ured$ - values that are well within the measurement uncertainty of our experiment. This provides a basis for our reliance on results obtained from the experimental system later in the study.

\begin{figure}
  \centering
  \includegraphics[width=0.41\textwidth]{figs/figure5}
  \caption{The normalised cylinder displacement measured as a function of time at $\ured = \urte$. The experiment was repeated several times to estimate the uncertainty of the measured quantities $\ystr$ and $\fstr$.}
  \label{fig:sampTimeHist}
\end{figure}




\chapter{Numerical setup validation} \label{chap:numSetup}
\section{Richardson extrapolation and the grid convergence index (GCI)} \label{sec:richExtrap}
In this work, we establish the grid independency of the solution using the Richardson extrapolation and the grid convergence index (GCI) \citep{Richardson1927,Stern2001}. The Richardson extrapolation and GCI provides us with a procedure to quantitatively measure on the degree of convergence for the quantities of interest in a numerical study. This method also forces us to pay attention to the trend of convergence of the quantities of interest, requiring a monotonic convergence before proceeding to data collection \citep{Stern2001,MatAli2011,Ali2012,Maruai2018}.

Let $f_{1},f_{2},f_{3},\dots,\fk$ be the quantity of interest obtained from several grid resolutions. We assign a larger subscript for a coarser grid, thus ascribing $f_{1}$ to the finest and $\fk$ to the coarsest grid. Let the difference between successive solutions be $\epsilon_{2,1},\epsilon_{3,2},\epsilon_{4,3},\dots,\epsilon_{n,n-1}$, where $\epsilon_{2,1} = f_{2} - f_{1}$, $\epsilon_{3,2} = f_{3} - f_{2}$ and so on. Then, the GCI is defined as

\begin{equation}
  \text{GCI}_{i+1,i} = F_{s} \frac{\left |\epsilon_{i+1,i} \right |}{f_{i} \left ( r^{p} - 1 \right )} \times 100\%,
  \label{eq:gci}
\end{equation}

\noindent where $F_{s}$, $f_{i}$ and $r^{p}$ denotes the safety factor $\left ( = 1.25 \right )$, quantity of interest and the refinement ratio, $r$, between successive grids raised to the order of accuracy of the series of solution, $p$. We refer the reader to \citet{Stern2001,Langley2018} for a more detailed discussion on $r^{p}$.

We can estimate the limit of the solution as the spacing between grid points approache zero via the $\text{p}^{\text{th}}$ method. In essence, we compute the generalised Richardson extrapolation of the quantity of interest as follows.

\begin{equation}
  \fre = f_{1} + \frac{f_{1} - f_{2}}{\rp - 1},
  \label{eq:richardsonExtrapolation}
\end{equation}

\noindent where $\fre$ is the Richardson extrapolation of the quantity of interest. Using $\fre$ to estimate the limit of the monotonically convergent series of $f_{i}$, we can determine the percentage difference of our solution on our finest grid from this limit as

\begin{equation}
  E_{i} = \frac{f_{i} - \fre}{\fre} \times 100\%.
  \label{eq:percentageDifference}
\end{equation}

Table \ref{tab:gridIndependency} summarises the result of our grid independency study for the SVIV reduced velocity of $\ured = 22.7$. The flow global quantities we check for convergence are the vibration amplitude, vibration frequency and lift coefficient of the cylinder. To simplify our grid independency study, we chose the pure cruciform configuration (the \angfi{} cruciform) as the representative, and collected data at $\ured = \urte$ on three sets of grid numbered $1$ for the finest, $2$ for the medium and $3$ for the coarsest, shown in Fig. \ref{fig:convergenceStudy}. With $v_{i}$ as the volume of the $i^{\text{th}}$ cell in the grid, and $N$ the total number of cells in our domain, the average cell size becomes

\begin{figure}
  \centering
  \begin{subfigure}[h]{0.3\textwidth}
    \includegraphics[width=\textwidth]{figs/figure6a}
    \caption{Coarse}
    \label{fig:coarseMesh}
  \end{subfigure}
  \hfill
  \begin{subfigure}[h]{0.3\textwidth}
    \includegraphics[width=\textwidth]{figs/figure6b}
    \caption{Medium}
    \label{fig:mediumMesh}
  \end{subfigure}
  \hfill
  \begin{subfigure}[h]{0.3\textwidth}
    \includegraphics[width=\textwidth]{figs/figure6c}
    \caption{Fine}
    \label{fig:fineMesh}
  \end{subfigure}

  \caption{Three meshes used in the grid convergence study. Figures \ref{fig:coarseMesh}, \ref{fig:mediumMesh} and \ref{fig:fineMesh} show the coarse, medium and fine meshes viewed perpendicular to three main viewing positions: from the inlet, the top and the front, which is looking directly at the cylinder end.} \label{fig:convergenceStudy}
\end{figure}

\begin{equation}
  h = \frac{1}{N} \left [ \sum_{i=1}^{N} v_{i} \right ]^{1/3},
  \label{eq:averageCellSize}
\end{equation}

\noindent and the normalised average cell size is hence 


\begin{equation}
  h/D = \frac{1}{ND} \left [ \sum_{i=1}^{N} v_{i} \right ]^{1/3}.
  \label{eq:normAveCellSize}
\end{equation}

\noindent Both $\yrms$ and $\clrms$ (see Figs. \ref{fig:gciYrms} and \ref{fig:gciClrms}) have initial values smaller than their Richardson extrapolations, $\fre$, before approaching $\fre$, with decreasing $h$. The vibration frequency, on the other hand, starts at a value larger than its $\fre$ before approaching $\fre$, as one can see in Fig. \ref{fig:gciFstr}.

The most significant drop in GCI is experienced by $\clrms$, with increasing refinement of the grid. Refinement from the coarse to medium grid returns a GCI of $30.92\%$, using a refinement ratio of $1.376$. We compute the refinement ratio by dividing the number of cells in one grid with the grid one stage refined. Generalising this to $i$ number of grids returns

\begin{equation}
  r_{i+1,i} = \frac{S_{\text{grid},i+1}}{S_{\text{grid},i}},
  \label{eq:refinementRatio}
\end{equation}

\noindent where $S_{\text{grid},i}$ denotes the total number of cells in the $i^{\text{th}}$ grid. The GCI of $\clrms$ decreases further to $1.63\%$ with further refinement of the grid. On the other hand, GCI for $\fstr$, shrinks to about one-sixth of its former value.

\begin{table}[!ht]
\centering
\caption{Summary of grid independency study.} \label{tab:gridIndependency}
\vspace{\baselineskip}
\begin{tabular}{l c c c}
  \hline
  \hline
Parameter/ metric                                                       & $\clrms$       & $\yrms = \ystr/D$ & $\fstr = \fcyl / \fn$ \\
  \hline
$\fre$                                                                  & $0.262$        & $0.369$           & $0.969$               \\
$f_{1}$                                                                 & $0.2598$       & $0.3687$          & $0.9695$              \\
$f_{2}$                                                                 & $0.2430$       & $0.3588$          & $0.9740$              \\
$f_{3}$                                                                 & $0.0805$       & $0.2374$          & $1.0220$              \\
$\left | \epsilon_{2,1} \right |$                                       & $0.02$         & $0.01$            & $0.004$               \\
$\left | \epsilon_{2,1} \right |$                                       & $0.16$         & $0.12$            & $0.48$                \\
$R = \left | \epsilon_{2,1} \right | / \left | \epsilon_{2,1} \right |$ & $0.10$         & $0.08$            & $0.094$               \\
$\text{GCI}_{3,2}$                                                      & $30.92$        & $6.00$            & $0.64$                \\  
$\text{GCI}_{3,2}$                                                      & $1.63$         & $0.52$            & $0.10$                \\
  \hline
  \hline
\end{tabular}
\end{table}

Inspecting Figs. \ref{fig:gciYrms}, \ref{fig:gciFstr} and \ref{fig:gciClrms}, we find the quantities of interest to be very close to its Richardson extrapolation at the fine grid (grid 1) for all $\clrms$, $\yrms$ and $\fstr$. We take this as an indication of sufficient spatial discretisation. At this point, we find the trade-off between a solution even closer to the Richardson extrapolation and the increased computational effort no longer appealing, compounded by our observation that values of $\yrms$ and $\fstr$ at the fine grid already fall within experimental uncertainty as evidenced by our measurement in \S \ref{sec:openFlowExp} and the work by \citet{Koide2013}. We use the fine grid in all simulations of the pure cruciform case, and implemented a similar mesh resolution to all cruciform variants studied in this work.

\begin{figure}
  \centering
  \begin{subfigure}[h]{0.38\textwidth}
    \includegraphics[width=\textwidth]{figs/gciYrms-1}
    \caption{}
    \label{fig:gciYrms-1}
  \end{subfigure}
  \hspace{6mm}
  \begin{subfigure}[h]{0.38\textwidth}
    \includegraphics[width=\textwidth]{figs/gciYrms-2}
    \caption{}
    \label{fig:gciYrms-2}
  \end{subfigure}

  \caption{The convergence diagram for $\yrms$. Figure \ref{fig:gciYrms-1} shows how $\yrms$ converges close to the Richardson extrapolation value while Fig. \ref{fig:gciYrms-2} shows how the error (difference between the value obtained from a particular mesh and the Richardson extrapolation) decreases with decreasing grid spacing.} \label{fig:gciYrms}
\end{figure}

\begin{figure}
  \centering
  \begin{subfigure}[h]{0.39\textwidth}
    \includegraphics[width=\textwidth]{figs/gciFstr-1}
    \caption{}
    \label{fig:gciFstr-1}
  \end{subfigure}
  \hspace{6mm}
  \begin{subfigure}[h]{0.39\textwidth}
    \includegraphics[width=\textwidth]{figs/gciFstr-2}
    \caption{}
    \label{fig:gciFstr-2}
  \end{subfigure}
  \caption{The convergence diagram for $\fstr$. Figure \ref{fig:gciFstr-1} shows how $\fstr$ converges close to the Richardson extrapolation value while Fig. \ref{fig:gciFstr-2} shows how the error (difference between the value obtained from a particular mesh and the Richardson extrapolation) decreases with decreasing grid spacing.} \label{fig:gciFstr}
\end{figure}

\begin{figure}
  \centering
  \begin{subfigure}[h]{0.39\textwidth}
    \includegraphics[width=\textwidth]{figs/gciClrms-1}
    \caption{}
    \label{fig:gciClrms-1}
  \end{subfigure}
  \hspace{6mm}
  \begin{subfigure}[h]{0.39\textwidth}
    \includegraphics[width=\textwidth]{figs/gciClrms-2}
    \caption{}
    \label{fig:gciClrms-2}
  \end{subfigure}
  \caption{The convergence diagram for $\clrms$. Figure \ref{fig:gciClrms-1} shows how $\clrms$ converges close to the Richardson extrapolation value while Fig. \ref{fig:gciClrms-2} shows how the error (difference between the value obtained from a particular mesh and the Richardson extrapolation) decreases with decreasing grid spacing.} \label{fig:gciClrms}
\end{figure}


\chapter{Streamwise vortex-driven vibration}\label{chap:svivRegime}

\section{The amplitude and frequency response}\label{sec:svivRegimeAmpFreqResp}

The pure cruciform case, i.e. \angfi{}, demonstrated a normalised \rms{} amplitude of cylinder displacement, $\yrms$ that starts quite expectedly with a low amplitude at reduced velocities $\uron$ and $\urtw$ , before reaching a value close to $\yrms = 0.1$ at $\ured = \urth$, as presented in Fig. \ref{fig:yStrRMS1}. Following the local maximum at $\ured = \urth$, $\yrms$ tapers off to less than $\yrms = 0.05$ between $\urfo \leq \yrms \leq \ursi$. This whole $\yrms$ trend of hitting a local maximum before tapering off bears a striking resemblance to the amplitude response of an isolated circular cylinder in KVIV at mass ratios of order $O(10^{1})$ \citep{Feng1963,Khalak1999}. This resemblance can be seen as an indication that the vibration of a pure cruciform between $\ured \leq \urse$ is driven primarily through the shedding cycle of Karman vortices.

\begin{figure}
  \centering
  \includegraphics[width=0.38\textwidth]{figs/yStrRMS1}
  \caption{Evolution of the normalised \rms{} amplitude of cylinder displacement $\yrms$, with respect to reduced velocity $\ured$, in the streamwise vortex-driven vibration regime.} \label{fig:yStrRMS1}
\end{figure}

Then at $\ured = \urse$, $\yrms$ experiences a very weak increase followed by a sudden jump close to $0.4$ at $\ured = \urei$. This is followed by a slight decline at $\ured = \urni$ and return to the previous level of $\yrms$ at $\ured = \urte$. Past $\ured = \urte$, we observe that $\yrms$ maintains a linear trend in its variation with respect to $\ured$. As $\ured = \urei$ is well within the lower branch for a system in KVIV, it is quite unlikely for the vibration within $\urei \leq \ured \leq \urtt$ to be the governed by the shedding of Karman vortices, leading previous investigators to attribute the vibration to the periodic shedding of streamwise vortical structures dominating the spatial region close to the cruciform juncture \citep{Shirakashi1989,Hemsuwan2018b,Hemsuwan2018d}. Hence, we name this range of $\ured$ the streamwise vortex-induced vibration regime.

\begin{figure}
  \centering
  \includegraphics[width=0.4\textwidth]{figs/expCompareAmp}
  \caption{Comparison between the evolution of $\yrms$ with respect to $\ured$of a pure cruciform system from our numerical and experimental work. The filled square represents the numerical, while the filled circle represents the experimental results.}
  \label{fig:expCompareAmp}
\end{figure}

The experimental system consisting of the closed loop open flow channel and the pure cruciform oscillator rig in \S\ref{sec:openFlowExp} is constructed not only for the purpose of validating the results of our pure cruciform numerical investigation, but also to corroborate in general, the sum total of our numerical setup. Admittedly, the best undertaking would be to perform equivalent experiment for each of the \angfi{}, \angfo{}, \angth{}, \angtw{} and \angon{} configurations, but the scale of such an exercise and subsequent discussion of the results in our opinion, deserves its own treatment separate from the current study. The degree of agreement between the results of our numerical and experimental investigation of the pure cruciform establishes the validity of our numerical setup, which we assume to extend to the rest of the cruciforms. We think that this assumption is somewhat founded because all cruciforms are simulated under similar boundary conditions, mesh resolution and solver algorithm.

Our experiments collect time series data of cylinder displacement $y$, from which the normalised \rms{} amplitude $\yrms$ is computed. Figure \ref{fig:expCompareAmp} compares both our numerical and experimental results of $\yrms$. We observe that both results agree in terms of magnitude and trend of the amplitude response. However, the jump to SVIV occurs at a higher $\ured \approx 19$, translating to a delay of about 3 units of $\ured$. Our numerical and experimental results are also able to capture the slight dip in $\yrms$ following the jump to SVIV, but the occurrence in our experiment is also delayed by about 3 units of $\ured$. This delay can perhaps be attributed to the fact that the raw $y$ time series were measured in succession from the lowest attainable channel flow velocity \uth{} to its highest \uel{} within one experimental run. In contrast, our simulations always start with the cylinder at rest at its neutral position at $t_{0} = \SI{0}{\second}$, with the freestream exactly at set at the desired value $\uon, \utw, \dots, \utt$. Thus, the delays found in our experimental results may simply be the consequence of ``flow memory'', a concept whose analogy can be found in undergraduate experiments to determine the critical Reynolds number transitioning from laminar to turbulent flow in smooth circular pipes. The ``flow memory'' is in our opinion none other than the manifestation of flow inertia due to fluid viscosity, where the flow has a natural tendency to retain its previous state before being overpowered by the flow momentum. This results in the delay found at the jump to SVIV and the local $\yrms$ minimum after the jump.

\begin{figure}
  \centering
  \includegraphics[width=0.38\textwidth]{figs/yStrFreq5}
  \caption{Evolution of the normalised cylinder displacement frequency, $\fstr$, with respect to reduced velocity $\ured$, for the pure cruciform case.}
  \label{fig:yStrFreq5}
\end{figure}

We show the evolution of the normalised cylinder vibration frequency $\fstr$ with respect to $\ured$ in Fig. \ref{fig:yStrFreq5}. Inspecting Fig. \ref{fig:yStrFreq5}, we immediately notice two distinct evolutionary pattern for $\fstr$with a sharp boundary at $\ured = \ursi$. Between $\uron \leq \ured \leq \ursi$, the $\fstr$ trend follows closely the shedding frequency of Karman vortices from an isolated, fixed circular cylinder \citep{Blevins1990}. The Karman vortex shedding frequency is given as an empirical equation in Eq. \ref{eq:karmanSheddingFreq}.

\begin{equation}
  \fvk = 0.198 \left( 1 - \frac{19.7}{\re} \right) DU
  \label{eq:karmanSheddingFreq}
\end{equation}

\noindent Here, $\fvk$, $D$ and $U$ are the vortex shedding frequency, diameter of the isolated circular cylinder and $U$ the freestream velocity respectively. We can easily see how $\fvk$ is a linear function of $U$, and this is what gives rise to the linear pattern of $\fstr$ within $\uron \leq \ured \leq \ursi$. Then, within $\urse \leq \ured \leq \urtt$, $\fstr$ drops close to 1, indicating synchronisation between lift and cylinder vibration. We think this synchronisation is what gives rise to the bigger $\ystr$, compared to $\uron \leq \ured \leq \ursi$.

Inspecting the evolution of \rms{} amplitude of lift coefficient $\clrms$ and the normalised lift coefficient frequency $\fclstr$ with respect to $\ured$ in Fig. \ref{fig:cl90}, provided more evidence supporting the assertion that $\ured = \urse$ is a boudary berween two vibration-driving mechanisms. In fact, the observation at $\urse$ in Fig. \ref{fig:clRMS5} indicates that the SVIV regime is still in its infancy, due to the dip in $\clrms$ at that $\ured$, compared to $\ured = \ursi$. In Fig. \ref{fig:clFreq5}, we also draw a dashed line illustrating $f^{*}_{v} = \fvk/D$, where $\fvk$ is the shedding frequency of Karman vortices from a smooth isolated circular cylinder described in Eq. \ref{eq:karmanSheddingFreq}.

The trend found in $\fclstr$ vs. $\ured$ is very similar to that found in Fig. \ref{fig:yStrFreq5}. We interpret this similarity as an indication of the symmetry of lift produced along the cylinder. Our reasoning stems from the findings of \citet{Zhao2018a}, who revealed how lift is distributed along the upstream cylinder of a two-cylinder \angfi{} cruciform. They did this by computing chapteral lift coefficients along the upstream cylinder. This system produces a symmetric distribution of chapteral lift coefficient, with $Z = 0$ being the plane of symmetry. An asymmetrical distribution of the chapteral lift coefficient may produce a trend in $\yrms$ and $\fstr$ that is dissimilar to those found in $\clrms$ and $\fclstr$, due to the irregular moment acting on the cylinder.

\begin{figure}
  \centering
  \begin{subfigure}[h]{0.38\textwidth}
    \includegraphics[width=\textwidth]{figs/clRMS5}
    \caption{Evolution of $\clrms$ with respect to $\ured$.}
    \label{fig:clRMS5}
  \end{subfigure}
  \hspace{6mm}
  \begin{subfigure}[h]{0.38\textwidth}
    \includegraphics[width=\textwidth]{figs/clFreq5}
    \caption{Evolution of $\fclstr$ with respect to $\ured$}
    \label{fig:clFreq5}
  \end{subfigure}

  \caption{Evolution of the lift coefficient \rms{} amplitude ($\clrms$) and normalised frequency of lift coefficient ($\fclstr$), with respect to reduced velocity $\ured$, for the pure cruciform case. The dashed line in Fig. \ref{fig:clFreq5} visualises the shedding frequency of Karman vortex computed from Eq. \ref{eq:karmanSheddingFreq}.} \label{fig:cl90}
\end{figure}

\section{Main vibration-driving vortical structure}\label{sec:svivRegimeVortStruct}
Recall Figs. \ref{fig:yStrRMS1} and \ref{fig:yStrFreq5}. Out of all thirteen variants of $\ured$ studied in the pure cruciform case, seven within $\urse \leq \ured \leq \urtt$ sustain high-amplitude vibrations with no foreseeable upper limit within our observation window. For a more complete understanding of the mechanism driving the vibration, we need to know what are the vortical structures dominating the flow are and how they interact with each other.

\begin{figure}
  \centering
\includegraphics[width=0.48\textwidth]{figs/probe90YU10}
\caption{Distribution of normalised frequency of vortex shedding, along the span of the cylinder of the pure cruciform at $\ured = \urte$.}
  \label{fig:probe90YU10}
\end{figure}

Let us denote coordinates in our simulation domain in the following manner: $\left( X, Y, Z \right) = \left( \frac{x}{D}, \frac{y}{D}, \frac{z}{D} \right)$. We sampled the $y$-component velocity fluctuations on the $\left ( X, Y \right ) = \left ( 1.96, 0 \right )$ line, along the span of the cylinder in $0.5D$ increments, i.e. at coordinates $\left ( 1.96, 0, 7.5 \right )$, $\left ( 1.96, 0, 7.0 \right )$, \dots, $\left ( 1.96, 0, -7.5 \right )$. The distance $X = 1.96$ from the origin is equivalent to $1D$ downstream the trailing edge of the strip plate, and we chose this location as it is not too close to the cruciform that the vortical structures have not fully formed, and not too far, obfuscating meaningful observation of the structures.

The shedding of vortical structures leave their footprint on the flow field in the form of velocity fluctuations. Our choice of analysing the fluctuations of the $y$-component of velocity is made due to the fact that our oscillator is constrained to move only in the transverse direction. Then, we processed the velocity fluctuations with FFT to obtain the Fourier transform of the fluctuation signals at each spanwise location. The combines Fourier transforms are presented using a colour map in Fig. \ref{fig:probe90YU10}. In this figure, every point is a result of that FFT giving us a spatial understanding of the vortical structures present in the flow. The abscissa and ordinates denotes $\fstr$ and $Z$ coordinates respectively, while the bar legend gives the amplitude of the FFT result.

Through inspection, we immediately notice two frequency bands with high amplitudes namely $\fstr \approx 1$ and $\fstr \approx 4.5$. The locations of these bands are between $3 \leq Z \leq 4.5$ for the former and $4.5 < Z \leq 7$ for the latter. Aided with this visualisation, we can give meaning to the $x$ and $z$-components of vorticity visualised in Fig. \ref{fig:vortStruct90} . The slices in Fig. \ref{fig:vortStruct90} visualise the distribution of the $x$ (streamwise) and $z$ (Karman) components of vorticity at the $X = 1.96$ plane. The plane is viewed from downstream (viewer standing at $X = 1.96$, looking towards the cruciform), and we present the vorticites in units of \si{\per\second}. Furthermore, the visualisations are made when the lift coefficient Cl is at a maximum ($\text{Cl}_{\text{max}}$).

Comparing Fig. \ref{fig:probe90YU10} with Fig. \ref{fig:vortStruct90} suggests that the $\fstr \approx 1$ band is actually due to the shedding of streamwise vortex of a scale close to $1D$ while the $\fstr \approx 4.5$ seems to be due to the shedding of Karman vortices. Contrary to the vortical structure commonly observed in studies of isolated circular cylinders \citep{Deng2007,Kinaci2016,Duranay2020}, in the pure cruciform case, two distinct vortical structures take shape in the flow, namely streamwise and Karman vortices. This is consistent with the findings in \citet{Koide2017} or \citet{Zhao2018a}, where they observed a pair of streamwise vortices on a scale of $\approx 1D$ form in the vicinity of the cruciform juncture, and Karman vortices further away in the spanwise direction. Note that the vibration-driving streamwise vortices forming close to the cruciform juncture exist in pairs: in Fig. \ref{fig:vorx90}, we observe one rotate in the clockwise direction when $Z > 0$ and the other in the counter-clockwise direction when $Z < 0$. What results from this counter-rotating vortex pair is a downward thrust, propelling the cylinder upwards, and consistent with the fact that we visualised the vorticity fields when the Cl is at a maximum. We also observe the core of both streamwise vortices lie approximately on the same $Y$-plane, parallel to the axis of the cylinder.

\begin{figure}
  \centering
  \begin{subfigure}[h]{0.28\textwidth}
    \includegraphics[width=\textwidth]{figs/vorx90}
    \caption{$x$-component vorticity}
    \label{fig:vorx90}
  \end{subfigure}
  \hspace{6mm}
  \begin{subfigure}[h]{0.28\textwidth}
    \includegraphics[width=\textwidth]{figs/vorz90}
    \caption{$z$-component vorticity}
    \label{fig:vorz90}
  \end{subfigure}

  \caption{Dominant vortical structures at $\ured = \urte$ observed in the pure cruciform case. The vorticity slices shown are the $x$ and $y$-component vorticities at $x/D = 1.96D$ ($1D$ downstream the trailing edge of strip plate) plane, viewed orthogonal to that plane from downstream. The vorticities have a unit of \si{\per\second}.} \label{fig:vortStruct90}
\end{figure}

\section{Phase lag between Cl and normalised cylinder displacement} \label{sec:phaseLag90}

In this study, we compute the phase lag $\plag$ by taking the Hilbert transform of both $\ystr$ and Cl signals, as \citet{Khalak1999} did in their VIV study of isolated circular cylinders. However, since Hilbert transform only produce physically meaningful results when used on monocomponent signals \citep{Huang1998,Huang2005,Huang2014}, the signals first needs to be decomposed into components that satisfy the aforementioned condition, referred to in the literature as the intrinsic mode function (IMF). To achieve this, we implement the ensemble empirical mode decomposition (EEMD) \citep{Wu2008}.

\begin{figure}
  \centering
  \includegraphics[width=0.4\textwidth]{figs/phaseLag5}
  \caption{Phase lag $\plag$ (\si{\degree}) between Cl and $\ystr$ when \angfi{}.}
  \label{fig:phaseLag90deg}
\end{figure}

\begin{equation}
  \plag = \frac{1}{T} \int_{0}^{T} \left[ \theta_{\text{Cl}}(t) - \theta_{y}(t) \right] dt.
  \label{eq:phaseLagDefinition}
\end{equation}

Out of $C_{i}$ IMFs for each of $\ystr$ and Cl, we select the ones for computation of instantaneous phase according to the following rule. First, we choose the IMF component of $\yrms$ with the largest \rms{} amplitude to represent the original $\ystr$ signal. Then, we choose the component of Cl with the highest correlation to the IMF component of $\ystr$, to represent the Cl signal. The degree of correlation is determined by computing the cross-correlation between the two.

The characteristic phase angle $\plag$ defined in Eq. \ref{eq:phaseLagDefinition} is what we summarise against $\ured$ in Fig. \ref{fig:phaseLag90deg}. Note that the $\plag$ pattern between $0 \leq \ured \leq \ursi$ resembles that which is found in isolated cylinder systems undergoing KVIV. We also observe that $\plag$ starts to drop when $\ured = \urse$, supporting the view that a fundamental change in vibration-driving mechanism took place at that $\ured$, culminating in the emergence of the initial branch for SVIV at $\ured = \urei$.

\chapter{Temporal evolution of the lift coefficient} \label{chap:tempEvo}

\section{Ensemble empirical mode decomposition and Hilbert transform} \label{sec:eemd}
To obtain a clearer picture of the temporal characteristics of the lift and cylinder displacement signals, we decided to employ the ensemble empirical mode decomposition (EEMD) method \citep{Huang1998,Wu2008} on the signals, and compute their instantaneous phase lag, frequency and amplitude using the Hilbert transform.

The Hilbert transform (HT) has been used in the past to study the instantaneous phase and frequencies of KVIV \citep{Khalak1999}. However, the signal must be monochromatic if we are to obtain a physically meaningful result after applying HT. EEMD is a way to pre-process the signal and get components that (1) have zero mean, and (2) have an equal number of extrema and zero crossings, or they differ only by one. Functions that fulfil these criteria are called intrinsic mode functions (IMF), and they guarantee a physically meaningful result to HT \citep{Gumelar2019,Zhou2019}. Unlike Fourier transform, which is an analytical method of signal decomposition based on circular functions in the complex plane, EEMD is algorithmic, and the processes undertaken can be summarised as follows.

Produce 150 white noise signals of length equal to the original signal and amplitude equal to 0.2 of the standard deviation of the original signal. Then, add to the set of white noises the original signal -- creating 150 variations of the original signal. Following that, we apply the empirical mode decomposition (EMD) algorithm on each of the 150 signals. The EMD algorithm is summarised below.

\begin{enumerate} \label{enumerate:emd}
  \item Construct the envelope of the signal by connecting all maxima/minima with cubic splines. \label{enum:envelope}
  \item Find the local mean of the envelope for the span of the data. \label{enum:localMean}
  \item Find the difference between the local mean and the original data. \label{enum:difference}
  \item Repeat steps \ref{enum:envelope} and \ref{enum:localMean} on the difference in \ref{enum:difference} for ten times \citep{Wu2008}.
\end{enumerate}

The steps above produce a set of intrinsic mode functions or IMFs for each of the 150 variations of the original signal. Then, we average the first IMF component from each of the decomposed original signal variations, to obtain the first EEMD IMF, $C_{1}$, of the original signal. We do the same for the second, third, until the $i^{\text{th}}$ component for each of the 150 original signal variations, thus obtaining $C_{2},C_{3},\dots,C_{i}$.

To compute the phase lag between the characteristic IMFs of the lift coefficient and normalised cylinder displacement, we select the IMF components with the highest correlation to the $\ystr$ signal at that particular $\ured$, to represent the signals, denoted as $\cysys$ for the characteristic normalised cylinder displacement, and $\cclys$ as the characteristic lift coefficient signal. The phase lag, instantaneous frequency and instantaneous amplitude of the signal is subsequently computed by constructing an analytical signal $z \left( t \right)$ from $C_{1},C_{2},\dots,C_{i}$ by computing the Hilbert transform of the IMF, $H_{i}$,

\begin{equation}
  H_{i} \left( t \right) = \frac{1}{\pi} \text{PV} \int\limits_{}^{\infty} \frac{C_{i} \left( \tau \right)}{t - \tau} d\tau,
  \label{eq:hilbertTransform}
\end{equation}

\noindent where PV denotes the Cauchy principal value, and then constructing the analytical signal as follows.
\begin{equation}
  z \left( t \right) = C_{i} \left( t \right) + i H_{i} \left( t \right)
  \label{eq:analiticalSignal}
\end{equation}

\noindent Note that $i$ in Eq. \ref{eq:analiticalSignal} is the complex number.

We refer the reader interested in the details of EEMD and Hilbert transform, also collectively known as the Hilbert-Huang transform (HHT), to the following excellent texts on the subject \citep{Huang2005,Huang2014}.

\section{The KVIV regime (reduced velocity below 13.6 )} \label{sec:phaseLag}
At reduced velocities  $\ured = 2.3$ and 4.5, the phase lags  $\phi$ (deg.) between Cl and  $\ured$ are practically zero throughout the whole observation time. The characteristic IMFs of Cl and  $\ystr$ at $\ured = 4.5$ exemplifies this trend, as showcased in Fig. \ref{fig:tempAnalysisKVIV}. Here, Fig. \ref{fig:tempAnalysisKVIV}a shows the temporal evolution of $\cysys$ and $\cclys$, which are the characteristic IMFs of $\ystr$ and Cl, respectively. Figure \ref{fig:tempAnalysisKVIV}b shows the phase lag between $\cysys$ and $\cclys$, and Fig. \ref{fig:tempAnalysisKVIV}c presents the HHT spectrogram of Cl. The HHT spectrogram visualises the instantaneous frequency and amplitude of the IMF components of Cl. The trend that one notices in Fig. \ref{fig:tempAnalysisKVIV}b is similar to what was observed in \citet{Khalak1999}, a study that also employs the Hilbert transform to obtain the instantaneous phase, albeit without EEMD. The dominant IMF component (IMF component sustaining the highest amplitude throughout the whole observation time) of the lift coefficient has a normalised frequency $\fclstr = f_{\text{Cl}}/\fn$ (Fig. \ref{fig:tempAnalysisKVIV}c) centred at approximately $\fclstr = 0.75$.

\begin{figure}
  \centering
  \includegraphics[width=0.5\textwidth]{figs/tempAnalysisKVIV}
  \caption{Temporal analysis of the lift coefficient and normalised cylinder displacement signal at $\ured = 4.5$. We show $\cclys$ and $\cclys$ side by side in Fig. \ref{fig:tempAnalysisKVIV}a, present the temporal evolution of the phase lag $\phi$ in Fig. \ref{fig:tempAnalysisKVIV}b and show the temporal evolution of the instantaneous frequency of the lift coefficient signal in Fig. \ref{fig:tempAnalysisKVIV}c.} \label{fig:tempAnalysisKVIV}
\end{figure}

Once we enter the upper branch of KVIV at  $\ured = 6.8$, $\phi$ jumps to approximately 110 deg. This jump in $\phi$ is characteristic of the transition to the upper branches as also observed by \citet{Maruai2018}, among others. Both $\cclys$ and $\cysys$ signals are visibly very periodic, and the dominant frequency band of Cl, is centred at $\approx 1$, as one can verify in Fig. \ref{fig:tempAnalysisUpper}c.

\begin{figure}
  \centering
  \includegraphics[width=0.5\textwidth]{figs/tempAnalysisUpper}
  \caption{Temporal analysis of the lift coefficient and normalised cylinder displacement signal at $\ured = 6.8$. We show $\cclys$ and $\cysys$ side by side in Fig. \ref{fig:tempAnalysisUpper}a, present the temporal evolution of the phase lag $\phi$ in Fig. \ref{fig:tempAnalysisUpper}b and show the temporal evolution of the instantaneous frequency of the lift coefficient signal in Fig. \ref{fig:tempAnalysisUpper}c.} \label{fig:tempAnalysisUpper}
\end{figure}

As we increase $\ured$ even further up to $\ured = \ursi$, we see a similar trend for all $\ured = 9.1, 11.4, 13.6$ examined: $\cysys$ and $\cclys$ are both qualitatively very periodic. Their phase lags are very close to $180$ deg., and the dominant Cl frequency bands exhibit a time-averaged value that increases linearly with respect to $\ured$, in a manner that the Strouhal number of Cl is always $\approx 0.16$ on average. We present the representative case of $\ured = \ursi$ in Fig. \ref{fig:tempAnalysisLower}. Note how $\phi$ in this range of $\ured$ varies much less with respect to time, compared to $\phi$ at $\ured = \urth$, and the dominant frequency band of Cl is much narrower compared to the dominant frequency band at $\ured = \urth$, indicating a highly periodic and self-similar oscillation of lift.

\begin{figure}
  \centering
  \includegraphics[width=0.5\textwidth]{figs/tempAnalysisLower}
  \caption{Temporal analysis of the lift coefficient and normalised cylinder displacement signal at $\ured = \ursi$. We show $\cclys$ and $\cysys$ side by side in Fig. \ref{fig:tempAnalysisLower}a, present the temporal evolution of the phase lag $\phi$ in Fig. \ref{fig:tempAnalysisLower}b and show the temporal evolution of the instantaneous frequency of the lift coefficient signal in Fig. \ref{fig:tempAnalysisLower}c.} \label{fig:tempAnalysisLower}
\end{figure}

\section{Transition to SVIV (reduced velocity between 15.9 and 18.2)} \label{sec:transSVIV}
Previously in the $\ured \leq \ursi$ range, we observed that the temporal profile of both Cl and  $\ystr$ are very similar to each other, except that Cl leads $\ystr$ by a certain amount. This similarity in profile supports the assertion that the vibration within $\ured \leq \ursi$ is driven exclusively by the shedding of Karman vortices, which brings the onset of the alternating lift. Analogously, one might expect a similar profile between Cl and $\ystr$ when streamwise vortices drive the vibration. However, this does not seem to be the case.

Once we reach $\ured = 15.9$, we observe that it has become difficult to argue that the profile of $\ystr$ is just a lagged version of the profile of Cl. This is shown in Fig. \ref{fig:tempEvoCompare}a, with the enlarged version in Fig. \ref{fig:tempEvoCompare}b. The profile of Cl looks like the result of several signals in superposition, which one can almost distinguish from the presence of two types of maxima at two different amplitude heights. We put a red dashed line and a red dashed-dot line in Fig. \ref{fig:tempEvoCompare}b as visual cues indicating the two amplitude heights. Decomposing the lift coefficient signal using EEMD reveals partial evidence supporting the compound signal hypothesis.

\begin{figure}
  \centering
  \includegraphics[width=0.4\textwidth]{figs/tempEvoCompare}
  \caption{Temporal evolution of $\ystr$ and Cl at $\ured 15.9$. Figure \ref{fig:tempEvoCompare}b shows an enlarged view of Fig. \ref{fig:tempEvoCompare}a. We can barely spot semblance of two signals with different amplitudes superimposed in the Cl signal in Fig. \ref{fig:tempEvoCompare}b.} \label{fig:tempEvoCompare}
\end{figure}

Once we have decomposed the signal using EEMD, we replot Fig. \ref{fig:tempEvoCompare}a using $\cclys$ and $\cysys$ in Fig. \ref{fig:tempAnalysisTransition}a. One can clearly see that the part of Cl signal responsible for driving the vibration at  $\ured = 15.9$ is embedded in the original Cl signal (Fig. \ref{fig:tempAnalysisTransition}a), and decomposition via EEMD managed to recover this signal, which leads $\cysys$ by approximately 150 deg. on average, throughout the whole observation time (Fig. \ref{fig:tempAnalysisTransition}b). This decline from $\phi \approx 180$ deg. at reduced velocities $\urfo \leq \ured \leq \ursi$, to $\phi \approx 150$ deg. at $\ured = \urse$ is quite sizeable, suggesting a fundamental change in flow dynamics, particularly in terms of vortical structure. Another notable change is the increased temporal variation in $\phi$ from its time-averaged value, in contrast to the evolution of $\phi$ in the range $\urfo \leq \ured \leq \ursi$, which has very little jitter throughout the observation time.

\begin{figure}
  \centering
  \includegraphics[width=0.45\textwidth]{figs/tempAnalysisTransition}
  \caption{Temporal analysis of the lift coefficient and normalised cylinder displacement signal at $\ured = 15.9$. We show $\cclystr$ and $\cysys$ side by side in Fig. \ref{fig:tempAnalysisTransition}a, present the temporal evolution of the phase lag $\phi$ in Fig. \ref{fig:tempAnalysisTransition}b and show the temporal evolution of the instantaneous frequency of Cl in Fig. \ref{fig:tempAnalysisTransition}c.}
  \label{fig:tempAnalysisTransition}
\end{figure}

Inspecting the HHT spectrogram in Fig. \ref{fig:tempAnalysisTransition}c reveals two dominant bands in the frequency domain. The first one, marked with a white continuous rectangular box, is the instantaneous frequency for the IMF component of lift shown in Fig. \ref{fig:tempAnalysisTransition}a, and its mean frequency lies close to the natural frequency of the system ($\fclstr \approx 1$). There is; however, a second band of the frequency with nearly similar amplitude around $\fclstr \approx 3.3$, marked with a white dashed rectangular box. Computing the Strouhal number from this frequency returns a value of $\st = 0.20$, which is very close to the Strouhal number for Karman vortices as predicted by Eq. \ref{eq:karmanSheddingFreq} at the Reynolds number equivalent to $\ured = 15.9$, which is $\re = 7.9 \times 10^{3}$. We thus attribute this second band of frequency as being the footprint left by the shedding of Karman vortices, and the first band as the result of streamwise vortex shedding. Through visual inspection of Fig. \ref{fig:tempAnalysisTransition}c, both of these dominant frequency bands are markedly wider and the individual values are more scattered from their time-averaged values than any of their counterparts within $\ured \leq \ursi$.

The knowledge that Karman vortices continue to exist and shed from a cruciform structure during SVIV is not new in the literature. However, this is the first time the lift signal from a cruciform structure undergoing SVIV has been subjected to EEMD, revealing the signature of the two dominant vortical structures regulating the flow around the cruciform. Although the amplitude size of the instantaneous frequency band due to Karman vortex is comparable to the streamwise vortex, the reason why the cylinder resists locking into its frequency is perhaps that its frequency too distant from the natural frequency of the system $\fn$. The shedding frequency of the streamwise vortex is much closer to $\fn$ and is thus preferred by the cylinder.

We consider the transition to SVIV to be complete at $\ured = 18.2$, when the time-averaged phase lag drops further to $\approx 20$ deg. Figure \ref{fig:tempAnalysisStableInitialBranch}a and \ref{fig:tempAnalysisStableInitialBranch}b documents this observation. The instantaneous phase lag is observed to slip through 360 deg. a little past the two second (\SI{2}{\second}) time stamp. By inspecting Fig. \ref{fig:tempAnalysisStableInitialBranch}a, we found that a little past \SI{2}{\second} is when distortions in the periodicity of $\cclys$ occur. The slipping through 360 deg. was also observed by \citet{Khalak1999} in their work on KVIV, highlighting the quasi-periodic nature of the signal being analysed. There, the slip appeared in \citet{Khalak1999} at the initial branch of KVIV. The overall low value of $\phi$ ($\approx 20$ deg. for the whole observation time at $\ured = 18.2$), coupled with the presence of $\phi$ slippage are suggestive of the possibility for $\ured = 18.2$ being the initial branch of SVIV.

\begin{figure}
  \centering
  \includegraphics[width=0.47\textwidth]{figs/tempAnalysisStableInitialBranch}
  \caption{Temporal analysis of the lift coefficient and normalised cylinder displacement signal at $\ured = 18.2$. We show $\cclys$ and $\cysys$ side by side in Fig. \ref{fig:tempAnalysisStableInitialBranch}a, present the temporal evolution of the phase lag $\phi$ in Fig. \ref{fig:tempAnalysisStableInitialBranch}b and show the temporal evolution of the instantaneous frequency of Cl in Fig. \ref{fig:tempAnalysisStableInitialBranch}c.}
  \label{fig:tempAnalysisStableInitialBranch}
\end{figure}

\section{The stable SVIV regime (reduced velocity greater than 20.5)} \label{sec:svivRegime}
As $\ured$ is increased to 20.5, we can see a jump in $\phi$ from a mean value of approximately 20 deg. to about 120 deg., shown in Fig. \ref{fig:phaseAngle}a. The phase slippage discussed previously is also observed, indicating the quasi-periodic nature of the lift coefficient signal at this $\ured$. Incidentally, this quasi-periodicity seems to be the norm for the lift signals up to $\ured = 27.3$, as suggested by the phase slippages evident in Figs. \ref{fig:phaseAngle}b, c and d. The slippage only stops once $\ured$ reaches 29.5, suggesting a more periodic behaviour of the lift coefficient compared to its counterparts between $20.5 \leq \ured \leq 27.3$. Although the instantaneous phase between $20.5 \leq \ured \leq 27.3$ implies a quasi-periodic nature, their time-averaged values at each $\ured$ are contained in the narrow region $114 < \phi$ (deg.) $< 135$, as is the value for $\phi$ at $\ured = 29.5$. This observation that the time-averaged value of $\phi$ to only slowly vary with respect to $\ured$, once $\ured$ increases past 20.5, can be interpreted as the dominant flow structures settling into a stable form that becomes more resilient against external excitations. Based on this feature, we classified $20.5 \leq \ured \leq 29.5$ as the upper branch of SVIV.

\begin{figure}
  \centering
  \includegraphics[width=0.45\textwidth]{figs/phaseAngle}
  \caption{The instantaneous phase lag $\phi$ of $\cclys$ in the range $\urni \leq \ured \leq \urtt$. We can observe $\phi$ slipping through 360 deg. between $\urni \leq \ured \leq \urtv$, before disappearing at $\ured = \urtt$; an indication of improved stability and resilience of the vortical structure driving the vibration.}
  \label{fig:phaseAngle}
\end{figure}

The data on the evolution of $\phi$ allows us to construct a map of the ``branches'' of vibration modes observed in the range of $\ured$ that we studied. As the branches are mapped against $\ured$, we need a representative value of $\phi$ at each $\ured$. To achieve this, we took the time-averaged values of $\phi$, i.e. $\phim$, and plotted them against $\ured$ in Fig. \ref{fig:phaseAngleRegime}. The region A indicates the initial branch of  KVIV, where  $\phim$ is close to zero. Region B denotes the upper/lower branch of  KVIV, where the system experiences a jump from  $\phim \approx 0$ to greater than 110 deg. The value of $\phim$ settles very close to 180 deg. towards the end of this upper/lower branch.

\begin{figure}
  \centering
  \includegraphics[width=0.37\textwidth]{figs/phaseAngleRegime}
  \caption{Vibration regimes identified during analysis of $\phi$. To capture the evolution of $\phi$ with respect to $\ured$, a representative value for $\phi$ at each $\ured$ must be selected. We chose to use the time-averaged $\phi$, $\phim$, as the representative value.}
  \label{fig:phaseAngleRegime}
\end{figure}

Then, $\phim$ experiences a slight drop from about one-sixth the value of $\phim$ in region B, as we enter region C, marking the start of the transition to the SVIV regime. Following this, the system undergoes a more sudden drop to $\phim \approx 20$ deg. at $\ured = 18.2$. This we designate as region D. Finally, in region E, we observe another jump in $\phim$ from $\phim \approx 20$ deg. in region D to approximately 120 deg. when $\ured \geq \urni$.

\chapter{Estimation of harnessable power from a pure cruciform} \label{chap:estimPow}
\section{Mathematical model for power estimation} \label{sec:mathModel}
The mathematical model for harnessable power estimation in this study follows that which had been derived in \citet{Raghavanetal2007}. In these works, the authors mentioned that work done by the oscillating cylinder $\wcl$ during one cycle of oscillation $\tosc$ is as follows.

\begin{equation}
  \wcl = \int_{0}^{\tosc} \left ( \fl \cdot \dot{y} \right ) dt
  \label{eq:workCylinder}
\end{equation}

\noindent where both the lift $\fl$ and cylinder velocity $\dot{y}$ are both functions of time. Through several manipulations and simplifying assumptions \citep{Sun2016}, the power captured by the system can be written, using our parameters, as the fluid power

\begin{equation}
  \pfrms = \frac{1}{2} \rho \pi \cclrms U^{2} \fcyl \yrms D L \sin(\phi),
  \label{eq:rmsFluidPower}
\end{equation}

\noindent or the mechanical power

\begin{equation}
  \pmrms = 8 \pi^{3} \meff \zetatot \left (\yrms \fcyl \right )^{2} \fn.
  \label{eq:rmsMechPower}
\end{equation}

Here, $\pfrms$, $\pmrms$, $L$, $\cclrms$, $\zetatot$ and $\meff$ are the \rms{} of fluid power, \rms{} of mechanical power, length of the circular cylinder, characteristic \rms{} of lift amplitude, total damping coefficient, and the system effective mass respectively. We use $\cclys$ to represent $\cclrms$ in Eq. \ref{eq:rmsFluidPower}. We choose to use \rms{} (parameters with subscript RMS) quantities in Eq. \ref{eq:workCylinder} instead of the maximum values like the original authors because that may lead to a misunderstanding that the maximum value is sustained throughout the observation window. This obviously is not always the case in our study, especially once the system transits into the SVIV regime. Recall that the time series analysis of $\ystr\left( t \right)$ and $\text{Cl}\left( t \right)$ in Chapter \ref{chap:svivRegime} revealed that there is a degree of intermittency in both signals that cannot be overlooked at certain ranges of $\ured$. Using the \rms{} value allows us to partially take this into account in the estimation of harnessable power.

Figure \ref{fig:powerComparison} shows the comparison between power estimated from our experiment and numerical results, with the experimental results of \citet{Nguyen2012} and the direct power measurement of \citet{Koide2013}. Only the value for $\pmrms$ is computed from our experimental results due to the absence of lift data. Our numerical results have both lift and cylinder displacement data, and hence, we calculated both $\pfrms$ and $\pmrms$. We estimated the power from the experimental results of \citet{Nguyen2012} by interpolating missing data points in both their amplitude and frequency responses to compute the value of $\pmrms$ at a given value of $\ured$. The direct power measurement by \citet{Koide2013} was done by connecting the elastic support of the cylinder to a coil. The coil moves with the cylinder, thus creating a relative pistoning motion against a fixed magnet and produces an alternating current.

\begin{figure}
  \centering
  \includegraphics[width=0.4\textwidth]{figs/powerComparison}
  \caption{Estimated \rms{} of mechanical power $\pmrms$, fluid power $\pfrms$, or both, of our experimental and numerical results, compared with results of similar studies in the literature. The fluid power $\pfrms$ is calculated only from the results of our numerical study as the others did not measure lift.}
  \label{fig:powerComparison}
\end{figure}

\begin{figure}
  \centering
  \includegraphics[width=0.47\textwidth]{figs/instantLiftFreq}
  \caption{The instantaneous frequency of the lift signal between $\urni \leq \ured \leq \urtt$. The white, solid boxes enclose the IMF component of Cl due to the shedding of the streamwise vortex, while the dashed, white boxes enclose the IMF component due to the shedding of Karman vortex. Through visual inspection, we can see how the degree of dispersion (i.e., height of the box) in the instantaneous frequency of the ``Karman component'' of lift is about twice that of the ``streamwise component'' of lift.}
  \label{fig:instantLiftFreq}
\end{figure}

The estimated power in the KVIV regime $\ured \leq \urse$ produces power only in the order of \si{\micro\watt}, which is relatively insignificant in contrast to the magnitude of power produced in the SVIV regime (mW). In the region $\urei \leq \ured \leq \urte$, $\pmrms$ for our experiment and numerical work exhibits a similar trend where we observed a sudden jump in power output, followed by a gradual decrease. This gradual decrease can be attributed to the increased turbulence level right after the onset of SVIV that imposes a degree of intermittency to the normalised cylinder displacement signal, $\ystr$. For $\pfrms$, however, the quantity exhibits a monotonic increase in the range $\urei \leq \ured \leq \urte$. We only observe a dip in $\pfrms$ at $\ured = \urel$, suggesting an increase in intermittency of $\cclys$ at this $\ured$. In the experimental work of \citet{Nguyen2012}, $\pmrms$ only experiences a monotonic increase in the region $\urei \leq \ured \leq \urte$. This decidedly different response of the system compared to ours most likely stem from the difference in the actual cruciform used by \citet{Nguyen2012}. They used two circular cylinders of diameter \SI{10}{\milli\metre} as their cruciform, whereas we used a circular cylinder - strip plate in both our experiments and numerical work. There are no data from the direct power measurement of \citet{Koide2013} to compare with within $\urei \leq \ured \leq \urte$.

In the range $\urel \leq \ured \leq \urtt$, we find a reasonably good agreement between the trend found in all data compared: they increase monotonically with respect to $\ured$. Although the value of our $\pfrms$ falls quite notably below the value of $\pmrms$ at $\ured = \urel$, other values of $\pfrms$, $\pmrms$ from our numerical results and the direct power measurements by \citet{Koide2013} agree well within $\urtv \leq \ured \leq \urtt$. The only set of power data that consistently falls quite a distance below the others is the $\pmrms$ estimated from the experimental data of \citet{Nguyen2012}, which again, is most probably due to the difference in the actual geometry of the cruciform used in their investigation.

\section{Possibility for increasing fluid power} \label{sec:possIncrease}
Recall in Fig. \ref{fig:powerComparison} that although $\pfrms$ is computed according to Eq. \ref{eq:rmsFluidPower}, which uses $\cclrms$ instead of the actual \rms{} amplitude of lift ($\clrms$), the resulting power estimate does not result in a trend that is totally different from the trend found in the other datasets. Furthermore, except for $\pmrms$ estimated from the experimental data of \citet{Nguyen2012}, the values of $\pfrms$ are in fairly good agreement with other data that it is compared against at high $\ured$ ($\ured = \urtv$ and $\urtt$). We see this is an indication that the lift component selected for use in computation of $\pfrms$ is an arguably faithful representation of the force driving the motion of the cylinder. This suggests that the motion of the cylinder, once it enters the SVIV regime, is driven only by one component, and not the totality, of the lift force. This component -- that has a time-averaged frequency close to the natural frequency of the system, $\fn$ -- is the ``streamwise component'' of lift.

Another significant IMF component of the lift force in the SVIV regime is the component whose mean frequency is close to the Karman frequency of vortex shedding, as explained in \S\ref{sec:transSVIV}. This Karman component of lift has a similar amplitude size as the streamwise component of lift, as evidenced in Fig. \ref{fig:instantLiftFreq}, and as such is also a dominant component of lift. The Karman components are marked with a dashed, white box, and the streamwise components are marked with a solid, white box, following the convention in Figs. \ref{fig:tempAnalysisKVIV}, \ref{fig:tempAnalysisUpper}, \ref{fig:tempAnalysisLower}, \ref{fig:tempAnalysisTransition} and \ref{fig:tempAnalysisStableInitialBranch}. However, the Karman component fails to affect the cylinder vibration like the streamwise component most probably due to the large difference between the mean frequency of the Karman component and the natural frequency of the system, $\fn$.  The streamwise component has a mean frequency close to $\fn$ and is hence able to synchronise with the vibration of the cylinder, producing a sizeable amplitude response.

\begin{figure}
  \centering
  \includegraphics[width=0.43\textwidth]{figs/karmanStreamwiseComponents}
  \caption{Evolution of the \rms{} amplitude of two dominant lift components due to Karman ($\cflkrms$) and streamwise ($\cflsrms$) vortices with respect to $\ured$. The region $\urei \leq \ured \leq \urte$ exhibits similar magnitude for both the Karman and streamwise components of lift. On the other hand, the magnitude of amplitude for the Karman component while the region $\urel \leq \ured \leq \urtt$ is almost always twice that of the streamwise component.}
  \label{fig:karmanStreamwiseComponents}
\end{figure}

Figure \ref{fig:karmanStreamwiseComponents} shows the \rms{} amplitude of the Karman and streamwise components of lift in the SVIV regime $\ured \geq \urei$. Between $\urei \leq \ured \leq \urte$, the magnitude of the Karman and streamwise components are nearly equal. However, once we exceed $\ured = \urte$, Fig. \ref{fig:karmanStreamwiseComponents} shows that the contribution to the \rms{} amplitude of total lift by the Karman component is on average twice the contribution of the streamwise component. Having such a significant contribution towards the \rms{} amplitude of total lift implies that there is a significant portion of energy from the free stream being used to energise the Karman vortex structure in the flow. Let us assume a hypothetical situation where we can transfer the contribution by the Karman component to the streamwise component of lift. In other words, consider the situation where we can completely redirect the energy from the Karman to the streamwise vortex. Then, the value for $\cclrms$ in Eq. \ref{eq:rmsFluidPower} will increase close to a factor of 2 when $\urei \leq \ured \leq \urte$, and close to a factor of 3 when $\urel \leq \ured \leq \urtt$. This increase in $\cclrms$ will lead to the scaling of $\pfrms$ by the same factor, keeping the other parameters in Eq. \ref{eq:rmsFluidPower} constant. This exercise demonstrates the room for improvement possible for $\pfrms$ in future developments of cruciform energy harvesters.

One possible method of improving $\pfrms$ is by implementing a modified version of the cruciform that is able to enforce the dominance of the vortical structure that is able to lock into $\fn$ - which in Fig. \ref{fig:karmanStreamwiseComponents} is the streamwise vortex - against the vortical structures that do not, i.e., the Karman vortices. We will outline such a method in our future work.

\chapter{Transition to Karman vortex-driven vibration}\label{chap:transitionToKarman}

\section{The amplitude and frequency response}\label{sec:transRegimeAmpFreqResp}

As we reduce the cruciform angle to \angfo{} and \angth{}, we find that the SVIV branch observed in the pure cruciform case as $\ured \geq \urse$ disappear, as one can inspect in Fig. \ref{fig:yStrRMS23}. For the \angth{} cruciform, even $\yrms$ at the KVIV upper branch ($\ured = \urtw$) is lower than the corresponding values for for both the \angfi{} and \angfo{} cruciforms.

\begin{figure}
  \centering
  \begin{subfigure}[h]{0.4\textwidth}
    \includegraphics[width=\textwidth]{figs/yStrRMS2}
  \caption{The \angfo{} cruciform.}
    \label{fig:yStrRMS2}
  \end{subfigure}
  \hspace{6mm}
  \begin{subfigure}[h]{0.4\textwidth}
    \includegraphics[width=\textwidth]{figs/yStrRMS3}
    \caption{The \angth{} cruciform.}
    \label{fig:yStrRMS3}
  \end{subfigure}

  \caption{Evolution of the normalised \rms{} amplitude of cylinder displacement $\yrms$, with respect to reduced velocity $\ured$, for the \angfo{} and \angth{} cruciform.}
  \label{fig:yStrRMS23}
\end{figure}

Another striking departure from the trend observed in the pure cruciform case, can be found in the evolution of $\fstr$ in Fig. \ref{fig:yStrFreq43}. For the \angfo{} cruciform in Fig. \ref{fig:yStrFreq4}, $\fstr$ seems to fluctuate with respect to $\ured$ - suggesting asymmetry in the vortical structures regulating the vibration as discussed previously in \S\ref{sec:svivRegimeAmpFreqResp}. This fluctuation is however, not as pronounced in Fig. \ref{fig:clFreq3}, compared to Fig. \ref{fig:clFreq4}. We think this behaviour is due to the \angth{} cruciform being less similar to the \angfi{} cruciform, in contrast to \angfo{}. In other words, the \angth{} cruciform is less in transition from the response of the \angfi{} cruciform and the flow around it is more evolved into its new configuration, unlike the \angfo{} cruciform.

\begin{figure}
  \centering
  \begin{subfigure}[h]{0.4\textwidth}
    \includegraphics[width=\textwidth]{figs/yStrFreq4}
    \caption{The \angfo{} cruciform.}
    \label{fig:yStrFreq4}
  \end{subfigure}
  \hspace{6mm}
  \begin{subfigure}[h]{0.4\textwidth}
    \includegraphics[width=\textwidth]{figs/yStrFreq3}
    \caption{The \angth{} cruciform.}
    \label{fig:yStrFreq3}
  \end{subfigure}

  \caption{Evolution of the normalised cylinder displacement frequency, $\fstr$, with respect to reduced velocity $\ured$, for the \angfo{} and \angth{} cruciforms.}
  \label{fig:yStrFreq43}
\end{figure}

We summarised the \rms{} lift coefficients $\clrms$ of the \angfo{} and \angth{} cruciforms in Fig. \ref{fig:yStrRMS23}. Here, we find that their evolution with respect to $\ured$ in Figs. \ref{fig:clRMS4} and \ref{fig:clRMS3} approximates their corresponding $\yrms$ trend in Figs. \ref{fig:yStrRMS2} and \ref{fig:yStrRMS3}. As for the variation of $\fclstr$ with respect to $\ured$, for both the \angfo{} and \angth{} cruciforms, both exhibit outstanding similarity to $\fvk$ of Eq. \ref{eq:karmanSheddingFreq}. This trait hints that vibrations resulting from the \angfo{} and \angth{} cruciforms are primarily regulated by the shedding of Karman vortices.

The fact that the $\fclstr$ trends observed in Figs. \ref{fig:clFreq4} and \ref{fig:clFreq3} do not lead to similar trends in the evolution of $\fstr$ in Figs. \ref{fig:yStrFreq4} and \ref{fig:yStrFreq3} leads us to believe there is something more fundamental at play in developing the $\fstr$ patterns we observed in Figs. \ref{fig:yStrFreq43}. Hence we examined the vortical structures present in the flow surrounding the \angfo{} and \angth{} cruciforms, and discuss our findings in \S\ref{sec:transitionalRegimeVortStruct}. 

\begin{figure}
  \centering
  \begin{subfigure}[h]{0.4\textwidth}
    \includegraphics[width=\textwidth]{figs/clRMS4}
    \caption{The \angfo{} cruciform.}
    \label{fig:clRMS4}
  \end{subfigure}
  \hspace{6mm}
  \begin{subfigure}[h]{0.4\textwidth}
    \includegraphics[width=\textwidth]{figs/clRMS3}
    \caption{The \angth{} cruciform.}
    \label{fig:clRMS3}
  \end{subfigure}

  \label{fig:clRMS43}
  \caption{Evolution of the normalised Cl \rms{} amplitude, $\clrms$, with respect to reduced velocity $\ured$, for the \angfo{} and \angth{} cruciforms.}
\end{figure}

\begin{figure}
  \centering
  \begin{subfigure}[h]{0.4\textwidth}
    \includegraphics[width=\textwidth]{figs/clFreq4}
    \caption{The \angfo{} cruciform.}
    \label{fig:clFreq4}
  \end{subfigure}
  \hspace{6mm}
  \begin{subfigure}[h]{0.4\textwidth}
    \includegraphics[width=\textwidth]{figs/clFreq3}
    \caption{The \angth{} cruciform.}
    \label{fig:clFreq3}
  \end{subfigure}

  \label{fig:clFreq43}
  \caption{Evolution of the normalised Cl frequency, $\fclstr$, with respect to reduced velocity $\ured$, for the \angfo{} and \angth{} cruciforms. The dashed lines outline $\fvk$ from Eq. \ref{eq:karmanSheddingFreq}.}
\end{figure}

\section{Main vibration-driving vortical structure}\label{sec:transitionalRegimeVortStruct}
As the first step, we computed the FFT of the $y$-component of velocity similar to what we did in Fig. \ref{fig:probe90YU10} for both the \angfo{} and \angth{} cruciforms. Our initial assesment of the $\fvkstr$ distribution along the cylinder axis in Figs. \ref{fig:probe675YU10} and \ref{fig:probe45YU10} is that there is a strong representation of the Karman shedding frequency in both cases, which at $\ured = \urte$ is $\fvkstr = 4.49$. However, unlike Fig. \ref{fig:probe90YU10}, there is no frequency band close to 1 around the cruciform juncture.

Our first guess is that streamwise vortices driving the vibration in the pure cruciform case do not get initiated once the cruciform angle deviates away from \angfi{}. However, $x$ and $z$-component vorticity visualisations in Fig. \ref{fig:vortStruct67545} points out otherwise. These visualisations are produced under the same conditions as Fig. \ref{fig:vortStruct90}. Inspecting Figs. \ref{fig:vorx675} and \ref{fig:vorx45}, we can quite clearly make out the large-scale streamwise vortices close to the cruciform juncture. We thus expect the cylinder vibration and streamwise vortices to synchronise to each other, resulting in an $\fstr \approx 1$ accross the values of $\ured$ after the lower branch of KVIV which appears at $\ured = \urth$. In addition, we expect there to be a large amplitude response from the cylinder due to the formation and sustenance of the streamwise vortices close to the cruciform juncture. This was not the case.

We think the reason behind this lies in the distribution of the streamwise vortex cells along the cylinder axis. As we mentioned in \S\ref{sec:svivRegimeVortStruct}, the streamwise vortex pair in the pure cruciform case lie on a plane parallel to the axis of the cylinder. The shared plane of formation parallel to cylinder axis is what we think as key to the large amplitude response observed in the pure cruciform case. A formation plane parallel to the cylinder axis ensures the resulting downward thrust to act perpendicular to the cylinder, securing a larger amplitude response. Also, the streamwise vortex cell on each side of the $Z = 0$ plane must be of opposing rotational direction for the production of thrust. We find these two aspects missing in the \angfo{} and \angth{} cruciforms in Fig. \ref{fig:vortStruct67545}.

What takes place in Figs. \ref{fig:vorx675} and \ref{fig:vorx45} is, two streamwise vortex cells of opposing poles form on each side of the $Z = 0$ plane. The result of this vortical arrangement is severe diminishing of useful thrust, and by extension, lift acting on the cylinder. This explains why the amplitude of $\fvstr$ at and within the vicinity of $1$ is extremely small in comparison to the dominant band close to $\fvkstr = 4.49$, resulting in a low amplitude response for most of the $\ured$ studied, even when $\fstr \approx 1$.

We also managed to find the source of asymmetry in the vortical structure distribution around the cruciform, which becomes apparent upon closer comparison of Fig. \ref{fig:vorz675} and Fig. \ref{fig:vorz45}. The Karman vortices of the \angfo{} cruciform are shed at different phases depending on which side of the $Z = 0$ plane we are observing. For the particular case in Fig. \ref{fig:vorz675}, Karman vortices are being shed from the top of the cylinder when $Z < 0$, and from the bottom of the cylinder when $Z > 0$. What suggested this interpretation is our observation of a strong expression of $-z$ vorticity when $Z < 0$ from the top of the cylinder, while on the $Z > 0$ half of our domain is a strong expression of $+z$ vorticity from the bottom of the cylinder. We do not see this take place in Fig. \ref{fig:vorz45}. On both sides of the $Z = 0$ plane, we note the strong expression of $-z$ vorticity from the top of the cylinder. This asymmetry leads to competing vibration-driving mechanisms resulting in the oscillatory behaviour of $\fstr$ seen in Fig. \ref{fig:yStrFreq4}.

\begin{figure}
  \centering

  \begin{subfigure}[h]{0.46\textwidth}
    \includegraphics[width=\textwidth]{figs/probe675YU10}
    \caption{The $\fvstr$ for the \angfo{} cruciform.}
    \label{fig:probe675YU10}
  \end{subfigure}
  \hspace{6mm}
  \begin{subfigure}[h]{0.46\textwidth}
    \includegraphics[width=\textwidth]{figs/probe45YU10}
    \caption{The $\fvstr$ for the \angth{} cruciform.}
    \label{fig:probe45YU10}
  \end{subfigure}

  \caption{Distribution of normalised frequency of vortex shedding, along the span of the cylinder of the \angfo{} and \angth{} cruciforms at $\ured = \urte$.}
  \label{fig:probe67545YU10}
\end{figure}

\begin{figure}
  \centering
  \begin{subfigure}[h]{0.4\textwidth}
    \centering
    \includegraphics[width=0.7\textwidth]{figs/vorx675}
    \caption{$x$-component vorticity, \angfo{} cruciform.}
    \label{fig:vorx675}
  \end{subfigure}
  \begin{subfigure}[h]{0.4\textwidth}
    \centering
    \includegraphics[width=0.7\textwidth]{figs/vorz675}
    \caption{$z$-component vorticity, \angfo{} cruciform.}
    \label{fig:vorz675}
  \end{subfigure}

  \begin{subfigure}[h]{0.4\textwidth}
    \centering
    \includegraphics[width=0.7\textwidth]{figs/vorx45}
    \caption{$x$-component vorticity, \angth{} cruciform.}
    \label{fig:vorx45}
  \end{subfigure}
  \begin{subfigure}[h]{0.4\textwidth}
    \centering
    \includegraphics[width=0.7\textwidth]{figs/vorz45}
    \caption{$z$-component vorticity, \angth{} cruciform.}
    \label{fig:vorz45}
  \end{subfigure}

  \caption{Dominant vortical structures at $\ured = \urte$ observed in the \angfo{} and \angth{} cases. The vorticity slices shown are the $x$ and $y$-component vorticities (\si{\per\second}) at the $x/D = 1.96D$ plane, viewed orthogonal to that plane from downstream.} \label{fig:vortStruct67545}
\end{figure}

\section{Phase lag between Cl and normalised cylinder displacement} \label{sec:phaseLag67545}
Evolution of $\plag$ for the \angfo{} and \angth{} cruciforms share a similar trend in the $\uron \leq \ured \leq \urth$ range. For the \angfo{} cruciform, past $\ured = \urth$, there seem to be two distinct VIV branches between $\uron \leq \ured \leq \ursi$, and between $\urse \leq \ured \leq \urtt$. In the first branch between $\uron \leq \ured \leq \ursi$, $\plag$ seems to remain close to \SI{70}{\degree}, while in the second branch, close to \SI{110}{\degree}. In some sense, this is fairly similar to the $\plag$ vs. $\ured$ pattern of the pure cruciform case, except for two features. First, the lower branch of KVIV -- which in the pure cruciform case exhibits a $\plag \approx \SI{180}{\degree}$ -- does not extend beyond $\ured = \urth$. Instead, the value for $\plag$ suddenly drops to $\approx \SI{70}{\degree}$ before jumping back up to $\approx \SI{110}{\degree}$ starting at $\ured = \urse$. This value of \SI{110}{\degree} is curiously close to $\plag$ of the pure cruciform between $\urni \leq \ured \leq \urtt$, equivalent to the upper branch of SVIV. If we work backwards from $\ured = \urni$ in the direction of decreasing $\ured$, and compare Fig. \ref{fig:phaseLag675deg} and Fig. \ref{fig:phaseLag90deg}, we confront the possibility that the $\plag = \SI{70}{\degree}$ branch of the \angfo cruciform is a \textit{variant} of the SVIV initial branch. For the pure cruciform case, this occurs within a narrow window of $\urse < \ured < \urni$, with $\plag \approx \SI{20}{\degree}$.

For the \angth{} cruciform in Fig. \ref{fig:phaseLag45deg}, right after lower branch of KVIV at $\ured = \urth$, $\plag$ transitioned to $\approx \SI{65}{\degree}$ between $\urfi \leq \ured \leq \urni$. The proximity between the values \SI{65}{\degree} and \SI{70}{\degree} for the \angfo{} cruciform suggests the equivalence of the two branches. However, we are inclined to a more cautious conclusion: the two are \textit{variants} of the same branch that is only similar in limited respects as they originate from different angled cruciforms. In this respect, the \angth{} cruciform is much less similar to the pure cruciform case, as one might expect, simply because \angth{} is a much larger deviation from \angfi{} compared to \angfo{}. This larger deviation is in our opinion what causes $\plag$ in the \angth{} case to drop further to $\approx \SI{35}{\degree}$ within $\urte \leq \ured \leq \urtt$, instead of jumping up to some mean value between $100 \leq \plag \; (\si{\degree}) \leq 130$ similar to what we observe in the \angfi{} and \angfo{} cruciforms.

\begin{figure}
  \centering
  \begin{subfigure}[h]{0.4\textwidth}
    \includegraphics[width=\textwidth]{figs/phaseLag4}
    \caption{Phase lag for the \angfo{} cruciform.}
    \label{fig:phaseLag675deg}
  \end{subfigure}
  \hspace{6mm}
  \begin{subfigure}[h]{0.4\textwidth}
    \includegraphics[width=\textwidth]{figs/phaseLag3}
    \caption{Phase lag for the \angth{} cruciform.}
    \label{fig:phaseLag45deg}
  \end{subfigure}

  \caption{Phase lag $\plag$ (\si{\degree}) between Cl and $\ystr$ when \angfo{} and \angth{}.}
  \label{fig:phaseLag67545deg}
\end{figure}


\chapter{Karman vortex-driven vibration}\label{chap:kvivRegime}
\section{The amplitude and frequency response}\label{sec:kvivAmpFreqResp}
As we decrease the cruciform angle even further to \angtw{}, we observe a significant change in the amplitude and frequency response, compared to the transitional stage previously explored in \S\ref{chap:transitionToKarman}. Consider Figs. \ref{fig:yStrRMS4} and \ref{fig:yStrFreq2}. There is no apparent KVIV upper branch similar to what we have seen in Figs. \ref{fig:yStrRMS2} and \ref{fig:yStrFreq4} at $\ured = \urtw$. Then, we observe no significant vibration elicited from the oscillator until $\ured = \urni$. When $\ured = \urni$, there is a abrupt jump in $\yrms$ from $\yrms < 0.02$ to $\yrms \approx 0.3$. The $\fstr$ vs. $\ured$ trend demonstrated a bypassing of the KVIV lower branch - which occurred at $\ured = \urth$ in the transitional stages of \S\ref{chap:transitionToKarman} - right into $\fstr \approx 1$. Past $\ured = \urni$, the value for $\yrms$ continues to increase and saturates at $\ured = \urel$, and $\fstr$ continues to be close to $1$ up to $\ured = \urtt$.

The \angon{} cruciform (Figs. \ref{fig:yStrRMS5}, \ref{fig:yStrFreq1}) in essence exhibits a very similar trend in its $\yrms$ and $\fstr$ evolutions with respect to $\ured$. However, the jump -- which for the \angtw{} cruciform occurs at $\ured = \urni$ -- occurs at a a much lower $\ured = \urfo$. Then, $\yrms$ continues to grow until $\ured = \urni$, reaching a staggering $\yrms \approx 0.8$, a value no other study on energy harvesting using cruciform oscillators has ever achieved. The value of $\yrms$ then saturates close to $0.8$ up to $\ured = \urtt$. Also similar to the \angtw{} cruciform, $\fstr$ falls close to $1$ at same $\ured$ the $\yrms$ jump occurs - hinting at synchronisation between vortex shedding and system natural frequency.

\begin{figure}
  \centering
  \begin{subfigure}[h]{0.4\textwidth}
    \includegraphics[width=\textwidth]{figs/yStrRMS4}
  \caption{The \angtw{} cruciform.}
    \label{fig:yStrRMS4}
  \end{subfigure}
  \hspace{6mm}
  \begin{subfigure}[h]{0.4\textwidth}
    \includegraphics[width=\textwidth]{figs/yStrRMS5}
    \caption{The \angon{} cruciform.}
    \label{fig:yStrRMS5}
  \end{subfigure}

  \caption{Evolution of the normalised \rms{} amplitude of cylinder displacement $\yrms$, with respect to reduced velocity $\ured$, for the \angtw{} and \angon{} cruciform.}
  \label{fig:yStrRMS45}
\end{figure}

\begin{figure}
  \centering
  \begin{subfigure}[h]{0.4\textwidth}
    \includegraphics[width=\textwidth]{figs/yStrFreq2}
    \caption{The \angtw{} cruciform.}
    \label{fig:yStrFreq2}
  \end{subfigure}
  \hspace{6mm}
  \begin{subfigure}[h]{0.4\textwidth}
    \includegraphics[width=\textwidth]{figs/yStrFreq1}
    \caption{The \angon{} cruciform.}
    \label{fig:yStrFreq1}
  \end{subfigure}

  \caption{Evolution of the normalised cylinder displacement frequency, $\fstr$, with respect to reduced velocity $\ured$, for the \angtw{} and \angon{} cruciforms.}
  \label{fig:yStrFreq21}
\end{figure}
Apart from the amplitude/frequency response of the \angtw{} cruciform, the evolution of $\clrms$ against $\ured$ showcases a similar trend to the evolution $\yrms$, as shown in Fig. \ref{fig:clRMS2}. The corresponding $\fclstr$ in Fig. \ref{fig:clFreq2} demonstrates the dominant frequency of Cl taking after Eq. \ref{eq:karmanSheddingFreq} between $\uron \leq \ured \leq \urei$, before abruptly dropping close to $1$ between $\urni \leq \ured \leq \urtt$. This in part informs us that the flow between $\uron \leq \ured \leq \urei$ is governed by flow physics that are similar to both \angfo{} and \angth{} within the same $\ured$ range.

For the \angon{} cruciform, we see that the jump in $\clrms$ occurs at the same $\ured$ as the jump in the corresponding $\yrms$ i.e., $\ured = \urfo$. After $\ured = \urfo$, the magnitude of $\clrms$ gradually drops to a final value of $\clrms \approx 0.45$ at $\ured = \urni$, and remains there up to $\ured = \urtt$. Similar to the \angtw{} cruciform, $\fclstr$ grows linearly in accordance with, again, Eq. \ref{eq:karmanSheddingFreq} until the jump in both $\clrms$ and $\yrms$, where $\fclstr$ drops close to $1$ for the rest the $\ured$ we examine in this study.

\begin{figure}
  \centering
  \begin{subfigure}[h]{0.4\textwidth}
    \includegraphics[width=\textwidth]{figs/clRMS2}
    \caption{The \angtw{} cruciform.}
    \label{fig:clRMS2}
  \end{subfigure}
  \hspace{6mm}
  \begin{subfigure}[h]{0.4\textwidth}
    \includegraphics[width=\textwidth]{figs/clRMS1}
    \caption{The \angon{} cruciform.}
    \label{fig:clRMS1}
  \end{subfigure}

  \label{fig:clRMS21}
  \caption{Evolution of the normalised Cl \rms{} amplitude, $\clrms$, with respect to reduced velocity $\ured$, for the \angtw{} and \angon{} cruciforms.}
\end{figure}

\begin{figure}
  \centering
  \begin{subfigure}[h]{0.4\textwidth}
    \includegraphics[width=\textwidth]{figs/clFreq2}
    \caption{The \angtw{} cruciform.}
    \label{fig:clFreq2}
  \end{subfigure}
  \hspace{6mm}
  \begin{subfigure}[h]{0.4\textwidth}
    \includegraphics[width=\textwidth]{figs/clFreq1}
    \caption{The \angon{} cruciform.}
    \label{fig:clFreq1}
  \end{subfigure}

  \label{fig:clFreq21}
  \caption{Evolution of the normalised Cl frequency, $\fclstr$, with respect to reduced velocity $\ured$, for the \angtw{} and \angon{} cruciforms. The dashed lines outline $\fvk$ from Eq. \ref{eq:karmanSheddingFreq}.}
\end{figure}


\section{Main vibration-driving vortical structure}\label{sec:kvivRegimeVortStruct}
To understand the cause behind the marked difference between the amplitude/frequency response in both \angfo{} and \angth{} cruciforms, we proceed to deduce the type of vortical structures that form in the flows around the \angtw{} and \angon{} cruciform. We first produce the visualise of $\fvstr$ along the cylinder, in the same manner as Figs. \ref{fig:probe90YU10} and \ref{fig:probe67545YU10}, in Fig. \ref{fig:probe225YU10}. Inspecting Fig. \ref{fig:probe225YU10}, we immediately notice that several frequency bands exist: $\fvstr \approx 1,\, 2,\, 3,\, 4.5$, the most visible of which is the $\fvstr \approx 1$ band. However, unlike the pure cruciform case in Fig. \ref{fig:probe90YU10} -- where the dominant frequency bands are localised to certain regions along the cylinder -- the $\fvstr \approx 1$ band of the \angtw{} cruciform seems to encompass the length of the cylinder. This also seems to be the case for the \angon{} cruciform, displaying a dominant $\fvstr$ band at approximately $1$ along the length of the cylinder.

We then queried the $x$ and $z$-component vorticity contours of the \angtw{} cruciform for deeper insight, presented in Figs. \ref{fig:vorx225} and \ref{fig:vorz225}. In doing so, we found that the large scale streamwise vortex pair near the cruciform juncture is absent. Instead, we noticed streamwise vortex cells distributed on both sides of the $Z = 0$ plane, except within the immediate neighbourhood of the cruciform juncture. We note that these streamwise vortex cells are distributed at a slight angle relative to the $z$-axis, seemingly following the angle of the strip plate, i.e. \SI{22.5}{\degree}. Visualisation of the $z$-component vorticity in Fig. \ref{fig:vorz225} intimates the nature of the tilted streamwise vortex cells: they are simply three-dimensional Karman vortex structures, akin to those observed in the experiments of \citet{Williamson1996}, especially mode B. We inferred this from comparison of the same visual region delimited by dashed rectangles in Figs. \ref{fig:vorx225} and \ref{fig:vorz225}, showing the overlap between the streamwise vortex cells and the strong Karman vortex component.

\begin{figure}
  \centering

  \begin{subfigure}[h]{0.46\textwidth}
    \includegraphics[width=\textwidth]{figs/probe225YU10}
    \caption{The $\fvstr$ for the \angtw{} cruciform.}
    \label{fig:probe225YU10}
  \end{subfigure}
  \hspace{6mm}
  \begin{subfigure}[h]{0.46\textwidth}
    \includegraphics[width=\textwidth]{figs/probe00YU10}
    \caption{The $\fvstr$ for the \angon{} cruciform.}
    \label{fig:probe00YU10}
  \end{subfigure}

  \caption{Distribution of normalised frequency of vortex shedding, along the span of the cylinder of the \angtw{} and \angon{} cruciforms at $\ured = \urte$.}
  \label{fig:probe22500YU10}
\end{figure}

\begin{figure}
  \centering
  \begin{subfigure}[h]{0.4\textwidth}
    \centering
    \includegraphics[width=0.7\textwidth]{figs/vorx225}
    \caption{$x$-component vorticity, \angfo{} cruciform.}
    \label{fig:vorx225}
  \end{subfigure}
  \begin{subfigure}[h]{0.4\textwidth}
    \centering
    \includegraphics[width=0.7\textwidth]{figs/vorz225}
    \caption{$z$-component vorticity, \angfo{} cruciform.}
    \label{fig:vorz225}
  \end{subfigure}

  \begin{subfigure}[h]{0.4\textwidth}
    \centering
    \includegraphics[width=0.7\textwidth]{figs/vorx00}
    \caption{$x$-component vorticity, \angth{} cruciform.}
    \label{fig:vorx00}
  \end{subfigure}
  \begin{subfigure}[h]{0.4\textwidth}
    \centering
    \includegraphics[width=0.7\textwidth]{figs/vorz00}
    \caption{$z$-component vorticity, \angth{} cruciform.}
    \label{fig:vorz00}
  \end{subfigure}

  \caption{Dominant vortical structures at $\ured = \urte$ observed in the \angtw{} and \angon{} cases. The vorticity slices shown are the $x$ and $y$-component vorticities (\si{\per\second}) at the $x/D = 1.96D$ plane, viewed orthogonal to that plane from downstream.} \label{fig:vortStruct22500}
\end{figure}

For the \angon{} cruciform in Figs. \ref{fig:vorx00} and \ref{fig:vorz00}, this overlap between the streamwise vortex cells and the distinct Karman vortex component is even more visually pronounced. We interpret this as further evidence to the hypothesis that three-dimensional Karman vortical structures govern the vibration of the cylinder for shallow angled cruciforms, including the \SI{0}{\degree} cruciform, which perhaps is more aptly named in-tandem configuration. Notice that for the \angon{} layout, both the streamwise vortex cells and the Karman vortical structure are arranged parallel to the $z$-axis, or at \SI{0}{\degree} relative to the cylinder. This observation seems to demonstrate the role of the strip plate in shallow angled cruciforms: it modifies the spatial distribution of the Karman vortical structure that is driving the vibration of the cylinder.

As we have seen in \S\ref{chap:svivRegime} and \ref{chap:transitionToKarman}, eliciting a significant vibration amplitude Karman vortex shedding is limited to a narrow $\ured$ range containing the upper branch of KVIV. Beyond that, Karman vortex shedding fails to lock into the natural frequency of the system to produce meaningful vibrations. The fact that shallow angled cruciforms in this \S\ref{sec:kvivRegimeVortStruct} are able to produce very large vibration amplitudes intimate the crucial role played by the strip plate in forcing the shedding of vortical structures and cylinder vibration to lock into the natural frequency of the system $\fn$. We think the mechanics of this forced lock-in is as follows.

\begin{enumerate}
  \item Shallow angled cruciforms have a larger overlap area between the cylinder and strip plate. This permits a more perceptible interaction among vortices shed from the cylinder and the strip plate.
  \item Upon exceeding a critical $\ured$, vortical structures shed from both the cylinder and the strip plate becomes sufficiently energised, and they start to behave as one.
  \item This vortical synchronisation locks into the natural frequency of the elastic structure in its immediate vicinity - our circular cylinder.
\end{enumerate}

\noindent The results also suggest the total cylinder-strip plate area overlap as a factor influencing the exact $\ured$ at which the vortical synchronisation occurs. The $\ured$ at which the synchronisation -- and the jump to large $\yrms$ amplitude -- occurs at a lower value when the overlap area is bigger. Nevertheless, there seems to be a limit to this lowering of $\ured$, which in this work we determined to be $\ured = \urfo$. Recall that this is the value where the synchronisation begins for the \angon{} layout; the layout where the totality of the cylinder and strip plate projections onto the $y-z$ plane coincides with each other.

\section{Phase lag between Cl and normalised cylinder displacement} \label{sec:phaseLag22500}
In Fig. \ref{fig:phaseLag225deg}, we summarise the evolution of $\plag$ with respect to $\ured$ for the \angtw{} cruciform and find $\plag \approx \SI{115}{\degree}$ at $\ured = \urtt$. The significance of $\ured = \urtt$ for the \angtw{} cruciform is that it is where a jump in $\plag$ occurs from $\approx \SI{30}{\degree}$ to $\approx \SI{115}{\degree}$. This abrupt jump usually demarcates transition to the KVIV lower branch, as stated in the discussion accompanying Figs. \ref{fig:phaseLag90deg} and \ref{fig:phaseLag67545deg}, with a value that is very close to \SI{180}{\degree}. The $\plag$ for the \angtw{} cruciform however, is about 36\% smaller than the expected $\approx \SI{180}{\degree}$. This may be the cause of absence of the KVIV lower branch in the $\fstr$ vs. $\ured$ plot of the \angtw{} cruciform (Fig. \ref{fig:yStrFreq2}).

The \angon{} arrangement produces a similar trend to Fig. \ref{fig:phaseLag225deg}, as one can see in Fig. \ref{fig:phaseLag00deg}. However, the $\plag$ jump at  $\ured =\urtt$ is more pronounced and achieves a value very close to \SI{180}{\degree}. Following the sudden jump is the sudden drop at $\ured = \urfo$, which brings $\plag$ to approximately \SI{25}{\degree}. This trend continues up to $\ured = \urse$, after which we find $\plag$ to monotonically increase up to $\ured = \urtt$.

\begin{figure}
  \centering
  \begin{subfigure}[h]{0.4\textwidth}
    \includegraphics[width=\textwidth]{figs/phaseLag2}
    \caption{Phase lag for the \angtw{} cruciform.}
    \label{fig:phaseLag225deg}
  \end{subfigure}
  \hspace{6mm}
  \begin{subfigure}[h]{0.4\textwidth}
    \includegraphics[width=\textwidth]{figs/phaseLag1}
    \caption{Phase lag for the \angon{} cruciform.}
    \label{fig:phaseLag00deg}
  \end{subfigure}

  \caption{Phase lag $\plag$ (\si{\degree}) between Cl and $\ystr$ when \angtw{} and \angon{}.}
  \label{fig:phaseLag22500deg}
\end{figure}

\chapter{Power characteristic in cruciform angle - reduced velocity parameter space}\label{chap:powerCharacteristic}
In this study, we estimate the mechanical power harnessed from the flow for each cruciform through the application of the formula in Eq. \ref{eq:rmsMechPower}.

To understand how mechanical power $\pmrms$ is influenced not only by $\ured$, but also by the strip plate tilt angle $\alpha$ (\si{\degree}), we decided to visualise $\pmrms$ as contour plots, where the abscissa and ordinate are $\ured$ and $\alpha$, respectively, and the colour of the contours denote the magnitude of $\pmrms$. As an example, we plotted the values of $\yrms$ against $\ured$ and $\alpha$ in Fig. \ref{fig:yRMSContour}. The data used to produce Fig. \ref{fig:yRMSContour} are those from Figs. \ref{fig:yStrRMS1}, \ref{fig:yStrRMS23} and \ref{fig:yStrRMS45}. We then performed linear interpolations on the $\yrms$ data along both $\ured$ and $\alpha$ axes to populate the $\alpha$--$\ured$ parameter space. The result of this two-dimensional interpolation is Fig. \ref{fig:yRMSContour}. The snapshot of $\yrms$ evolution in the $\alpha$--$\ured$ parameter space summarises our observations made previously in Figs. \ref{fig:yStrRMS1}, \ref{fig:yStrRMS23} and \ref{fig:yStrRMS45}.

\begin{figure}
  \centering
  \includegraphics[width=0.55\textwidth]{figs/yRMSContour}
  \caption{Isocontours describing the map of the normalised RMS amplitude of cylinder displacment, $\yrms$ in the cruciform angle - reduced velocity ($\alpha$--$\ured$) parameter space.}
  \label{fig:yRMSContour}
\end{figure}

\begin{figure}
  \centering
  \includegraphics[width=0.55\textwidth]{figs/mechanicalPowerContours}
  \caption{Isocontours describing the map of the estimated mechanical power in the cruciform angle - reduced velocity ($\alpha$--$\ured$) parameter space.}
  \label{fig:mechanicalPowerContour}
\end{figure}

We then perform the same two-dimensional interpolation on our $\pmrms$ results and summarised them in Fig. \ref{fig:mechanicalPowerContour}. Two regions of significant power generation exist: first, in the $\urei \leq \ured \leq \urtt$ range as $\alpha$ approaches \angfi{}, and second, starting as low as $\ured = \urfo$ up to $\ured = \urtt$ as $\alpha$ approaches \angon{}. The estimated power in the region approaching $\alpha = \angfi{}$ agrees well in magnitude with previous experimental works such as \citet{Koide2013}. Also, even though no power data was measured in pure cruciform oscillator studies of \citet{Koide2009,Nguyen2012}, their $\yrms$ data agrees well in trend and in magnitude with Fig. \ref{fig:yRMSContour}, along the \angfi{} line. The $\pmrms$ contour of Fig. \ref{fig:mechanicalPowerContour} visualises the high power region as $\alpha$ approaches \angon{} and $\ured$ approaches $\urtt$. The highest $\pmrms$ estimated from our simulation reaches $O(10^{1})$ \si{\milli\watt}, which is one order of magnitude larger than any cruciform energy harvester of this scale have ever recorded in the literature.

The $\pmrms$ contour also elucidated the feasibility of cruciform angle variation as a means to control the range of $\ured$ within which substantive quantity of power can be harnessed. For a given operating range of $\ured$, one can choose for substantive power generation to take place in the high $\ured$ region ($\ured \geq \urei$), or across a larger range starting from a minimum of $\ured = \urfo$, by manipulating the cruciform angle. For power generation in the high $\ured$ region, one simply sets the cruciform angle to \angfi{}, and for power generation starting from the lowest possible $\ured$, one should opt for a shallow-angled cruciform, with increasing magnitude of power generated as $\alpha \rightarrow \SI{0}{\degree}$.

Our estimation of mechanical power efficiency $\etamech$ (\%) is based on the definition of efficiency in \citet{Sun2018}. Our version of $\etamech$ is shown in Eq. \ref{eq:mechanicalEfficiency}.

\begin{equation}
  \etamech \;(\%) = \frac{\pmrms}{P_{\text{Fluid}}} = \frac{1}{2}\rho U^{3}_{\infty} \left( 2 y_{\text{RMS}} + D \right) L
  \label{eq:mechanicalEfficiency}
\end{equation}

\noindent Here, $U_{\infty}$ and $L$ are the freestream velocity of the flow and the length of the cylinder, respectively. Our variant of efficiency uses $y_{\text{RMS}}$ instead of $y_{\text{Max.}}$ like \citet{Sun2018}, due to our focus on time-averaged quantities instead of possibly one-off values.

Unlike Fig. \ref{fig:mechanicalPowerContour}, the efficiency contour does not display a trend similar to the contour of $\yrms$ in Fig. \ref{fig:yRMSContour}. The pure cruciform achieves maximum $\etamech$ close to $\ured = \urei$, steep-angled cruciforms (\angfo{} and \angth{} cruciforms) attain maximum $\etamech$ within the vicinity of $(\ured,\alpha) = (\urtw,\angfo)$, and shallow-angled cruciforms (\angtw{} and \angon{} layouts) strikes maximum $\etamech$ in the neighbourhood of $(\ured,\alpha) = (\urfo,\angon)$. These high-efficiency regions is consistent with our assertion that different flow dynamics govern the vibration resulting from either the pure, steep-angled or shallow-angled cruciforms.

\begin{figure}
  \centering
  \includegraphics[width=0.55\textwidth]{figs/powerEfficiencyContours}
  \caption{Isocontours describing the map of the estimated mechanical power in the cruciform angle - reduced velocity ($\alpha$--$\ured$) parameter space.}
  \label{fig:powerEfficiencyContour}
\end{figure}

\chapter{Conclusion}

In this study, we numerically investigated the temporal evolution of the lift coefficient and cylinder displacement signals of an elastically supported cruciform system in the range $1.1 \times 10^{3} < \re < 14.6 \times 10^{3}$, or $\uron < \ured < \urtt$. Our circular cylinder diameter is \SI{10}{\milli\metre} and the natural frequency of the system is \SI{4.4}{\hertz}. Validation of key numerical results was made experimentally in a custom-built open flow channel, using a cruciform system whose parameters were tuned as close as possible to the quantities used in the numerical study. Decomposing the lift coefficient signal in the SVIV regime ($\urse \leq \ured \leq \urtt$) using EEMD allows us to see that the complexity of the lift coefficient signal as being caused by the superpositioning of two dominant components of lift. One due to the shedding of Karman and the other due to the shedding of streamwise vortices. The former has a frequency close to the vortex shedding frequency of Karman vortex from a smooth, isolated circular cylinder, while the latter has a mean frequency close to $\fn$. Application of the Hilbert-Huang transform on the dominant component of cylinder displacement -- and the component of lift most correlated to it -- allows for the computation of the instantaneous phase lag between lift and cylinder displacement. The time-averaged phase lag revealed five ``branches'' of vibration, among which is the initial branch of SVIV at $\ured = \urei$, which has never been identified before in the literature. We also computed the instantaneous frequency of the lift coefficient, thus revealing the loss of periodicity and self-similarity in the lift coefficient signal as the system enters the SVIV regime. Estimation of power from our results show that the \rms{} mechanical and fluid power computed from our experimental and numerical work agree to varying degrees depending on $\ured$ with data from similar studies in the literature. Finally, we estimated that the \rms{} fluid power can potentially be increased close to a factor of 2 within $\urei \leq \ured \leq \urte$ and close to a factor of 3 when $\urel \leq \ured \leq \urtt$. We base this estimation on the premise of redirecting the contribution to the \rms{} amplitude of total lift from Karman vortex shedding, towards the streamwise component of lift alone.

In this study, we numerically investigated the temporal evolution of the lift coefficient and cylinder displacement signals of an elastically supported cruciform system in the range $1.1 \times 10^{3} < \re < 14.6 \times 10^{3}$, or $\uron < \ured < \urtt$, for cruciform angles $\alpha = \angfi,\,\angfo,\,\angth,\,\angtw$ and $\angon$. We chose the \angfi{} cruciform as the representative case to validate our numerical setup through a GCI study and comparison with experimental measurements in a custom-built closed loop open flow channel. After successful validation of the \angfi{} cruciform, we impose the same conditions on all cruciforms studied, which includes mesh resolution, boundary conditions and solver settings. Here are the main conclusions from this work.
\renewcommand{\labelenumi}{(\alph{enumi})}
\begin{enumerate}
  \item The large amplitude vibrations of the pure cruciform (\angfi{}) are governed by streamwise vortex pairs that are localised within the vicinity of the cruciform juncture when $\ured \geq \urei$, producing power in the order of $O(10^{0})$ \si{\milli\watt}. The highest efficiency is attained in the neighbourhood of $\ured = \urei$.
  \item The small amplitude vibrations of the steep-angled cruciforms (\angfo{} and \angth{}) are due to asymmetrical distribution of vortical structures relative to the $Z = 0$ plane, producing sub-\si{\milli\watt} power. The highest efficiency is attained in the neighbourhood of $(\ured,\alpha) = (\urtw,\angfo)$.
  \item The large amplitude vibrations of the shallow-angled cruciforms (\angtw{} and \angon{}) are due to the shedding of three-dimensional Karman vortices resulting from synchronisation of vortices shed from the cylinder and the strip plate, which end up locking into the natural frequency of the cylinder. The power produced by these cruciforms can reach an order of $O(10^{1})$ \si{\milli\watt}. The highest efficiency is attained in the neighbourhood of $(\ured,\alpha) = (\urfo,\angon)$.
\end{enumerate} \label{item:conclusion}

In the future, we will consider improving the resolution of the contour plots by investigating smaller increments of the cruciform angle to shed light on the sensitivity of each vibration-driving mechanism with respect to the cruciform layout.
% \section{Research Outcomes}
% \section{Contributions to Knowledge}
% \section{Future Works}

%select one
%Use authordate with natbib, comment if using numbering
\bibliographystyle{utmthesis-authordate}
%When using numbering, comment when using author-year
%Numbering does not use \citep nor citet (natbib)
%\bibliographystyle{utmthesis-numbering}
\bibliography{reference}

% \appendix
% \chapter{Time-series Data}
% Some data

%-----------List of Publication-----------%
%This is not required for FYP project report
\listofpublications

\noindent \textbf{Journal with Impact Factor}
\begin{enumerate}
\item Paper 1
\item Paper 2
\end{enumerate}
\textbf{Indexed Journal (SCOPUS)}
\begin{enumerate}
\item Paper 3
\end{enumerate}
\noindent \textbf{Non-Indexed Journal}
\begin{enumerate}
\item Paper 4
\end{enumerate}
\noindent \textbf{Indexed conference proceedings}
\begin{enumerate}
\item Paper 5
\end{enumerate}
\noindent \textbf{Non-Indexed conference proceedings}
\begin{enumerate}
\item Paper 6
\end{enumerate}

%This is required to make List of Appendices possible. Remove when have no appendix.
% \endmatter

\end{document}
