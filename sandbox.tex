%% Original abstract
  From off-grid charging of electronic devices to energising independent wireless sensor networks, the demand for stand-alone, low-power generators from renewable energy sources is becoming ever more prevalent.
  This study aims to address this need, by numerically investigating a cruciform energy harvester that comprises of an elastically supported circular cylinder, and a downstream strip plate at right angles in the Reynolds number range $1.1 \times 10^{3} \leq \text{Re} \leq 14.6 \times 10^{3}$ and Scruton number 9.94.
  The continuity and three dimensional, unsteady Reynolds-averaged Navier-Stokes equations are solved on the numerical domain using a free and open-source C++ library called OpenFOAM. The Spalart-Allmaras turbulence model is used to provide closure to the governing equations. Previous studies on the power output from a 10 mm diameter cylinder show that meaningful power generation only begins when the reduced velocity $\ured$, exceeds 15 and produces a consistent output in the order of 1 mW over the whole observation window.
  To improve upon this, a more fundamental understanding of why this observation takes place is indispensable. This is done by investigating the temporal evolution of the lift and displacement signals using the Hilbert-Huang transform, leading to the discovery of a route through which a significant quantity of energy is lost during energy harvesting. To eliminate said route, this work examines energy harvesting of a generalised cruciform structure, with varying cruciform angles, and discovered the following. For steep-angled cruciforms ($45 \le \alpha (\si{\degree}) \le 67.5$) this study found asymmetries in the vortical structures that prevents lock-in and thus high-amplitude vibrations from taking place. However, for shallow-angled cruciforms ($0 \le \alpha (\si{\degree}) \le 22.5$), this work discovered a high degree of symmetry in the distribution of vortical structures, leading to the onset of meaningful power generation as early as $\ured = \urfo$ up to the upper limit of observation, with a maximum mechanical power that is one order of magnitude larger than the highest reported by similar studies in the literature.
  Finally, the mechanical power and efficiency of the generalised cruciform energy harvester are presented as a map in cruciform angle-reduced velocity ($\alpha(\si{\degree})-\ured$) parameter space, thus making it possible for future engineers to tailor the design of their cruciform energy harvester to their specific power and efficiency needs.
