\documentclass[]{article}
\usepackage{lmodern}
\usepackage{amssymb,amsmath}
\usepackage{ifxetex,ifluatex}
\usepackage{fixltx2e} % provides \textsubscript
\ifnum 0\ifxetex 1\fi\ifluatex 1\fi=0 % if pdftex
  \usepackage[T1]{fontenc}
  \usepackage[utf8]{inputenc}
\else % if luatex or xelatex
  \ifxetex
    \usepackage{mathspec}
  \else
    \usepackage{fontspec}
  \fi
  \defaultfontfeatures{Ligatures=TeX,Scale=MatchLowercase}
\fi
% use upquote if available, for straight quotes in verbatim environments
\IfFileExists{upquote.sty}{\usepackage{upquote}}{}
% use microtype if available
\IfFileExists{microtype.sty}{%
\usepackage{microtype}
\UseMicrotypeSet[protrusion]{basicmath} % disable protrusion for tt fonts
}{}
\usepackage[unicode=true]{hyperref}
\hypersetup{
            pdfborder={0 0 0},
            breaklinks=true}
\urlstyle{same}  % don't use monospace font for urls
\usepackage{longtable,booktabs}
\usepackage{graphicx,grffile}
\makeatletter
\def\maxwidth{\ifdim\Gin@nat@width>\linewidth\linewidth\else\Gin@nat@width\fi}
\def\maxheight{\ifdim\Gin@nat@height>\textheight\textheight\else\Gin@nat@height\fi}
\makeatother
% Scale images if necessary, so that they will not overflow the page
% margins by default, and it is still possible to overwrite the defaults
% using explicit options in \includegraphics[width, height, ...]{}
\setkeys{Gin}{width=\maxwidth,height=\maxheight,keepaspectratio}
\IfFileExists{parskip.sty}{%
\usepackage{parskip}
}{% else
\setlength{\parindent}{0pt}
\setlength{\parskip}{6pt plus 2pt minus 1pt}
}
\setlength{\emergencystretch}{3em}  % prevent overfull lines
\providecommand{\tightlist}{%
  \setlength{\itemsep}{0pt}\setlength{\parskip}{0pt}}
\setcounter{secnumdepth}{0}
% Redefines (sub)paragraphs to behave more like sections
\ifx\paragraph\undefined\else
\let\oldparagraph\paragraph
\renewcommand{\paragraph}[1]{\oldparagraph{#1}\mbox{}}
\fi
\ifx\subparagraph\undefined\else
\let\oldsubparagraph\subparagraph
\renewcommand{\subparagraph}[1]{\oldsubparagraph{#1}\mbox{}}
\fi

\date{}

\begin{document}

MICRO-WATT ENERGY HARVESTING BY EXPLOITING STREAMWISE VORTEX INDUCED
VIBRATION IN WATER FLOWS

AHMAD ADZLAN FADZLI bIN kHAIRI

A dissertation submitted in partial fulfilment

Of the requirements for the degree of

Doctor of Philosophy

Malaysia-Japan International Institute of Technology

Universiti Teknologi Malaysia

MAY 2020

I hereby submit this thesis entitled ``Micro-watt energy harvesting by
exploiting streamwise vortex-induced vibration in water flows'' is the
product of my own work except for writings and summaries for which their
sources I have made evident.

Signature : ....................................................

Author : AHMAD ADZLAN FADZLI BIN KHAIRI

Date :

\emph{``My dearest wife, children, Dr. Mohamed Sukri and friends at the
WEE iKohza''}

This is for all of you.

\protect\hypertarget{_Toc41048784}{}{}ACKNOWLEDGEMENTS

I would like to first express my highest sense of gratitude to my
supervisor Dr. Mohamed Sukri Mat Ali. His knowledge and proficiency in
computational fluid dynamics have made my entry into this field of study
a little bit more bearable and has since been an excellent tool in
making novel discoveries and conclusions. His open stance in receiving
ideas and suggestions to strengthen the foundations of this research has
been fundamental to the preservation of the originality and timeliness
of this work. I have learned a lot from him about the value of logical
continuity and how to achieve it throughout the course of completing
this work. The relentless effort for logical continuity throughout the
thesis has been, in my opinion, crucial towards a manuscript that not
only is easily accessible but helps target audiences to build upon this
work in the future, hence advancing the field as a whole.

As for my beloved family, especially my wife and two babies, I really
cannot think of any way to repay your hardships and understanding
throughout the years of my study. Even a lifetime of unconditional love
and devotion towards all of you squares only but a small fraction of
what you have given up making my studies a reality. To that end I can
only trust the Most Merciful to fully even the debt and make all of you
among those who are most beloved by the Most Gracious. I pray to the
Almighty to accept the fruit of our jihad in finding knowledge and may
all the trials and tribulations we have encountered together,
physically, psychologically and financially, will stand witness in the
hereafter that we have answered His call towards submission, and His
call towards success. Aamin ya Rabbal `alamin.

\protect\hypertarget{_Toc328857619}{}{\protect\hypertarget{_Toc328858959}{}{\protect\hypertarget{_Toc328859064}{}{\protect\hypertarget{_Toc344911337}{}{}}}}

\protect\hypertarget{_Toc461032217}{}{\protect\hypertarget{_Toc461032367}{}{\protect\hypertarget{_Toc461035508}{}{\protect\hypertarget{_Toc461036251}{}{\protect\hypertarget{_Toc461036427}{}{\protect\hypertarget{_Toc461037231}{}{\protect\hypertarget{_Toc461037414}{}{\protect\hypertarget{_Toc483907979}{}{\protect\hypertarget{_Toc41048785}{}{}}}}}}}}}ABSTRACT

Drainage monitoring for flood early warning systems and small cells used
by 5G networks are examples of widely distributed low-electronic
devices. To become fully independent of the national grid, these devices
must 1) harness power from a renewable source, 2) produce only the
required amount of power for safe operation, and 3) is easily scalable.
Koide et al. demonstrated one method to achieve the first two
requirements. The type of vortex that triggers the vibration of the
cylinder in their work is not the Karman vortices usually shed from
isolated cylinders, but rather a pair of streamwise vortices shed
alternately from the top and bottom of the cylinder. However, while
energy harvesters of the Karman vortex type can be scaled at a fixed
flow velocity by placing several additional cylinders downstream the
leading cylinder, a simple method to increase the power harnessed at a
fixed flow velocity for the streamwise vortex type remains an open
question. Since the streamwise vortices form at the juncture between the
cylinder, we hypothesise that more of them can be triggered by placing
additional interfering plates downstream the cylinder to gain more lift,
and hence, power from the flow. To test this hypothesis, we conduct an
experimental investigation of a two-plate setup and vary its spacing
from each other. We aim to elucidate new cylinder response regimes,
identify optimal power regions from a power - plate spacing - flow
velocity map, and propose an explanation regarding the observed regimes
from the perspective of vortex shedding and lift - cylinder displacement
phase difference. The result of this study is an improved guideline for
the design of an easily scalable streamwise vortex type energy harvester
that produces power within the optimal operating range of the device.

\protect\hypertarget{_Toc41048786}{}{}TABLE OF CONTENTS

\textbf{CHAPTER ITEM PAGE}

DECLARATION ii

DEDICATION iii

\protect\hyperlink{_Toc41048784}{ACKNOWLEDGEMENTS iv}

\protect\hyperlink{_Toc41048785}{ABSTRACT v}

\protect\hyperlink{_Toc41048786}{TABLE OF CONTENTS vi}

\protect\hyperlink{_Toc41048787}{LIST OF TABLES ix}

\protect\hyperlink{_Toc41048788}{LIST OF FIGURES x}

\protect\hyperlink{_Toc41048789}{LIST OF ABBREVIATIONS xv}

\protect\hyperlink{_Toc41048790}{LIST OF SYMBOLS xvi}

\protect\hyperlink{_Toc41048791}{CHAPTER 1 1}

\protect\hyperlink{introduction}{1.1 Introduction 1}

\protect\hyperlink{background}{1.2 Background 2}

\protect\hyperlink{problem-statement}{1.3 Problem Statement 5}

\protect\hyperlink{research-objectives}{1.4 Research Objectives 6}

\protect\hyperlink{research-questions}{1.5 Research Questions 7}

\protect\hyperlink{significance-of-study}{1.6 Significance of Study 8}

\protect\hyperlink{_Toc41048798}{CHAPTER 2 10}

\protect\hyperlink{_Toc41048799}{2.1 Karman vortex-induced vibration
(KVIV) energy harvester 10}

\protect\hyperlink{_Toc41048800}{2.2 Streamwise vortex-induced vibration
(SVIV) energy harvester 12}

\protect\hyperlink{_Toc41048801}{2.3 Sensors for River/Drainage
Monitoring 17}

\protect\hyperlink{strip-plate-at-an-angle-as-the-modified-cruciform-configuration}{2.4
Strip-plate at an Angle as the Modified Cruciform Configuration 19}

\protect\hyperlink{_Toc41048803}{CHAPTER 3 21}

\protect\hyperlink{problem-geometry}{3.1 Problem Geometry 21}

\protect\hyperlink{numerical-method}{3.2 Numerical Method 23}

\protect\hyperlink{dynamic-mesh-motion}{3.3 Dynamic Mesh Motion 26}

\protect\hyperlink{_Toc41048807}{3.4 Experimental Validation of
Benchmark Case 27}

\protect\hyperlink{grid-independency-study}{3.5 Grid Independency Study
31}

\protect\hyperlink{_Toc41048809}{CHAPTER 4 38}

\protect\hyperlink{_Toc41048810}{4.1 Amplitude Response 38}

\protect\hyperlink{frequency-response}{4.2 Frequency Response 42}

\protect\hyperlink{ensemble-empirical-mode-decomposition-eemd-and-hilbert-transform}{4.3
Ensemble Empirical Mode Decomposition (EEMD) and Hilbert Transform 43}

\protect\hyperlink{phase-lag-in-the-kviv-regime-mathbfumathbfmathbf-14}{4.4
Phase lag in the KVIV regime () 44}

\protect\hyperlink{_Toc41048814}{4.5 Transition to SVIV () 49}

\protect\hyperlink{the-sviv-regime-mathbfumathbfmathbf-20}{4.6 The SVIV
Regime, 55}

\protect\hyperlink{mathematical-model-for-power-estimation}{4.7
Mathematical Model for Power Estimation 58}

\protect\hyperlink{section-4}{CHAPTER 5 65}

\protect\hyperlink{the-amplitude-frequency-response}{5.1 The
amplitude-frequency response 66}

\protect\hyperlink{the-main-components-of-lift-driving-cylinder-vibration}{5.2
The main components of lift driving cylinder vibration 69}

\protect\hyperlink{estimated-power-map-in-mathbfalphamathbfumathbf-space-and-its-interpretation}{5.3
Estimated power map in -- space and its interpretation 71}

\protect\hyperlink{section-5}{CHAPTER 6 72}

\protect\hyperlink{conclusions}{6.1 Conclusions 72}

\protect\hyperlink{_Toc41048823}{REFERENCES 74}

\protect\hypertarget{_Toc41048787}{}{}LIST OF TABLES

\textbf{TABLE TITLE PAGE}

\protect\hyperlink{_Toc40994423}{Table 1: Sample power inputs for flow
velocity and water level monitoring equipment. 18}

\protect\hyperlink{_Toc12528554}{Table 1: Sample power inputs for flow
velocity and water level monitoring equipment. 18}

\protect\hypertarget{_Toc41048788}{}{}LIST OF FIGURES

\textbf{FIGURE TITLE PAGE}

Figure 1: Number of publications with keywords {[}"vortex induced
vibration" energy{]}. Retrieved from SCOPUS. 3

Figure 2: Apparent power \(\text{Pa}\) (W) versus reduced velocity
\(U*\) for cases of KVIV and TVIV. Adapted from (Koide \emph{et al.},
2013). 4

Figure 3: A schematic view of the periodic shedding of streamwise
trailing vortex pair that induces vibration in SVIV. The sketches are
not drawn to scale. 14

Figure 4: The domain size, not according to scale, used in this study.
This domain setup has been shown to successfully yield realistic results
from similar simulations found in (Maruai \emph{et al.}, 2017; Maruai
\emph{et al.}, 2018). 23

Figure 5: Schematic of the experimental setup. (a) presents a three
dimensional schematic of the experimental rig while (b) shows an
enlarged schematic of the damping system. 28

Figure 6: Schematic and key dimensions from our experimental setup. The
acoustic Doppler velocimeter (ADV) was placed at the same location where
the cylinder is located during experimental runs. 29

Figure 7: A sample of the time history for cylinder displacement from a
test run of our experimental setup. The value of \(U* = 22.7\). 31

Figure 8: Three meshes used in the grid convergence study. (a), (b) and
(c) show the coarse, medium and fine meshes viewed perpendicular to
three main viewing positions: from the inlet, the top and the front,
which is looking directly at the cylinder end. 33

Figure 9: The convergence diagram for \(yRMS*\). (a) shows how \(yRMS*\)
converges close to the Richardson extrapolation value while (b) shows
how the error (difference between the value obtained from a particular
mesh and the Richardson extrapolation) decreases with decreasing grid
spacing. 35

Figure 10: The convergence diagram for \(f*\). (a) shows how \(f*\)
converges close to the Richardson extrapolation value while (b) shows
how the error (difference between the value obtained from a particular
mesh and the Richardson extrapolation) decreases with decreasing grid
spacing. 36

Figure 11: The convergence diagram for \(\text{ClRMS}\). (a) shows how
\(\text{ClRMS}\) converges close to the Richardson extrapolation value
while (b) shows how the error (difference between the value obtained
from a particular mesh and the Richardson extrapolation) decreases with
decreasing grid spacing. 37

Figure 12: The amplitude and frequency response of our cruciform system,
in lieu of results from (Nguyen \emph{et al.}, 2012; Koide \emph{et
al.}, 2013). (a) shows the amplitude response using \(yRMS*\), (b) the
frequency response using \(f*\) and (c) also the frequency response, but
using the Strouhal number for vibration St. 39

Figure 13: The time series of cylinder displacement between
\(18 < U* < 30\). (a) groups the cylinder displacement signal between
\(18 < U* < 23\), where there seems to be an increase in intermittency
in the displacement signal, while (b) groups the cylinder displacement
signal between \(25 \leq U* < 30\), where the intermittency in the
displacement signal vanishes. 41

Figure 14: Temporal analysis of the lift coefficient and normalised
cylinder displacement signal at \(U* = 4.5\). We show the lift
coefficient and normalised cylinder displacement signal side by side in
(a), present the temporal evolution of the phase lag \(\phi\) of Cl in
(b) and show the temporal evolution of the instantaneous frequency of
the lift coefficient signal in (c). The blue line in (a) represents the
lift coefficient signal, while the black line represents the normalised
cylinder displacement. 45

Figure 15: Temporal analysis of the lift coefficient and normalised
cylinder displacement signal at \(U* = 6.8\). We show the lift
coefficient and normalised cylinder displacement signal side by side in
Fig. (a), present the temporal evolution of the phase lag \(\phi\) of Cl
in (b) and show the temporal evolution of the instantaneous frequency of
the lift coefficient signal in (c). The blue line in (a) represents the
lift coefficient signal, while the black line represents the normalised
cylinder displacement. 46

Figure 16: Temporal analysis of the lift coefficient and normalised
cylinder displacement signal at \(U* = 13.6\). We show the lift
coefficient and normalised cylinder displacement signal side by side in
(a), present the temporal evolution of the phase lag \(\phi\) of Cl in
(b) and show the temporal evolution of the instantaneous frequency of
the lift coefficient signal in (c). The blue line in (a) represents the
lift coefficient signal, while the black line represents the normalised
cylinder displacement. 48

Figure 17: Temporal evolution of \(y*\) and Cl at \(U* = 15.9\). (b)
shows an enlarged view of (a). We can barely spot semblance of two
signals with different amplitudes superimposed in the Cl signal in (b).
50

Figure 18: Temporal analysis of the lift coefficient component that has
the highest correlation to the original (normalised) cylinder
displacement signal, \(CCl,\ y*\), and the normalised cylinder
displacement signal at \(U* = 15.9\). The component was obtained by
decomposing the lift coefficient signal using EEMD. We show \(CCl,y*\)
and \(y*\) signal side by side in (a), present the temporal evolution of
the phase lag \(\phi\) of \(CCl,y*\) in (b) and show the temporal
evolution of the instantaneous frequency of the \(CCl,y*\) in (c). The
blue line in (a) represents the lift coefficient component signal, while
the black line represents the normalised cylinder displacement. 51

Figure 19: Temporal analysis of the lift coefficient component that has
the highest correlation to the original (normalised) cylinder
displacement signal, \(CCl,\ y*\), and the normalised cylinder
displacement signal at \(U* = 18.2\). The component was obtained by
decomposing the lift coefficient signal using EEMD. We show \(CCl,y*\)
and \(y*\) signal side by side in (a), present the temporal evolution of
the phase lag \(\phi\) of \(CCl,y*\) in (b) and show the temporal
evolution of the instantaneous frequency of the \(CCl,y*\) in (c). The
blue line in (a) represents the lift coefficient component signal, while
the black line represents the normalised cylinder displacement. 54

Figure 20: The instantaneous phase lag \(\phi\) of the dominant
component of the normalised cylinder displacement signal (\(y*\))
against \(CCl,y*\) in the range \(20 < U* < 30\). See Figure 19 for the
definition of \(CCl,y*\). 56

Figure 21: Vibration regimes identified during analysis of \(\phi\). To
capture the evolution of \(\phi\) with respect to \(U*\), a
representative value for \(\phi\) at each \(U*\) must be selected. We
chose to use the mean \(\phi\) as the representative value. 57

Figure 22: The evolution of \(yRMS*\) with respect to \(U*\) and plate
tilt angle. 66

Figure 23: The evolution of \(FL,RMS\) with respect to \(U*\) and plate
tilt angle. 66

Figure 24: Sketches of the vortical structure in the moderate to high
\(U*\) region (\(11 < U* < 30\)). The sketches are made based on the
flow visualisation at maximum lift in a vibration cycle. This definition
is visualised in the lift/displacement versus time sketch at the top
left of the figure. The crucifix system is visualised from downstream.
The vortical structures at \(\theta plate = 0\ rad\) and
\(\theta plate = \pi 8\ rad\) are not streamwise vortex in the usual
sense as observed in \(\theta plate = \pi\ 2rad\). Rather, they are the
streamwise undulation of Karman vortices. The vortical structures
driving the vibrations at \(\theta plate = 0\ rad\) and
\(\theta plate = \pi 8\ rad\) are Karman vortices with streamwise
undulations. 67

Figure 25: Sketches of the vortical structure in the low \(U*\) region
(\(2 < U* < 10\)). The sketches are made based on the flow visualisation
at maximum lift in a vibration cycle. This definition is visualised in
the lift/displacement versus time sketch at the top left of the figure.
The crucifix system is visualised from downstream. The vortical
structures summarised are not streamwise vortex in the usual sense as
observed when \(\theta plate = \pi\ 2rad\) at moderate to high \(U*\).
Rather, they are the streamwise undulation of Karman vortices. The
vortical structures driving the vibrations (or lack thereof) in the low
\(U*\) region are Karman vortices with streamwise undulations. 68

Figure 26: Evolution of the lift force with varying \(U*\). The results
are summarised for every tilt angle of the strip plate. The horizontal
axis denotes reduced velocity while the vertical axis denotes the
root-mean-square of the IMF component of lift that is most correlated to
the dominant IMF component of \(y*\) (blue point) and the IMF component
of lift with the highest root-mean-square amplitude (grey point). 69

Figure 27: Evolution of the lift force with varying \(U*\). Like Figure
33, but with enlarged vertical axes to ease comparison between tilt
angles. Refer to Figure 33 for caption. 70

Figure 28: The ratio of the root-mean-square amplitude of the IMF
component of lift most correlated to the \(y*\) signal, to the sum
between the root-mean-square amplitude of the IMF component of lift most
correlated to the \(y*\) signal and the IMF component of lift with
maximum root-mean-square amplitude. 70

Figure 29: The estimated apparent power computed using the following
formula, \(Pa = 8\pi 3meff.\zeta tot.Afosc2fn,water\), based on the
power estimation formula for mechanical power described in (Sun \emph{et
al.}, 2015). 71

\protect\hypertarget{_Toc41048789}{}{}LIST OF ABBREVIATIONS

\begin{longtable}[]{@{}lll@{}}
\toprule
VIV & - & \begin{quote}
Vortex-induced vibration
\end{quote}\tabularnewline
\midrule
\endhead
KVIV & - & \begin{quote}
Karman vortex-induced vibration
\end{quote}\tabularnewline
SVIV & - & \begin{quote}
Streamwise vortex-induced vibration
\end{quote}\tabularnewline
\bottomrule
\end{longtable}

\protect\hypertarget{_Toc41048790}{}{}LIST OF SYMBOLS

\section{\texorpdfstring{\protect\hypertarget{_Toc461037239}{}{\protect\hypertarget{_Toc461037422}{}{\protect\hypertarget{_Toc41048791}{}{}}}}{}}\label{section}

\textbf{INTRODUCTION}

\hypertarget{introduction}{\subsection{Introduction}\label{introduction}}

Vortex-induced vibration (VIV) is a type of vibration that grows from
instabilities in fluid flows moving past a solid object, i.e. bluff
body. When the flow exceeds a critical velocity, the flow develops
vortices that are shed alternately downstream the bluff body. This
triggers the onset of unsteady lift and drag forces that initiate and
sustain its vibration (Nakamura \emph{et al.}, 2013). Numerous
occurrences of VIV are readily observable in the field of engineering.
In the ocean, marine currents give rise to the vibration of risers and
offshore drilling platforms (Quen \emph{et al.}, 2014; Vandiver \emph{et
al.}, 2009; Xu \emph{et al.}, 2009). Up in the sky, aeroplane wings
vibrate, and high-rise buildings experience sway as strong gusts blow
around the mighty structures (Tanaka \emph{et al.}, 2012; Kim and You,
2002). Closer to the ground, power transmission lines vibrate as the
result of wind blowing over them (Barry \emph{et al.}, 2010; Diana
\emph{et al.}, 2003).

The common denominator for all these examples is the potential damage to
the engineering construct experiencing it. Thus, methods are devised and
implemented to mitigate the effects of the vibrations by dissipating the
vibrational energy or delaying/aborting its onset in the first place.

However, the past decade has seen efforts to do exactly the opposite:
purposely maximising the vibration induced by the vortices, with the aim
to generate electrical power. The simplicity of design and scalability
attracts many to contribute to this multidisciplinary field of study,
along with the prospect of successful development and subsequent
commercialization of a new generation of energy harvester.

\hypertarget{background}{\subsection{Background}\label{background}}

The term ``flow-induced vibration'' refers to a wide range of phenomena:
flutter (Xia, S. S. Michelin, \emph{et al.}, 2015; Doare and Michelin,
2011; Pineirua \emph{et al.}, 2015; Xia, S. Michelin, \emph{et al.},
2015), galloping (Kluger \emph{et al.}, 2013; Barrero-Gil \emph{et al.},
2010; Luo \emph{et al.}, 2003; Chen \emph{et al.}, 2012),
turbulence-induced vibration (Nakamura \emph{et al.}, 2013),
wake-induced vibration (Ogink and Metrikine, 2010; Bearman, 2011; Assi
\emph{et al.}, 2013; Derakhshandeh \emph{et al.}, 2014), and VIV of
various kinds, which are the main object of study in this proposal. The
study of VIV is traditionally motivated by the potential failure of
engineering structures resulting from the fluid moving around them
(Larsen and Halse, 1997; Khalak and Williamson, 1999; Shiraishi \emph{et
al.}, 1988; Nakashima, 1986). Nevertheless, technical publications since
the 2000s saw a surge in contributions toward the subject from the
perspective of energy harvesting. A simple search in SCOPUS shown in
Figure 1 reveals this trend for keywords {[}``vortex induced vibration''
energy{]} for the last 4 decades.

At the cutting edge of this field of research is a group at The
University of Michigan, that has already built prototypes of the energy
harvester, named VIVACE (Norman, 2012). They compared the cost of power
production in \(\$/kW \cdot h\) between VIVACE and a wide selection of
common (pulverised coal, integrated gasification combined cycle, natural
gas combined cycle, etc.) and new power generation technologies
(anaerobic digester, landfill gas, solar, etc.). The result of this
comparison demonstrated how VIVACE is on par in terms of power
production cost with most of the technologies it was contrasted to (M.
M. M. Bernitsas \emph{et al.}, 2008).

\protect\hypertarget{_Toc520544808}{}{}\includegraphics[width=4.09547in,height=2.75591in]{media/image1.wmf}

\protect\hypertarget{_Ref520282982}{}{\protect\hypertarget{_Toc520544685}{}{\protect\hypertarget{_Toc41048824}{}{}}}Figure
1: Number of publications with keywords {[}"vortex induced vibration"
energy{]}. Retrieved from SCOPUS.

The VIV phenomenon utilised by the team at the University of Michigan is
of the Karman VIV type (KVIV), capable of producing power in the order
of MW when installed as a large-scale energy farm (Raghavan, 2007).
However, as pointed out by Koide et al. (Koide \emph{et al.}, 2013) the
reduced velocity (\(U^{*}\)) range within which KVIV can be relied upon
for power generation is about one order of magnitude smaller than what
can be expected from another form of VIV namely the streamwise VIV
(SVIV). Reduced velocity \(U^{*}\) is a nondimensional form of
characteristic velocity that allows comparison of results between
similar systems of differing dimensions. Since SVIV power generation is
possible for a large range of \(U^{*}\), it is better suited for
deployment in flows with large velocity changes.

\protect\hypertarget{_Toc520544809}{}{}\includegraphics[width=1.60458in,height=0.47244in]{media/image2.wmf}\includegraphics[width=4.34099in,height=2.83465in]{media/image2.wmf}

\protect\hypertarget{_Toc520544686}{}{\protect\hypertarget{_Toc41048825}{}{}}Figure
2: Apparent power \(\mathbf{P}_{\mathbf{a}}\) (W) versus reduced
velocity \(\mathbf{U}^{*}\) for cases of KVIV and TVIV. Adapted from
(Koide \emph{et al.}, 2013).

Despite this, the main shortcoming of SVIV is its maximum power output
which is demonstrated at this stage of development to cap at a mW scale
for a single-cylinder setup. An isolated cylinder setup for KVIV
produces a maximum power in the order of 10 W (Bernitsas \emph{et al.},
2009). The apparent power \(P_{a}\) (W) for both KVIV and SVIV is shown
in (Koide \emph{et al.}, 2013). Following this \emph{present} limitation
of the \emph{unoptimized} SVIV energy harvesters, their application is
currently limited to mW electronics e.g., sensors and signal
transmitters.

One of the most immediate uses of such sensors in Malaysia is part of a
flood early warning system. Flood forecasting is possible without much
difficulty using conventional methods if there is a sufficient number of
hydrological observatories along the body of water (Department of
Irrigation and Drainage Malaysia, 2010). For this to take place,
electricity must be available to run the observatories, in addition to
favourable terrain near the body of water to house the equipment.

In urban areas, even though sourcing electricity may not be a hurdle,
placement of the observatory itself can be, due to land ownership
issues. Nevertheless, these issues do not in any way lessen the need to
have an adequate number of observatories in urban areas. After all,
urban areas are known to bear a greater risk of flash floods compared to
rural areas owing to disrupted natural systems of runoff production as a
result of poorly planned development (Takaijudin \emph{et al.}, 2010;
Abdullah, 2004; Abdullah, 2011, p.3). Therefore, the prospect of
devising a simple river monitoring system that consists of only a few
sensors and electronics totalling to a maximum combined number of three
(3), powered by a TVIV energy harvester hence becomes worthy of further
inquiry.

\hypertarget{problem-statement}{\subsection{Problem
Statement}\label{problem-statement}}

The preceding section has established the viability of harnessing energy
from a flow by exploiting the VIV phenomenon. Multiple modes of VIV have
been observed, and SVIV stands out as better oriented for deployment in
fluid flows that vary greatly in terms of free-stream velocity. Even
with very rudimentary optimisations, SVIV has demonstrated its ability
to generate power in the order of mW consistently over a large range of
free-stream velocities. This can be harnessed to develop a
self-contained river monitoring device that deploys with ease,
especially in urban neighbourhoods to facilitate early warning of
floods.

To achieve this, the problems outlined below must be addressed to close
relevant gaps in the current body of knowledge.

\begin{enumerate}
\def\labelenumi{\roman{enumi})}
\item
  A lack of understanding on the transition mechanism from Karman to
  streamwise vortex-induced vibration.
\item
  A paucity in the knowledge on what contributes to the magnitude of the
  alternating lift force acting on the cylinder, and its vibrational
  frequency components.
\item
  A deficiency of new methods to control the flow perturbation which
  gives rise to a strong, stable and periodic forcing of the cylinder
  vibration, sustainable over the desired operating range of \(U^{*}\).
\end{enumerate}

Addressing the above problems will provide a better understanding on the
origins of the streamwise vortex pairs, uncover new perspectives on
variables that affect the alternating lift force in the context of
streamwise vortex pairs, and generate novel insight on new flow regimes,
which can enlighten us to a better configuration of the energy
harvester, i.e. producing more power than ever before.

\hypertarget{research-objectives}{\subsection{Research
Objectives}\label{research-objectives}}

Following the problems outlined in the previous section, the objectives
that define the scope of work in this proposal are listed below.

\begin{enumerate}
\def\labelenumi{\roman{enumi})}
\item
  To investigate what takes place when the dominant vortical structure
  forcing the vibration transitions from Karman to streamwise
  vortex-induced vibration in terms of lift and vibration signals.
\end{enumerate}

\begin{enumerate}
\def\labelenumi{\roman{enumi})}
\item
  To characterise the lift signal in terms of its components and how the
  components interact to modify the frequency-amplitude response of the
  cylinder.
\item
  To propose a new passive control method for an SVIV-oriented energy
  harvester that modifies the vortical structures and their distribution
  around the oscillator to control its region of operability.
\end{enumerate}

The aim of objective i) is to get a better understanding of how the
advent of streamwise vortical structures perturb the lift acting on the
cylinder, which directly modifies the vibration signal of the cylinder.
Objective ii) is an attempt to identify the footprint of dominant
vortical structures in the flow in the lift signal and relate those to
the resulting vibration signal. Finally, objective iii) seeks to
recommend and evaluate a modified version of the cruciform structure
that alters the vortical structure in the flow, thus modulating the lift
signal acting on the cylinder and its frequency-amplitude response. The
power envelope will give us a more generalised operability condition for
the energy harvester, and how we can vary the cruciform configuration to
cater to a particular flow environment.

\hypertarget{research-questions}{\subsection{Research
Questions}\label{research-questions}}

The answer to several questions is sought in this proposed study. These
questions are meant to drive the study towards its objectives.

\begin{enumerate}
\def\labelenumi{\roman{enumi})}
\item
  How does the lift signal evolve as the flow transitions from being
  driven primarily by Karman vortex to streamwise vortex?
\end{enumerate}

\begin{enumerate}
\def\labelenumi{\roman{enumi})}
\item
  How does the ratio of energy transferred from the flow to the lift
  components evolve with respect to \(U^{*}\)?
\item
  What deviations do the modified cruciform configuration impose on the
  vortical structure, vis-à-vis the vortical structure observed around a
  pure cruciform?
\item
  How do the deviations mentioned in iii) affect the lift magnitude, and
  by extension the frequency-amplitude response?
\item
  Where in the power envelope can we obtain maximum (minimum) power with
  the largest (narrowest) operability range, and how does this translate
  into a new mode of flow control to suit the operating conditions of
  the cruciform energy harvester?
\end{enumerate}

Concrete steps to be taken to address the questions above are outlined
in CHAPTER 3.

\hypertarget{significance-of-study}{\subsection{Significance of
Study}\label{significance-of-study}}

\protect\hypertarget{_Hlk21508814}{}{}As a whole, Malaysia receives an
average rainfall of about 2990 mm annually (Harris \emph{et al.}, 2014).
East Malaysia registers an approximate 3250 mm - nearly 1000 mm in
excess of the average rainfall figures for west Malaysia which is about
2500 mm (Department of Irrigation and Drainage Malaysia, n.d.).
Contribution to these values is greater during the monsoon seasons which
occur during the months of November to March (north-west monsoon) (Kuala
Lumpur Monsoon Activity Centre, 2015) and May to September (south-west
monsoon) (Billa \emph{et al.}, 2004). The increased rainfall during the
monsoon season inevitably saturates catchments, producing several times
the usual amount of runoff that rivers can drain to the sea within
sufficient time (Chia, 2004). Consequently, water lever rises past the
river banks and progresses into the surrounding terrain. This is how
floods commonly come into being during the monsoon season.

Outside the monsoon season, the inception of floods is due to convective
rainfall. The flood frequency and extent due to convective rain are
worse in an urban setting compared to rural areas. The underlying cause
of this is none other than the major disruption of pre-urbanization
mechanisms that govern the original rainfall-runoff processes of an
area. Thus, crippling floods can manifest within a matter of hours from
the start of rainfall, i.e. flash floods.

Structural efforts to put flash floods in check are almost always very
costly - RM730 million allocated in the 2016 Malaysia Budget alone
(Kementerian Kewangan Malaysia, 2015, p.45). However, the rationale for
structural solutions to the floods becomes increasingly suspect
considering the worsening climate change in recent decades. On the other
hand, non-structural methods of reducing flood damages such as flood
forecasting and warning protocols, receive only about one-twelfth of the
allocation budgeted for structural methods (Kementerian Kewangan
Malaysia, 2015, pp.45--46). This fact, along with the decreasing
effectiveness of structural efforts highlights the need to develop an
inexpensive, simple, yet reliable non-structural system to tackle the
flood problem.

Against this background, this study aims to contribute significantly
towards the development of an in-situ device to power water level and
flow velocity/discharge sensors for a given river/drainage system using
vibration energy harnessed from the flow itself. The minimalist and
self-powered feature of the device allows installation of as many of the
devices as necessary along the river/drainage system for adequate
monitoring, even in places where reliance on solar power is negated.
Finally, pursuant to further scaling studies, the upscaled version of
the energy harvester may one day be on par with diesel generators for
use at off-grid areas.

\section{\texorpdfstring{\protect\hypertarget{_Toc461037248}{}{\protect\hypertarget{_Toc461037431}{}{\protect\hypertarget{_Toc41048798}{}{}}}}{}}\label{section-1}

\textbf{LITERATURE REVIEW}

\subsection{\texorpdfstring{\protect\hypertarget{_Toc41048799}{}{\protect\hypertarget{_Hlk21508927}{}{}}Karman
vortex-induced vibration (KVIV) energy
harvester}{Karman vortex-induced vibration (KVIV) energy harvester}}\label{karman-vortex-induced-vibration-kviv-energy-harvester}

\protect\hypertarget{_Hlk525229593}{}{}The canonical KVIV energy
harvester employs a circular cylinder as its bluff body (Williamson and
Govardhan, 2004; Ahsan, 2015; Hayashi \emph{et al.}, 2006). This bluff
body is suspended in a flow using an elastic support, and flow
instabilities trigger the formation of Karman vortices that induce a
periodic variation of the lift and drag forces. When the motion of the
bluff body is constrained to the mean flow normal, periodic variation of
the lift force produces a one-dimensional KVIV.

Efforts to improve the efficiency of KVIV energy harvesters have seen
experimental and numerical studies of various aspects of the technology
such as a) bluff body geometry, number and arrangement (Derakhshandeh
\emph{et al.}, 2014; Gonçalves \emph{et al.}, 2013; Seyed-Aghazadeh
\emph{et al.}, 2015), b) passive wake control (Sukri Mat Ali \emph{et
al.}, 2011; Ali \emph{et al.}, 2012), c) restoring force i.e., springs
(Mackowski and Williamson, 2013), d) damping (Govardhan and Williamson,
2006), and e) turbulence control (Wu \emph{et al.}, 2011).

Most non-circular cylinders studied in a) - b) are stationary i.e.,
having zero degrees of freedom. It follows that vibrational behaviour of
non-circular cylinders - galloping included - has received limited
coverage in the literature. Galloping is known to produce bigger
amplitudes of the bluff body displacement compared to simple KVIV
(Barrero-Gil \emph{et al.}, 2010; Assi \emph{et al.}, 2010), thus
potentially improving non-circular cylinder KVIV energy harvesters when
incorporated in the system.

Compared to circular cylinders, non-circular ones have fixed flow
separation points at their edges. Fixed flow separation points assist
the establishment of a linear relationship between the Reynolds and
Strouhal numbers, which is useful for flow velocity measurements
(Venugopal \emph{et al.}, 2011). Apart from that, the edges can be
modified to create steep pressure gradients that induce the formation
strong vortices. These vortices are known to be exploited by scramjet
engines for accelerated fuel-air mixing (Sunami \emph{et al.}, 2002;
Arai \emph{et al.}, 2011).

Aside from bluff body shape, other variables that affect non-circular
cylinder KVIV strength also stand to be optimised. Examples of these are
listed in c) - e), and related literatures deal almost exclusively with
circular cylinders. Whether the same results can be reproduced by
employing a square cylinder remains an open question. The inclusion of
some mode of passive wake control in KVIV of a square cylinder is even
more uncommon within the current research landscape. Passive wake
control methods, for example, splitter or interfering plates, are
relatively simple ways to control vortex-vortex and vortex-shear layer
interaction (Sukri Mat Ali \emph{et al.}, 2013; Mat Ali \emph{et al.},
2011). It is reasonable to imagine that more active research on this
subject will pave the way towards a low-key vortical flow control, akin
to those utilised by fishes (Borazjani and Daghooghi, 2013; Drucker and
Lauder, 1999; Liao \emph{et al.}, 2003) to facilitate their locomotion.

In KVIV, there exists different modes of vibration with different
resulting power characteristics. The simplest categorization, groups
them according to their amplitude response: the initial, upper and lower
branch (Williamson, 1996; Williamson and Govardhan, 2004). An additional
galloping branch also exists for certain configurations. For optimal
energy harnessing, one should operate their system in the upper branch,
where the vibration amplitude is the highest with least variation, and
the vibration frequency is locked into the natural frequency of the
system. Although the cylinder attains even higher vibration amplitudes
in the galloping branch, the upper branch still has better efficiency in
the sense that the cylinder harnesses a bigger percentage of energy from
the total energy conducted by the flow (Sun \emph{et al.}, 2016).

We can design the vibrational system so that the system operates in the
upper branch if the flow has a well-behaved, relatively narrow velocity
band, e.g. river and ocean currents (M. M. Bernitsas \emph{et al.},
2008; Raghavan, 2007). However, one may find such a method unsuitable
for flows with an intermittent, largely varying velocity, e.g. wind
flows. The large flow velocity variation reduces the probability of the
cylinder operating in the upper branch. Hence, most of the time, the
cylinder operates in the initial or lower branch, where the vibration
amplitude has a smaller magnitude and larger variance compared to the
upper branch. In the galloping branch, the system loses the
amplitude-limiting mechanism, in addition to a lower harnessing
efficiency compared to the upper branch. This means that we need to
apply a larger safety factor when harnessing energy in the galloping
regime in order to benefit from the large vibration amplitude, while
avoiding possible damage to the system from unexpectedly large outlier
vibration amplitude. Exploiting streamwise vortex-induced vibration
(SVIV) alleviates this concern.

\subsection{\texorpdfstring{\protect\hypertarget{_Ref40969711}{}{\protect\hypertarget{_Ref40969718}{}{\protect\hypertarget{_Toc41048800}{}{}}}Streamwise
vortex-induced vibration (SVIV) energy
harvester}{Streamwise vortex-induced vibration (SVIV) energy harvester}}\label{streamwise-vortex-induced-vibration-sviv-energy-harvester}

\protect\hypertarget{_Hlk525229690}{}{}Streamwise vortex-induced
vibration (SVIV) is a form of VIV that develops from a pair of
streamwise vortices, shed alternately from opposing sides of a bluff
body parallel to the mean flow. Among the earliest mention of this type
of vortex is in Ref. (Zdravkovich, 1981), from a configuration of two
cylinders forming a plus sign (┼). In this setting, the upstream
cylinder periodically sheds a pair of streamwise trailing vortex from
its top and bottom surfaces. This periodic formation of streamwise
trailing vortex pairs is seen by some as a welcome addition to the class
of fluid mechanical phenomena through which energy can be extracted, and
hence elucidating its fundamental mechanics becomes necessary. Several
studies were then undertaken to address this concern.

The authors of Ref. (Shirakashi \emph{et al.}, 1989) for example,
experimentally investigated the behaviour of vortex shed from the
upstream cylinder in the same plus sign configuration mentioned. The
vortex shedding frequency is linearly related to the characteristic
velocity of the flow, and the Strouhal number is 3 to 7 times smaller
than that of a Karman vortex. Additionally, the periodic lift
coefficient variation reached a maximum when the normalised gap between
the two cylinders (\(s/d\)) is close to 0.125.

The normalised gap is observed to influence the type of streamwise
vortex shed from the upstream cylinder. A cylindrical trailing vortex
pair is formed when \(0 \leq s/d < 0.25\), whereas a U-shaped trailing
vortex pair (necklace vortex, as named by the authors) forms between
\(0.25 < s/d < 0.5\ \) (Shirakashi \emph{et al.}, 1989; Bae \emph{et
al.}, 1992). The vortex shedding of both types of streamwise trailing
vortex is shown to be most pronounced at \(s/d = 0.08\) and
\(s/d = 0.28\) for the cylindrical and U-shaped vortex pairs
respectively (Takahashi \emph{et al.}, 1999).

Considering the qualitative response shown by the vortex pairs towards
the normalised gap, attempts to describe them quantitatively, naturally
follow. One of the findings in this respect is the variation of the
nondimensional frequency of vortex shedding i.e. Strouhal number, with
regards to the Reynolds number. For Reynolds number below 10000, the
Strouhal number increases with increasing Reynolds number. The Strouhal
number ceases to change as the Reynolds number exceeds 10000 (Bae
\emph{et al.}, 2001). Similar experiments were repeated in wind tunnels
of different sizes, and in a water tunnel. The results showed good
reproducibility in both mediums (Koide \emph{et al.}, 2004).

Despite the overall reproducibility, certain features of the flow in
wind tunnels differ to that in a water tunnel. For example, the onset
and range of sustenance of the streamwise trailing vortex vibration are
different in water from what it is in the air (Koide \emph{et al.},
2006). A suggested explanation is that the discrepancy is brought about
by the difference in mass ratio between the two. The mass ratio is the
ratio between the mass of the bluff body to the amount of fluid
displaced due to its motion.

To simplify and miniaturise the system for generating SVIV, there are
efforts to swap the downstream cylinder in the plus sign configuration
with something even geometrically simpler - for example, a rectangular
plate, as shown in (Kato \emph{et al.}, 2006). Studies were then
conducted to test the response of this new system, shown in Figure 3,
and compare it to the previous two-cylinder system. In (Kato \emph{et
al.}, 2007), experiments showed that both the circular and U-shaped
streamwise trailing vortices can be generated from this new system,
albeit at different values of \(s/d\), depending on \(w/d\), which is
the normalized plate width. The net fluctuating lift coefficient is also
shown to be similar to the values measured in a two-cylinder system
(Koide, Kato, \emph{et al.}, 2009).

\includegraphics[width=5.40208in,height=2.89610in]{media/image3.png}

\protect\hypertarget{_Ref40994137}{}{\protect\hypertarget{_Toc520544687}{}{\protect\hypertarget{_Toc41048826}{}{}}}Figure
3: A schematic view of the periodic shedding of streamwise trailing
vortex pair that induces vibration in SVIV. The sketches are not drawn
to scale.

In recent years, research on SVIV took a new turn in terms of design by
introducing a square cylinder as the upstream bluff body, coupled with a
downstream rectangular plate to form a plus sign (Kawabata \emph{et
al.}, 2009). In such a setting, galloping is observed when the reduced
velocity \(U^{*} > 15\). Furthermore, very large amplitudes of the bluff
body displacement are observed when \(1 < s/d < 2\), if the width of the
plate (\(w\)) is equal to the length of the sides of the square cylinder
(\(d\)), i.e. \(w/d = 1\). This bluff body change may be an attempt to
facilitate dual-use of the VIV energy harvester: power generation and
flow velocity measurement.

With regards to the application of SVIV for energy harvesting, related
literature branched into two areas: first, how the mechanical properties
of the harvester (e.g., mass ratio, damping, etc.) affects the amplitude
and frequency response of SVIV (Koide, Sekizaki, \emph{et al.}, 2009;
Nguyen \emph{et al.}, 2012; Koide \emph{et al.}, 2013) and second, how
the minutiae of the flow field affect the force acting on the cylinder
(Deng \emph{et al.}, 2007; Zhao and Lu, 2018; Koide \emph{et al.},
2017).

In the first focus area, researchers proposed a method to convert the
vibration into electrical power via a coil and magnet system, where the
coil moves with the vibrating cylinder, thus creating relative motion
against the magnet, placed inside the coil (Koide, Sekizaki, \emph{et
al.}, 2009). The increased damping due to energy harvesting however,
reduces the maximum vibration amplitude close to a factor of 4.
Amplitude reduction due to increased total damping was also mentioned in
(M. M. Bernitsas \emph{et al.}, 2008; Bernitsas and Raghavan, 2008;
Bernitsas \emph{et al.}, 2009). Further investigation in (Nguyen
\emph{et al.}, 2012) damping not only affects the amplitude response of
the cylinder, but also narrows the synchronisation region between vortex
shedding and cylinder vibration. Moreover, (Nguyen \emph{et al.}, 2012)
demonstrated a strong coupling between mass ratio and damping in
determining both the width of the synchronisation region and the maximum
amplitude response of the cylinder.

In the second focus area, investigators turned their attention to the
details of the flow containing streamwise vortex shedding. The
understanding that the forces driving the vibration of the cylinder
ultimately comes as a by-product of vortex shedding motivates the
community to examine the flow field closer. One such study carefully
shoots motion pictures of the dye-injected flow (Koide \emph{et al.},
2017) at Reynolds number approximately one order of magnitude smaller
than what was available in the literature. A lower Reynolds number (Re)
reduces the amount of visual turbulence in the flow, allowing a clearer
shot of the vortex structures. Their study also highlights the higher
level of turbulence produced by the circular cylinder--strip-plate
cruciform in contrast to the twin circular cylinder cruciform, which
diminishes the periodicity of vortex shedding. Although visually
enlightening, this and other more qualitative studies contribute little
towards improving our understanding of the relationship between vortex
shedding and the resulting lift. (Deng \emph{et al.}, 2007) demonstrated
a way to overcome such a shortcoming.

In their study, (Deng \emph{et al.}, 2007) examined the flow field of a
twin circular cylinder cruciform using computational fluid dynamics
(CFD). Their domain stretches \(28D\) in the streamwise direction,
\(16D\) in the transverse direction and \(12D\) in the spanwise
direction. They studied an Re range yet another order of magnitude
smaller than that studied by (Koide \emph{et al.}, 2017), possibly to
get an even clearer visualisation of the vortical structures with less
turbulence, and to ease computational requisites.

At a fixed \(\text{Re} = 150\), streamwise vortices form even at a gap
ratio of 2. This result differs quite strikingly from studies we
reviewed earlier (Koide \emph{et al.}, 2007; Koide \emph{et al.}, 2006),
conducted at an Re twice the order of magnitude of (Deng \emph{et al.},
2007). This perhaps indicate that the minimum gap ratio needed for the
onset of streamwise varies with respect to Re.

They also observed that when the gap ratio \(G\), which they denote as
\(L/D\) in their paper, increases from 3 to 4, the maximum amplitude of
the lift coefficient increases by almost threefold. This can be
attributed quite easily to the current vortex pair shed by the upstream
cylinder. The downstream cylinder immediately disturbs the pair shed
from the upstream cylinder when \(G = 3\). The lift coefficient
increases by about a factor of 3 when this immediate disturbance
diminishes at \(G = 4\). The visualisation of three-dimensional (3D)
vorticity isocontours enables us to quickly establish this link
vis-à-vis the lift coefficient signal. The authors use of CFD made this
possible.

A similar study (\(Re\sim O(10^{2})\)) by (Zhao and Lu, 2018) highlights
the immense utility of CFD as a tool to research SVIV or flow around a
cruciform in general. They computed the sectional lift coefficient along
the upstream cylinder and the time history of this sectional lift
coefficient points towards 2 different modes of vortex shedding namely,
parallel and K-shaped. They also brought to our attention the local flow
patterns that vary along the length of the upstream cylinder such as the
trailing vortex flow, necklace vortex flow and flow in the small gap
(denoted as SG flow). As shown by the discontinuities in the phase angle
of the sectional lift coefficient along the upstream cylinder, we begin
to wonder whether the total lift coefficient should even be considered
as the result of the streamwise vortex shedding alone, especially with
Karman vortex streamlines clearly forming as we move away from the
junction of the cruciform.

Up to this point, we made a sharp distinction between Karman and
streamwise vortex-induced vibration. Although one can argue the
appropriateness of such a distinction for the amplitude and frequency
response of the cylinder in KVIV and SVIV, we propose that such a clear
distinction may not be entirely possible for the lift signal of a system
in SVIV. The main reason behind this lies in the fact that the cylinder
continues to shed Karman vortices even after the onset for streamwise
vortex shedding and SVIV (Shirakashi \emph{et al.}, 1989; Zhao and Lu,
2018). This point lead us to hypothesise that the lift (coefficient)
signal is more appropriately viewed as the streamwise-Karman
vortex-induced composite lift signal. However, we could not find studies
that took this viewpoint in their investigation of SVIV or flow around a
cruciform in general.

\subsection{\texorpdfstring{\protect\hypertarget{_Ref520288997}{}{\protect\hypertarget{_Toc41048801}{}{}}Sensors
for River/Drainage
Monitoring}{Sensors for River/Drainage Monitoring}}\label{sensors-for-riverdrainage-monitoring}

At the current stage of SVIV energy harvester research, the forecasted
magnitude of power obtainable is in the order of mW. Therefore, as
mentioned in the previous chapter, the most immediate and
straightforward application of this technology in for powering sensors
for river/drainage monitoring. The drainage monitoring sites in Malaysia
generally operates via remote telemetry units (RTU). These telemetry
units consist of rain gauges and water level sensors. They are powered
using solar panels, and the data stream is then relayed to the master
station of that state. Several factors define the reliability of a solar
such as panel rating, power tolerance, cell efficiency, and temperature
coefficient. A reliable solar panel for use by telemetry units typically
cost around RM1,000 for 2 W, and around RM1,600 for 5W of output.

To facilitate future installation of drainage monitoring equipment, one
sensible way is to reduce the power supply cost of the telemetry units.
This means developing a power generation mechanism that can either:

\begin{enumerate}
\def\labelenumi{\roman{enumi})}
\item
  complement the solar power so that we can use a cheaper panel with a
  lower power rating,
\end{enumerate}

\begin{enumerate}
\def\labelenumi{\roman{enumi})}
\item
  work as a fully functioning alternative to the solar panel,
\end{enumerate}

or both. A sample of power requirements of drainage monitoring equipment
are summarized in Table 1.

\protect\hypertarget{_Ref500253149}{}{\protect\hypertarget{_Ref500253126}{}{\protect\hypertarget{_Toc520543710}{}{\protect\hypertarget{_Toc520545591}{}{\protect\hypertarget{_Toc12528554}{}{\protect\hypertarget{_Toc40994423}{}{}}}}}}Table
1: Sample power inputs for flow velocity and water level monitoring
equipment.

\begin{longtable}[]{@{}llll@{}}
\toprule
Measured quantity & Brand & Measured range & Power input\tabularnewline
\midrule
\endhead
\begin{quote}
Flow velocity
\end{quote} & Global Flow Probe FP111 & 0.1 m/s \textasciitilde{} 6.1
m/s & 0.012 W \textasciitilde{} 0.08 W\tabularnewline
& Global Water Electromagnetic flow sensors EX-100/200 & 0.08 m/s
\textasciitilde{} 6.09 m/s & 3 W \textasciitilde{} 6.25 W\tabularnewline
& OTT MF Pro Water flow meter & 0 m/s \textasciitilde{} 6 m/s & 0.12 W
(assuming current input of 20 mA)\tabularnewline
& OTT C2 Mini current meter & 0.025 m/s \textasciitilde{} 5 m/s & 0.18 W
(assuming current input of 20 mA)\tabularnewline
\begin{quote}
Water level
\end{quote} & Nile series 502/504/517 & 502: 1 m \textasciitilde{} 20 m,
504: 2 m \textasciitilde{} 40 m, 517: 2 m \textasciitilde{} 70 m & 0.135
W \textasciitilde{} 0.216 W\tabularnewline
& CRS 451 & 0 m \textasciitilde{} 5.1 m,

0 m \textasciitilde{} 10.2 m,

0 m \textasciitilde{} 20.4 m,

0 m \textasciitilde{} 50.9 m,

0 m \textasciitilde{} 102 m & 0.04 W (assuming operating voltage of 10
V)\tabularnewline
& SR50A-L & 0.5 m \textasciitilde{} 10 m & 2.5 W\tabularnewline
& ToughSonic remote 50 & 0.305 m \textasciitilde{} 10 m & 0.864
W\tabularnewline
\bottomrule
\end{longtable}

A streamwise vortex-induced vibration (SVIV) method of generating power
has been investigated by Koide et al. (Koide \emph{et al.}, 2013), and
they have demonstrated the power generating capability of this setup.
However, there are several problems.

\begin{enumerate}
\def\labelenumi{\roman{enumi})}
\item
  The generated power is only in the order of \(10^{- 2}\) W.
\end{enumerate}

\begin{enumerate}
\def\labelenumi{\roman{enumi})}
\item
  Tangible power is only generated for reduced velocity \(U^{*} > 20\).
\item
  The efficiency of the generator decreases with increasing \(U^{*}\).
\end{enumerate}

As can be seen from Table 1, most of the drainage monitoring devices
require a power input in the order of 0.1 W. This is one order of
magnitude higher than the power observed by Koide et al. A \(U^{*}\)
threshold must also be exceeded to produce useful amount of power, which
means low-velocity drainage monitoring is difficult. Finally, a
decreasing efficiency means more energy left uncaptured at high
\(U^{*}\). Therefore, we need to address these issues and thus propose
and validate methods to:

\begin{enumerate}
\def\labelenumi{\roman{enumi})}
\item
  increase and sustain the \emph{magnitude} of generated power,
\end{enumerate}

\begin{enumerate}
\def\labelenumi{\roman{enumi})}
\item
  lower the \(U^{*}\) \emph{threshold} for significant power generation,
  and
\item
  modulate the \emph{operating range} of the generator,
\end{enumerate}

in contrast to the pure cruciform configuration.

\hypertarget{strip-plate-at-an-angle-as-the-modified-cruciform-configuration}{\subsection{Strip-plate
at an Angle as the Modified Cruciform
Configuration}\label{strip-plate-at-an-angle-as-the-modified-cruciform-configuration}}

The SVIV-based energy harvesting system is essentially a circular
cylinder oscillator system with a downstream passive flow control
mechanism. The canonical design calls for an arrangement where the axis
of the passive flow control mechanism -- usually some form of cylinder
-- is perpendicular to the axis of the oscillator. However, what happens
if one positions the passive flow control mechanism -- the strip-plate
-- at an acute angle to the axis of the oscillator is an open question.
Specifically, i) how the vortical structures around the cruciform is
altered, ii) how the changes in vortical structures affect the
alternating lift behaviour acting on the oscillator, and iii) how does
this impact the frequency-amplitude response and the estimated
obtainable mechanical power, when the strip-plate is tilted between
\(0 \leq \alpha\ \left( \text{rad} \right) < \pi/2\), \(\alpha\) being
the angle between the axis of the strip-plate and the axis of the
circular oscillator cylinder.

The desired outcome is an estimated mechanical power contour map against
reduced velocity \(U^{*}\) and \(\alpha\), that guides relevant
stakeholders in this technology in the process of tuning their system to
best suit the operating conditions where the energy harvester is going
to be installed.

\section{\texorpdfstring{\protect\hypertarget{_Toc461037259}{}{\protect\hypertarget{_Toc461037442}{}{\protect\hypertarget{_Toc41048803}{}{\protect\hypertarget{_Ref520286610}{}{}}}}}{}}\label{section-2}

\textbf{METHODOLOGY}

\hypertarget{problem-geometry}{\subsection{Problem
Geometry}\label{problem-geometry}}

The geometrical setup for this study builds on the work of (Maruai
\emph{et al.}, 2017; Maruai \emph{et al.}, 2018) who studied both
experimentally and numerically the FIM of a square cylinder with a
downstream flat plate. Their simulation results are in good agreement
with their own experiment, and with the experimental results of
(Kawabata \emph{et al.}, 2013), in the Reynolds number range
\(3.6 \times 10^{3} < \text{Re} < 12.5 \times 10^{3}\). This is well
within the Reynolds number studied in this work, i.e.
\(1.1 \times 10^{3} < \text{Re} < 14.6 \times 10^{3}\).

Our \(x - y\) plane fundamentally follows the dimensions used in (Maruai
\emph{et al.}, 2017; Maruai \emph{et al.}, 2018), except for the
cylinder shape, which in this study is circular, and the \(20D\)
distance to the outlet is measured from the downstream face of the
strip-plate. This is shown in Figure 4(a). We chose the cylinder-plate
gap \(G\) to be \(0.16D\), as previous works have shown this gap size
sustains the highest SVIV amplitude over the widest range of \(U^{*}\),
in comparison to other gap sizes.

As the problem geometry is explicitly three-dimensional (3D), the
\(x - y\) plane is extruded in the \(z\) direction, thus obtaining a 3D
domain. As can be seen in Figure 4(b), the circular cylinder extends
from \(z/D = 7.5\) to \(z/D = - 7.5\), while the strip-plate extends
from \(y/D = - 10.5\) to \(y/D = 10.5\). The \(z\)-direction extent is
set as \(z/D = \pm 7.5\) to provide enough spanwise length to allow
formation of the streamwise vortices. Based on the visualisations in the
experiments of (Koide \emph{et al.}, 2017), we reasoned that
\(z/D = \pm 7.5\) is already more than twice the spanwise reach of the
streamwise vortex, thus sufficient for the vortices to materialise in
our numerical solution. To compare, the spanwise extent of the numerical
study by (Deng \emph{et al.}, 2007), is \(z/D = \pm 6\) and the spanwise
extents of experiments by (Nguyen \emph{et al.}, 2012; Koide \emph{et
al.}, 2013), is \(z/D = \pm 5\).

\includegraphics[width=5.35417in,height=4.94805in]{media/image5.png}

\protect\hypertarget{_Ref520289508}{}{\protect\hypertarget{_Toc520544689}{}{\protect\hypertarget{_Toc41048827}{}{}}}Figure
4: The domain size, not according to scale, used in this study. This
domain setup has been shown to successfully yield realistic results from
similar simulations found in (Maruai \emph{et al.}, 2017; Maruai
\emph{et al.}, 2018).

\hypertarget{numerical-method}{\subsection{Numerical
Method}\label{numerical-method}}

The objectives of our study necessitate the solution of the continuity
and three-dimensional unsteady Reynolds averaged Navier-Stokes (3D
URANS) equations. We achieve this by using OpenFOAM, an open-source
computational fluid dynamics (CFD) platform written in C++.
Specifically, we work to solve the following continuity and URANS
equations.

\begin{longtable}[]{@{}lll@{}}
\toprule
& \(\frac{\partial U_{i}}{\partial x_{i}} = 0,\) &
\protect\hypertarget{_Ref41041729}{}{}(1)\tabularnewline
\midrule
\endhead
\bottomrule
\end{longtable}

\begin{longtable}[]{@{}lll@{}}
\toprule
&
\(\frac{\partial U_{i}}{\partial t} + U_{j}\frac{\partial U_{i}}{\partial x_{j}} = - \frac{1}{\rho}\frac{\partial P}{\partial x_{i}} + \frac{\partial}{\partial x_{j}}\left( 2\nu S_{\text{ij}} - \overset{\overline{}}{u_{j}^{'}u_{i}^{'}} \right).\)
& \protect\hypertarget{_Ref41041747}{}{}(2)\tabularnewline
\midrule
\endhead
\bottomrule
\end{longtable}

The symbols \(U\), \(x\), \(t\), \(\rho\), \(P\), \(\nu\), \(S\), and
\(u^{'}\) are the mean component of velocity, spatial component, time,
density, pressure, kinematic viscosity, mean strain rate and the
fluctuating component of velocity, respectively. The mean strain rate
\(S_{\text{ij}}\) is given by

\begin{longtable}[]{@{}lll@{}}
\toprule
&
\(S_{\text{ij}} = \frac{1}{2}\left( \frac{\partial U_{i}}{\partial x_{j}} + \frac{\partial U_{j}}{\partial x_{i}} \right).\)
& (3)\tabularnewline
\midrule
\endhead
\bottomrule
\end{longtable}

This study employs the Spalart-Allmaras turbulence model to approximate
the Reynolds stress tensor
\(\tau_{\text{ij}} = \overset{\overline{}}{u_{j}^{'}u_{i}^{'}}\). This
turbulence model has been shown to produce results that agree reasonably
well with experiments in similar flow-induced motion (FIM) studies
(Ding, Zhang, Wu, \emph{et al.}, 2015; Ding, Zhang, Kim, \emph{et al.},
2015). We use the Boussinesq approximation to relate the Reynolds stress
tensor to the mean velocity gradient

\begin{longtable}[]{@{}lll@{}}
\toprule
& \(\tau_{\text{ij}} = 2\nu_{T}S_{\text{ij}},\) & (4)\tabularnewline
\midrule
\endhead
\bottomrule
\end{longtable}

where \(\nu_{T}\) represents the kinetic eddy viscosity. \(\nu_{T}\) is,
in turn, a function of \(\tilde{\nu}\) and \(f_{\nu 1}\), while
\(f_{\nu 1}\) is a function of \(\chi\) and \(c_{\nu 1}\), and \(\chi\)
a function of \(\tilde{\nu}\) and \(\nu\), as shown in Eq.
\textbf{Error! Reference source not found.}).

\begin{longtable}[]{@{}lllll@{}}
\toprule
& \(\nu_{T} = \tilde{\nu}f_{\nu 1},\) &
\(f_{\nu 1} = \frac{\chi^{3}}{\chi^{3} + c_{\nu 1}^{3}},\) &
\(\chi = \frac{\tilde{\nu}}{\nu}.\) & (5)\tabularnewline
\midrule
\endhead
\bottomrule
\end{longtable}

Here, \(\tilde{\nu}\) serves to mediate the turbulence model and
Eq.\textbf{Error! Reference source not found.}) dictates how
\(\tilde{\nu}\) is conserved.

\begin{longtable}[]{@{}lll@{}}
\toprule
&
\(\frac{\partial\tilde{\nu}}{\partial t} + U_{j}\frac{\partial\tilde{\nu}}{\partial x_{j}} = c_{b1}\tilde{S}\tilde{\nu} - c_{w1}f_{w}\left( \frac{\tilde{\nu}}{D} \right)^{2} + \frac{1}{\sigma}\left\{ \frac{\partial}{\partial x_{j}}\left\lbrack \left( \nu + \tilde{\nu} \right)\frac{\partial\tilde{\nu}}{\partial x_{j}} \right\rbrack + c_{b2}\frac{\partial\tilde{\nu}}{\partial x_{i}}\frac{\partial\tilde{\nu}}{\partial x_{i}} \right\}\)
& (6)\tabularnewline
\midrule
\endhead
\bottomrule
\end{longtable}

\(c_{b1}\), \(c_{b2}\), and \(c_{\nu 1}\) are constants with values
0.1335, 0.622 and 7.1 respectively. \(c_{w1}\) is given by

\begin{longtable}[]{@{}lll@{}}
\toprule
& \(c_{w1} = \frac{c_{b1}}{\kappa} + \frac{1 + c_{b2}}{\sigma},\) &
(7)\tabularnewline
\midrule
\endhead
\bottomrule
\end{longtable}

where additional constants \(\kappa\) and \(\sigma\) are 0.41 and
\(2/3\) respectively. \(f_{w}\), on the other hand, is given by

\begin{longtable}[]{@{}lll@{}}
\toprule
&
\(f_{w} = g\left( \frac{1 + c_{w3}^{6}}{g^{6} + c_{w3}} \right)^{\frac{1}{6}}.\)
& (8)\tabularnewline
\midrule
\endhead
\bottomrule
\end{longtable}

Here, \(c_{w3} = 2\) while \(g\) is given by

\begin{longtable}[]{@{}lll@{}}
\toprule
& \(g = r + c_{w2}\left( r^{6} - r \right),\) & (9)\tabularnewline
\midrule
\endhead
\bottomrule
\end{longtable}

where \(r\) is

\begin{longtable}[]{@{}lll@{}}
\toprule
&
\(r = \min\left( \frac{\tilde{\nu}}{\tilde{S}\kappa^{2}d^{2}},10 \right).g = r + c_{w2}\left( r^{6} - r \right),\)
& (10)\tabularnewline
\midrule
\endhead
\bottomrule
\end{longtable}

Additionally, \(\tilde{S}\) is

\begin{longtable}[]{@{}lll@{}}
\toprule
& \(\tilde{S} = \Omega + \frac{\tilde{\nu}}{\kappa^{2}d^{2}}f_{\nu 2},\)
& (11)\tabularnewline
\midrule
\endhead
\bottomrule
\end{longtable}

where \(\Omega\) and \(d\) are the magnitude of vorticity and the
distance from the mesh nodes to the nearest wall, respectively. Finally,
\(f_{\nu 2}\) is

\begin{longtable}[]{@{}lll@{}}
\toprule
& \(f_{\nu 2} = 1 - \frac{\chi}{1 + \chi f_{\nu 1}}.\) &
(12)\tabularnewline
\midrule
\endhead
\bottomrule
\end{longtable}

We solve these equations numerically using the PIMPLE algorithm, which
combines the transient solver PISO with the steady-state solver SIMPLE
for improved numerical stability.

\hypertarget{dynamic-mesh-motion}{\subsection{Dynamic Mesh
Motion}\label{dynamic-mesh-motion}}

In this study, the cylinder in VIV moves perpendicular to the free
stream direction. The motion unavoidably distorts the mesh around it,
degrading important mesh metrics such as non-orthogonality and skewness.
However, we can diffuse the mesh deformation to the neighbouring nodes
as per the following Laplace equation,

\begin{longtable}[]{@{}lll@{}}
\toprule
& \(\nabla \cdot \left( \gamma\nabla u \right) = 0.\) &
(13)\tabularnewline
\midrule
\endhead
\bottomrule
\end{longtable}

Here, \(u\) represents the mesh deformation velocity and \(\gamma\) is
displacement diffusion. We chose \(\gamma = 1/l^{2}\), where \(l\) is
the cell centre distance to the nearest cylinder edges. We implement the
GAMG linear solver with the Gauss-Seidel smoother to solve Eq.(13). The
dynamic mesh algorithm then updates the mesh node positions according to
the following equation.

\begin{longtable}[]{@{}lll@{}}
\toprule
& \(x_{\text{new}} = x_{\text{old}} + ut\) & (14)\tabularnewline
\midrule
\endhead
\bottomrule
\end{longtable}

The solver resumes the solution of Eqs. (1) and (2) once the mesh node
positions are updated.

Another dynamic mesh handling technique used in this study is the
arbitrarily coupled mesh interface (ACMI), that allows non-conforming
meshes to slide over another, thus preserving the mesh quality around a
moving object. The very small gap between the cylinder and strip-plate,
limits our ability to diffuse the mesh deformation to the surrounding
space. ACMI is thus implemented at the centre of the gap between the
circular cylinder and the strip-plate, as shown in Figure 4, to
circumvent this problem. This method has been successfully implemented
in the works of (Ding, Zhang, Kim, \emph{et al.}, 2015; Xu \emph{et
al.}, 2017) and (B. Zhang \emph{et al.}, 2018), preserving the quality
of their mesh and controlling their Courant-Fredichs-Lewy (CFL) number.

\subsection{\texorpdfstring{\protect\hypertarget{_Ref40998709}{}{\protect\hypertarget{_Toc41048807}{}{}}Experimental
Validation of Benchmark
Case}{Experimental Validation of Benchmark Case}}\label{experimental-validation-of-benchmark-case}

One of the aims of this work is to study the transition between Karman
and streamwise vortex-induced vibration. Therefore, we set up an
experimental rig to study the VIV motion in the vicinity of \(U^{*}\)
where the vortex-induced vibration transitions from Karman to streamwise
vortex driven. To this end, we constructed a closed-loop open channel
circuit based on the water tunnel used by (Nguyen \emph{et al.}, 2012),
shown in Fig. 3. The cross-section of our test section is a rectangle of
width 100 mm and height 200 mm. The water level is kept at a height of
100 mm from the bottom of the test section throughout its length of 1500
mm.

The system for providing elastic support and damping to the circular
cylinder follows closely those used by (Kawabata \emph{et al.}, 2013;
Koide \emph{et al.}, 2013, 2017), which can be summarised as follows.
The stiffness coefficient \(k\) of the plate spring is determined
through a simple weight versus displacement test (Sun \emph{et al.},
2016), at various active lengths of the spring. This provides a
calibration curve of stiffness coefficient, \(k\) against plate spring
length, \(l\). We can then adjust the length of the plate spring to
obtain the desired value for \(k\).

On the other hand, the damping of the system is adjusted using T-shaped
aluminium plates fixed at either ends of the cylinder end plate, and a
pair of neodymium magnets contained in a claw-shaped casing. The further
the T-shaped plate is pushed into the opening of the claw, the denser
the magnetic field it needs to cut through during motion, thus
dissipating more energy. We then calibrate the damping produced at
various depths at which the T-shaped plate is pushed into the casing,
via free-decay tests of the cylinder in still water. The procedure for
conducting free-decay tests are detailed in (Raghavan, 2007).

\includegraphics[width=5.40208in,height=5.18611in]{media/image9.png}

\protect\hypertarget{_Toc41048828}{}{}Figure 5: Schematic of the
experimental setup. (a) presents a three dimensional schematic of the
experimental rig while (b) shows an enlarged schematic of the damping
system.

Flow inside the open channel is driven by a 3.728 kW (5 hp) centrifugal
pump, controlled using a voltage controller. The input voltage for the
centrifugal pump is calibrated against the centreline velocity of the
test section, 750 mm from the inlet, i.e. mid-length of the test
section. We show this schematically in Figure 6. Here, we define the
centreline of the test section as the line 50 mm from the bottom and 50
mm from either of the side walls of the test section. We placed the
cylinder at the same position during experimental runs.

\includegraphics[width=5.40208in,height=2.24719in]{media/image13.png}

\protect\hypertarget{_Ref40996269}{}{\protect\hypertarget{_Toc41048829}{}{}}Figure
6: Schematic and key dimensions from our experimental setup. The
acoustic Doppler velocimeter (ADV) was placed at the same location where
the cylinder is located during experimental runs.

The centreline velocity, \(U_{\text{cent.}}\) is measured using an
acoustic Doppler velocimeter (ADV), sampling at 200 Hz. The resulting
calibration curve is applicable for determining \(U_{\text{cent.}}\) at
input voltages \({30 < V}_{\text{in}}\ (V) < 100\). We measured the
turbulence intensity along the centreline to be about 5\%.

We obtained the time history for cylinder displacement by using a video
camera pointed normal to the cylinder end plate. We placed a visual
marker on the end plate, and the motion of the marker captured by the
camera is analysed using Tracker: a motion analysis tool built on the
Open Source Physics Java framework.

To validate our experimental setup, we tuned to the best of our ability
our experimental parameters to the values used by (Koide \emph{et al.},
2013) and test whether we can replicate their results. The following
table summarises the parameters in lieu of (Koide \emph{et al.}, 2013).

Table 2: Summary of experimental parameters in contrast to those used in
the experimental work of (Koide \emph{et al.}, 2013).

\begin{longtable}[]{@{}lll@{}}
\toprule
& Current study & (Koide \emph{et al.}, 2013)\tabularnewline
\midrule
\endhead
Cylinder diameter, \(D\) (m) & 0.01 & 0.01\tabularnewline
Cylinder length, \(l_{\text{cylinder}}\) (m) & 0.09 &
0.098\tabularnewline
Strip-plate width (m) & 0.01 & 0.01\tabularnewline
Strip-plate length (m) & 0.1 & 0.1\tabularnewline
Effective mass, \(m_{e}\) (kg) & 0.162 & 0.174\tabularnewline
Logarithmic damping, \(\delta\) & 0.178 & 0.24\tabularnewline
Scruton number, Sc & 9.94 & 7.74\tabularnewline
System natural frequency, \(f_{n}\) (Hz) & 4.42 & 4.4 \textasciitilde{}
4.79\tabularnewline
\bottomrule
\end{longtable}

We conducted an experiment at \(U^{*} = 22.7\), a value that falls in
the SVIV regime, and show a sample of the normalised displacement time
series in Figure 7. Computing the statistics of the normalised
displacement \(y^{*} = y/D\) and normalised vibration frequency
\(f^{*} = f_{\text{cyl.}}/f_{n}\ \)of the cylinder from several runs
gave us a value of \(y^{*} = 0.33 \pm 0.03\) and
\(f^{*} = 1.03 \pm 0.04\). (Koide \emph{et al.}, 2013) obtained
\(y^{*} = 0.32\) and \(f^{*} = 1.09\) under a similar \(U^{*}\)
condition. We thus take this fairly successful reproduction of the
results of (Koide \emph{et al.}, 2013) can be taken as an indication of
readiness for further data collection.

\includegraphics[width=5.40208in,height=2.67944in]{media/image15.png}

\protect\hypertarget{_Ref40996667}{}{\protect\hypertarget{_Toc41048830}{}{}}Figure
7: A sample of the time history for cylinder displacement from a test
run of our experimental setup. The value of \(U^{*} = 22.7\).

\hypertarget{grid-independency-study}{\subsection{Grid Independency
Study}\label{grid-independency-study}}

In this study, checking for grid independence involves deciding whether
or not the quantities of interest tend towards a value, as one solves
the governing equation on successively finer grid resolutions
(Richardson and Gaunt, 1927; Stern \emph{et al.}, 2001). This method, of
checking for convergence pays attention not only on the presumed
converged value, but also on the trend of convergence. Literatures that
employ this method impose a monotonic convergence condition (Stern
\emph{et al.}, 2001; Mat Ali, Doolan and Wheatley, 2011; Ali, Doolan and
Wheatley, 2012; Maruai \emph{et al.}, 2018) on their quantities of
interest, adding an extra layer of confidence in the final form of their
spatial discretisation.

Additionally, this method allows for a quantitative description of the
degree of convergence through the grid convergence index (GCI). Let
\(f_{1},f_{2},f_{3},\cdots,f_{n}\) denote the quantity of interest
obtained from several grids. A larger subscript indicates a coarser
grid, thus \(f_{1}\) denotes the finest while \(f_{n}\) denotes the
coarsest grid. Let the difference between successive solutions be
\(\epsilon_{2,1},\epsilon_{3,2},\epsilon_{4,3},\cdots,\epsilon_{n,n - 1}\),
where \(\epsilon_{2,1} = f_{2} - f_{1}\),
\(\epsilon_{3,2} = f_{3} - f_{2}\) and so on. Then, the GCI is defined
as

\begin{longtable}[]{@{}lll@{}}
\toprule
&
\(\text{GC}I_{i + 1,i} = F_{s}\frac{\left| \epsilon_{i + 1,i} \right|}{f_{i}\left( r^{p} - 1 \right)} \times 100\%,\)
& (15)\tabularnewline
\midrule
\endhead
\bottomrule
\end{longtable}

where \(F_{s}\), \(f_{i}\) and \(r^{p}\) denotes the safety factor
(\(= 1.25\)), quantity of interest and the refinement ratio (\(r\))
between successive grids raised to the order of accuracy of the series
of solutions (\(p\)). We refer the reader to ­­­­­­(Stern \emph{et al.},
2001; Langley Research Centre, 2018) for a more detailed discussion on
\(r^{p}\).

We can estimate what the solution approaches as the grid size approaches
zero by using the p\textsuperscript{th} method. Briefly, we compute the
generalised Richardson extrapolation of the quantity of interest as
follows.

\begin{longtable}[]{@{}lll@{}}
\toprule
& \(f_{\text{RE}} = f_{1} + \frac{f_{1} - f_{2}}{r^{p} - 1},\) &
(16)\tabularnewline
\midrule
\endhead
\bottomrule
\end{longtable}

where \(f_{\text{RE}}\) is the Richardson extrapolation of the quantity
of interest. Using \(f_{\text{RE}}\) to estimate the limit of the
monotonically convergent series of \(f_{i}\), we can estimate the
percentage difference of our solution on our finest grid from this limit
as

\begin{longtable}[]{@{}lll@{}}
\toprule
& \(E_{i} = \frac{f_{i} - f_{\text{RE}}}{f_{\text{RE}}} \times 100\%.\)
& (17)\tabularnewline
\midrule
\endhead
\bottomrule
\end{longtable}

Table 3 summarises the result of our grid independency study for the
SVIV reduced velocity of \(U^{*} = 22.7\). We identified three
quantities central to the investigation of fluid-structure phenomena,
especially flow-induced vibration of a circular cylinder. They are the
vibration amplitude, vibration frequency and lift coefficient of the
cylinder. We solve the governing equations on three grids which are
numbered 1 for the finest, 2 for the medium and 3 for the coarsest,
shown in Figure 8. If we let \(v_{i}\) be the volume of the
\emph{i}\textsuperscript{th} cell in the grid, then, the average cell
size is

\begin{longtable}[]{@{}lll@{}}
\toprule
&
\(h = \frac{1}{N}\left\lbrack \sum_{i = 1}^{N}v_{i} \right\rbrack^{1/3},\)
& (18)\tabularnewline
\midrule
\endhead
\bottomrule
\end{longtable}

and the normalised average cell size is hence
\(h/D = \frac{1}{\text{ND}}\left\lbrack \sum_{i = 1}^{N}v_{i} \right\rbrack^{1/3}\).

\includegraphics[width=5.67532in,height=4.22431in]{media/image16.jpeg}

\protect\hypertarget{_Ref40997559}{}{\protect\hypertarget{_Toc41048831}{}{}}Figure
8: Three meshes used in the grid convergence study. (a), (b) and (c)
show the coarse, medium and fine meshes viewed perpendicular to three
main viewing positions: from the inlet, the top and the front, which is
looking directly at the cylinder end.

Both \(y_{\text{RMS}}^{*}\) and \(Cl_{\text{RMS}}\) starts at an initial
value smaller than their Richardson extrapolations, \(f_{\text{RE}}\),
before approaching \(f_{\text{RE}}\) as we decrease the average cell
size, \(h\). This similar trend can perhaps be attributed to the causal
relationship between the lift coefficient and vibration amplitude. The
lift drives and sustains the vibration, hence a small lift produces a
small vibration, and when the lift amplitude becomes higher, so too does
the vibration amplitude. The vibration frequency, on the other hand,
starts at a value larger than its \(f_{\text{RE}}\) before falling as it
approaches its \(f_{\text{RE}}\).

\protect\hypertarget{_Ref40997535}{}{}Table 3: Summary of grid
independency study.

\begin{longtable}[]{@{}llll@{}}
\toprule
Parameter/

metric & \(Cl_{\text{RMS}}\) & \(y_{\text{RMS}}^{*} = y_{\text{RMS}}/D\)
& \(f^{*} = f_{\text{cyl.}}/f_{n}\)\tabularnewline
\midrule
\endhead
\(f_{\text{RE}}\) & 0.262 & 0.369 & 0.969\tabularnewline
\(f_{1}\) & 0.2598 & 0.3687 & 0.9695\tabularnewline
\(f_{2}\) & 0.2430 & 0.3588 & 0.9740\tabularnewline
\(f_{3}\) & 0.0805 & 0.2374 & 1.0220\tabularnewline
\(\left| \epsilon_{2,1} \right|\) & 0.02 & 0.01 & 0.004\tabularnewline
\(\left| \epsilon_{3,2} \right|\) & 0.16 & 0.12 & 0.48\tabularnewline
\(R = \left| \epsilon_{2,1} \right|/\left| \epsilon_{3,2} \right|\) &
0.10 & 0.08 & 0.094\tabularnewline
\(\text{GC}I_{3,2}\ (\%)\) & 30.92 & 6.00 & 0.64\tabularnewline
\(\text{GC}I_{2,1}\ (\%)\) & 1.63 & 0.52 & 0.10\tabularnewline
\bottomrule
\end{longtable}

The quantity C\(l_{\text{RMS}}\) experiences the most significant drop
in GCI as we refine the grid. The GCI is close to one-third
(\(30.92\%\)) as we refine the grid from coarse to medium with a
refinement ratio of 1.376. The refinement ratio is calculated by
dividing the number of cells in one grid with the next one down the
refinement line. Following the grid numbering convention explained
previously, dividing the number of cells in the fine grid (grid 1) with
the number of cells in the medium grid (grid 2) gives us the refinement
ratio from medium to fine, of \(r_{2,1}\). Similarly, dividing the
number of cells in the medium grid (grid 2) with the number of cells in
the coarse grid (grid 3) gives us the refinement ratio from coarse to
medium, or \(r_{3,2}\). We can generalise this to \(n\)--number of grids
as follows.

\begin{longtable}[]{@{}lll@{}}
\toprule
& \(r_{i + 1,i} = \frac{S_{\text{gri}d_{i + 1}}}{S_{\text{gri}d_{i}}},\)
& (19)\tabularnewline
\midrule
\endhead
\bottomrule
\end{longtable}

where \(S_{\text{gri}d_{i}}\) denotes the total number of cells in the
\(i^{\text{th}}\) grid. The GCI of \(\backslash nCl_{\text{RMS}}\) drops
further to \(1.63\%\) as the mesh is refined more with a refinement
ratio of 1.304.

The GCI for \(y_{\text{RMS}}^{*}\) also drops by one order of magnitude
as can be seen by comparing \(\text{GC}I_{3,2}\) with
\(\text{GC}I_{2,1}\). Again, this similar trend of improvement points to
the causal relationship between lift and displacement of the cylinder.
The GCI for \(f^{*}\), however, drops by approximately a factor of 6
instead of one order or magnitude, unlike the GCIs of
\(y_{\text{RMS}}^{*}\) and \(Cl_{\text{RMS}}\).

I provide visual representations of the convergent \(Cl_{\text{RMS}}\),
\(y_{\text{RMS}}^{*}\) and \(f^{*}\) series in Figs. 9, 10 and 11. Note
how the quantity of interest is very close to its Richardson
extrapolation at the fine grid (grid 1) for all \(Cl_{\text{RMS}}\),
\(y_{\text{RMS}}^{*}\) and \(f^{*}\). This implies that the fine grid
already provides adequate spatial discretisation for the problem we are
studying, and further refinements, while able to nudge our solutions
even closer to the limit that is the Richardson extrapolation, may not
be optimal in terms of usage of computational resources. Values of
\(y_{\text{RMS}}^{*}\) and \(f^{*}\) at the fine grid already fall
within experimental uncertainty as evidenced by our measurement in §3.4
and the work by (Koide \emph{et al.}, 2013). Hence, all succeeding
numerical data are gathered from grids with a resolution like the fine
grid.

\includegraphics[width=5.40208in,height=4.16667in]{media/image25.png}

\protect\hypertarget{_Toc41048832}{}{}Figure 9: The convergence diagram
for \(y_{\text{RMS}}^{*}\). (a) shows how \(y_{\text{RMS}}^{*}\)
converges close to the Richardson extrapolation value while (b) shows
how the error (difference between the value obtained from a particular
mesh and the Richardson extrapolation) decreases with decreasing grid
spacing.

\includegraphics[width=5.65915in,height=4.45833in]{media/image27.png}

\protect\hypertarget{_Toc41048833}{}{}Figure 10: The convergence diagram
for \(f^{*}\). (a) shows how \(f^{*}\) converges close to the Richardson
extrapolation value while (b) shows how the error (difference between
the value obtained from a particular mesh and the Richardson
extrapolation) decreases with decreasing grid spacing.

\includegraphics[width=5.66875in,height=4.93506in]{media/image29.png}

\protect\hypertarget{_Toc41048834}{}{}Figure 11: The convergence diagram
for \(\text{Cl}_{\text{RMS}}\). (a) shows how \(\text{Cl}_{\text{RMS}}\)
converges close to the Richardson extrapolation value while (b) shows
how the error (difference between the value obtained from a particular
mesh and the Richardson extrapolation) decreases with decreasing grid
spacing.

\section{\texorpdfstring{\protect\hypertarget{_Toc461037268}{}{\protect\hypertarget{_Toc461037451}{}{\protect\hypertarget{_Toc41048809}{}{}}}}{}}\label{section-3}

\textbf{THE PURE CRUCIFORM}

\subsection{\texorpdfstring{\protect\hypertarget{_Toc41048810}{}{\protect\hypertarget{_Ref41049198}{}{}}Amplitude
Response}{Amplitude Response}}\label{amplitude-response}

Figure 12 shows the amplitude response of our single plate experiment
and simulation. Besides simplifying comparison between our work and the
results of (Koide \emph{et al.}, 2013) as we did previously, this time
we also compare our data with the results of (Nguyen \emph{et al.},
2012). We use the root-mean-square value of the cylinder displacement to
represent the amplitude responses instead of the maximum displacement.
The reason for this is twofold: first, using \(y_{\text{RMS}}^{*}\)
facilitates comparison of data with (Nguyen \emph{et al.}, 2012; Koide
\emph{et al.}, 2013), who also used \(y_{\text{RMS}}^{*}\) in their
work. Second, because the cylinder displacement is an intermediate
quantity for the estimation harnessed power (Maruai \emph{et al.}, 2017,
2018), the usage of root-mean-square of cylinder displacement gives a
preview of mean harnessed power, once the vibration is converted into
alternating current. There is virtually no vibration for both our
experiment and simulation when \(U^{*} < 18\), except for a small peak
close to 0.1 at \(U^{*} \approx 7\). We attribute this peak to the upper
branch of KVIV, which still exists, although suppressed due to the
cruciform configuration of the system (Shirakashi, Mizuguchi and Bae,
1989; Nguyen \emph{et al.}, 2012). However, when \(U^{*}\) exceeds 18,
we observe a sudden jump in \(y_{\text{RMS}}^{*}\) right up to about
0.4, for both our experiment and simulation. This we attribute to the
formation of the streamwise vortices that drive SVIV.

\includegraphics[width=5.40208in,height=5.76944in]{media/image31.png}

\protect\hypertarget{_Ref41040609}{}{\protect\hypertarget{_Toc41048835}{}{}}Figure
12: The amplitude and frequency response of our cruciform system, in
lieu of results from (Nguyen \emph{et al.}, 2012; Koide \emph{et al.},
2013). (a) shows the amplitude response using \(y_{\text{RMS}}^{*}\),
(b) the frequency response using \(f^{*}\) and (c) also the frequency
response, but using the Strouhal number for vibration St.

After the inception of SVIV, value for \(y_{\text{RMS}}^{*}\) drops down
to approximately 0.3, before recovering to a value that is close to what
was observed by (Nguyen \emph{et al.}, 2012) and (Koide \emph{et al.},
2013). This sudden jump followed by a gradual drop and a gradual rise in
\(y_{\text{RMS}}^{*}\) was not observed in the works of (Nguyen \emph{et
al.}, 2012) nor (Koide \emph{et al.}, 2013), even though their
experimental parameters are fairly close to what we use in both our
experiment and simulation.

We therefore attribute this difference to the higher turbulence level
set in our work. The turbulence level in the works of (Nguyen \emph{et
al.}, 2012), for example was \(< 2.8\%\) throughout their range of
Reynolds number. Instead, the initial turbulence level in our setup,
both experimental and numerical is approximately double that value.
Because of this, the turbulence amplification due to the onset of
streamwise vortices (Zhao and Lu, 2018) -- especially for a circular
cylinder-strip plate cruciform (Koide \emph{et al.}, 2017) -- is also
greater compared to the experiments of (Nguyen \emph{et al.}, 2012) and
(Koide \emph{et al.}, 2013). This higher compound turbulence warps the
dominant vortical structure and introduces an increasing amount of
intermittency to the lift signal, and by extension, to the displacement
time history of the cylinder.

One can simply inspect the error bars within \(18 < U^{*} < 23\) in
Figure 12(a) to verify the greater sample dispersion within that range
of \(U^{*}\). This intermittency ultimately vanishes as the dominant
vortical structures become sufficiently stable to retain enough
periodicity in its formation. Our numerical results also seem to support
this argument, as evidenced by the time history of \(y^{*}\) within
\({18 < U}^{*} < 30\) in Figure 13. There exists a distinct increase in
intermittency for the time histories in Figure 13(a), that disappears
once \(U^{*} > 23\) as can be seen in Figure 13(b).

\includegraphics[width=5.40208in,height=4.65169in]{media/image35.png}

\protect\hypertarget{_Ref41044658}{}{\protect\hypertarget{_Toc41048836}{}{}}Figure
13: The time series of cylinder displacement between
\(18 < U^{*} < 30\). (a) groups the cylinder displacement signal between
\(18 < U^{*} < 23\), where there seems to be an increase in
intermittency in the displacement signal, while (b) groups the cylinder
displacement signal between \(25 \leq U^{*} < 30\), where the
intermittency in the displacement signal vanishes.

We see these as grounds for further study on streamwise vortex shedding
onset, perhaps from the perspective of transition from convective to
absolute instability. However, such studies are more commonly done under
low Reynolds number conditions (Wang \emph{et al.}, 2019; X. Li \emph{et
al.}, 2019) to ease the isolation of the phenomenon and is therefore out
of the scope of this study.

\hypertarget{frequency-response}{\subsection{Frequency
Response}\label{frequency-response}}

Figure 12(b) compares the frequency responses of our experiment and
numerical results with those in (Nguyen \emph{et al.}, 2012) and (Koide
\emph{et al.}, 2013). We use the normalised frequency \(f^{*}\) in
Figure 12(b) and the vibration Strouhal number in Figure 12(c) to aid
comparison between the results. In our experiments, the value for
\(f^{*}\) always fall close to unity. However, if the inspect the size
of the error bars we observed a range of \(U^{*}\) where there exists a
higher degree of variance in the sample measurements between
\(13 < U^{*} < 20\). The reason for this lies in \(13 < U^{*} < 20\)
coinciding with the desynchronization region of the KVIV regime up to
\(U^{*} < 18\), and then overlaps with the intermittent vibration regime
up to \(U^{*} < 20\). Within these two regimes, the cylinder
displacement history -- from which \(f^{*}\) is calculated -- varies
considerably in amplitude and in periodicity, resulting in larger error
bars. In Figure 12(c) we can see the overall trend being more similar to
the results of (Koide \emph{et al.}, 2013) rather than (Nguyen \emph{et
al.}, 2012), which is likely due to a higher similarity between our
experimental setup with that of (Koide \emph{et al.}, 2013), most
striking in terms of the gap ratio \(G\), which is exactly the same.

Our numerical results exhibit a significantly different trend, but only
up to \(U^{*} < 17\). We observe in Figure 12(b) that the vibration
frequency of the cylinder increases linearly, even past \(U^{*} = 7\),
which is the upper branch of the KVIV regime. Converting \(f^{*}\) into
Strouhal number reveals that the cylinder is vibrating close to the
Karman frequency of the system. The Karman frequency of a smooth, fixed
circular cylinder refers to the shedding frequency of Karman vortices in
its wake. An empirical relationship with Reynolds number exists for
\(250 < Re < 2 \times 10^{5}\), which is the following Eq. (20)
(Blevins, 1990).

\begin{longtable}[]{@{}lll@{}}
\toprule
& \(\text{St} = 0.198\left( 1 - \frac{19.7}{\text{Re}} \right)\) &
\protect\hypertarget{_Ref41045031}{}{}(20)\tabularnewline
\midrule
\endhead
\bottomrule
\end{longtable}

The values we get using Eq. (20) are nearly constant about \(0.19\) for
\(U^{*} < 15\). The slight discrepancy from our Strouhal number mean
(\(\approx 0.16\)) in the \(U^{*} < 15\) range can be ascribed to us
studying a cruciform structure instead of the smooth circular cylinder
upon which Eq. (20) was originally based (Blevins, 1990).

The experimental studies benchmarked in Figure 12 quite possibly were
simply unable to observe this phenomenon due to insufficient sensitivity
in the equipment used to measure the cylinder displacement. The lowest
\(y_{\text{RMS}}^{*}\) recorded in our simulation within
\(7 < U^{*} < 15\) was in the order of \(10^{- 5}\) m (10 microns). A
numerical study has no problem recording vibration of this order as the
precision of the numerical solution is only limited by the processor
architecture. Experimental work, however, requires not only the
sensitivity, but also the isolation from the background noise that
forces the cylinder to vibrate close to the natural frequency of the
system \(f_{n}\) (Nguyen \emph{et al.}, 2012), hence overpowering this
very small amplitude vibration. Once streamwise vortices form, their
shedding and cylinder vibration synchronises close to \(f_{n}\), thus
locking the normalised vibration frequency back to \(f^{*} \approx 1\).

\hypertarget{ensemble-empirical-mode-decomposition-eemd-and-hilbert-transform}{\subsection{Ensemble
Empirical Mode Decomposition (EEMD) and Hilbert
Transform}\label{ensemble-empirical-mode-decomposition-eemd-and-hilbert-transform}}

To obtain a clearer picture of the temporal characteristics of the lift
and cylinder displacement signals, we decided to employ the ensemble
empirical mode decomposition (EEMD) method (Huang \emph{et al.}, 1998;
WU and HUANG, 2008) on the signals, and compute their instantaneous
phase lag, frequency and amplitude. EEMD is a method of signal
decomposition that breaks down a signal into a finite number of
constituents \(C_{i}\) (\(i = 1,\ 2,\ \cdots,\ n\)), each component
\(C_{i}\) orthogonal to \(C_{i + 1}\). To compute the phase lag between
lift coefficient Cl and normalised cylinder displacement \(y^{*}\), we
select the component with the highest correlation to the original
signal, to represent the original signal.

The Hilbert transform (HT) has been used in the past to study the
instantaneous phase and frequencies of KVIV (Khalak and Williamson,
1999). However, the signal must be monochromatic if we are to obtain a
physically meaningful result after applying HT. EEMD is a way to
pre-process the signal and obtain components that (1) have zero mean,
and (2) have an equal number of extrema and zero crossings, or they
differ only by one. Functions that fulfil these criteria are called
intrinsic mode functions (IMF), and they guarantee a physically
meaningful result to HT (Gumelar \emph{et al.}, 2019; Zhou, Meng and
Abbaspour, 2019). We refer the reader interested in the details of EEMD
and Hilbert transform, also collectively known as the Hilbert-Huang
transform (HHT), to the following excellent texts on the subject (Huang
and Attoh-Okine, 2005; Huang, 2014).

\hypertarget{phase-lag-in-the-kviv-regime-mathbfumathbfmathbf-14}{\subsection{\texorpdfstring{Phase
lag in the KVIV regime
(\(\mathbf{U}^{\mathbf{*}}\mathbf{< 14}\))}{Phase lag in the KVIV regime (\textbackslash{}mathbf\{U\}\^{}\{\textbackslash{}mathbf\{*\}\}\textbackslash{}mathbf\{\textless{} 14\})}}\label{phase-lag-in-the-kviv-regime-mathbfumathbfmathbf-14}}

At reduced velocities \(U^{*} = 2.3\) and 4.5, the phase lags \(\phi\)
(\(\)) between Cl and \(y^{*}\) are practically zero. The characteristic
IMFs of Cl and \(y^{*}\) at \(U^{*} = 4.5\) exemplifies this trend, as
showcased in Fig. 12. Recall that the characteristic IMFs are the EEMD
components of Cl and \(y^{*}\) that has the highest correlation with the
original signals. The trend that one notices in Fig. 12 is similar to
what was observed in (Khalak and Williamson, 1999), a study that also
employs the Hilbert transform to obtain instantaneous phase, albeit
without EEMD. Both Cl and \(y^{*}\) are in phase with each other and the
normalised dominant frequency of the lift coefficient
\(f_{\text{Cl}}^{*} = f_{\text{Cl}}/f_{n}\) (Fig. 12(c)) falls about one
quarter short of the system natural frequency \(f_{n}\).

\includegraphics[width=5.40208in,height=5.59845in]{media/image47.png}

\protect\hypertarget{_Toc41048837}{}{}Figure 14: Temporal analysis of
the lift coefficient and normalised cylinder displacement signal at
\(U^{*} = 4.5\). We show the lift coefficient and normalised cylinder
displacement signal side by side in (a), present the temporal evolution
of the phase lag \(\phi\) of Cl in (b) and show the temporal evolution
of the instantaneous frequency of the lift coefficient signal in (c).
The blue line in (a) represents the lift coefficient signal, while the
black line represents the normalised cylinder displacement.

Once we enter the upper branch of KVIV at \(U^{*} = 6.8\), \(\phi\)
jumps to approximately \(110\). This jump in \(\phi\) is characteristic
of the transition to the upper branches as observed also by (Maruai
\emph{et al.}, 2018), among others. Both Cl and \(y^{*}\) signals are
visibly very periodic, and the dominant frequency of Cl
\(f_{\text{Cl}}^{*}\) is \(\approx 1\), as one can verify in Fig. 13(c).

\includegraphics[width=5.40208in,height=5.55833in]{media/image53.png}

\protect\hypertarget{_Toc41048838}{}{}Figure 15: Temporal analysis of
the lift coefficient and normalised cylinder displacement signal at
\(U^{*} = 6.8\). We show the lift coefficient and normalised cylinder
displacement signal side by side in Fig. (a), present the temporal
evolution of the phase lag \(\phi\) of Cl in (b) and show the temporal
evolution of the instantaneous frequency of the lift coefficient signal
in (c). The blue line in (a) represents the lift coefficient signal,
while the black line represents the normalised cylinder displacement.

As we increase \(U^{*}\) even further up to \(U^{*} < 14\), we see a
similar trend for all \(U^{*} = 9.1,\ 11.4,\ 13.6\) examined: the signal
of Cl and \(y^{*}\) are both qualitatively very periodic, the phase lag
is very close to \(180\), and the dominant Cl frequency increases
linearly in a manner that the Strouhal number of Cl is always
\(\approx 0.16\) on average. We present a sample of the (1) Cl and
\(y^{*}\) signals, (2) \(\phi\), and (3)
\protect\hypertarget{_Hlk23621866}{}{}\(f_{\text{Cl}}^{*}\) in the
\(6.8 < U^{*} < 14\) range in Figure 16(a), (b) and (c) respectively.
The sample is taken from the numerical results at \(\ U^{*} = 13.6\),
and it is characteristic of a KVIV system in the lower branch.

\includegraphics[width=5.40208in,height=5.78333in]{media/image56.png}

\protect\hypertarget{_Ref41047456}{}{\protect\hypertarget{_Toc41048839}{}{}}Figure
16: Temporal analysis of the lift coefficient and normalised cylinder
displacement signal at \(U^{*} = 13.6\). We show the lift coefficient
and normalised cylinder displacement signal side by side in (a), present
the temporal evolution of the phase lag \(\phi\) of Cl in (b) and show
the temporal evolution of the instantaneous frequency of the lift
coefficient signal in (c). The blue line in (a) represents the lift
coefficient signal, while the black line represents the normalised
cylinder displacement.

\subsection{\texorpdfstring{\protect\hypertarget{_Ref24287997}{}{\protect\hypertarget{_Toc41048814}{}{}}Transition
to SVIV
(\(\mathbf{15.9 <}\mathbf{U}^{\mathbf{*}}\mathbf{< 18.2}\))}{Transition to SVIV (\textbackslash{}mathbf\{15.9 \textless{}\}\textbackslash{}mathbf\{U\}\^{}\{\textbackslash{}mathbf\{*\}\}\textbackslash{}mathbf\{\textless{} 18.2\})}}\label{transition-to-sviv-mathbf15.9-mathbfumathbfmathbf-18.2}

Previously in the \(U^{*} < 14\) regime, we observe that the temporal
profile of both Cl and \(y^{*}\) are very similar to each other, except
that Cl leads \(y^{*}\) by a certain amount. This similarity in profile
supports the assertion that the vibration within \(U^{*} < 14\) is
driven exclusively by the shedding of Karman vortices, which brings the
onset of the alternating lift. By extension, one might expect a similar
profile between Cl and \(y^{*}\) even when the vibration is driven by
streamwise vortices. However, this presumption seems to not be the case.

Once we reach \(U^{*} = 15.9\), we observe that it has become difficult
to argue that the profile of \(y^{*}\) is just a lagged version of the
profile of Cl. This is shown in Figure 17(a), with the enlarged version
in Figure 17(b). The profile of Cl looks like the result of several
superimposed signals, which one can almost distinguish from the presence
of two types of maxima at two different amplitude heights. We put a red
dashed line and a red dashed-dot line in Figure 17(b) as visual cues
indicating the two amplitude heights. Decomposing the lift coefficient
signal using EEMD reveals partial evidence supporting the superimposed
(compound) signal hypothesis.

\includegraphics[width=5.40208in,height=4.29583in]{media/image65.png}

\protect\hypertarget{_Ref41047682}{}{\protect\hypertarget{_Toc41048840}{}{}}Figure
17: Temporal evolution of \(y^{*}\) and Cl at \(U^{*} = 15.9\). (b)
shows an enlarged view of (a). We can barely spot semblance of two
signals with different amplitudes superimposed in the Cl signal in (b).

Once we have decomposed the signal using EEMD, we replot Fig. 15(a)
using the component of Cl with the highest correlation to the original
\(y^{*}\) signal and present the comparison in Fig. 16(a). To represent
\(y^{*}\) in Fig. 16(a), we again chose its IMF component with the
highest correlation to the original \(y^{*}\) signal, as we have done in
Figs. 12, 13, and 14. One can clearly see that the part of Cl signal
responsible for driving the vibration at \(U^{*} = 15.9\) is embedded in
the original Cl signal, and decomposition via EEMD managed to recover
this signal whose profile is indeed similar to the profile of the
characteristic IMF of \(y^{*}\), except that it leads \(y^{*}\) on
average by approximately \(150\) (Fig. 16(b)). This decline from
\(\phi \approx 180\) at reduced velocities \(6.8 < U^{*} < 14\), to
\(\phi \approx 150\) at \(U^{*} = 15.9\) is quite sizeable, suggesting a
fundamental change in flow dynamics, particularly in terms of vortical
structure.

\includegraphics[width=5.40208in,height=5.80486in]{media/image67.png}

\protect\hypertarget{_Ref41047933}{}{\protect\hypertarget{_Toc41048841}{}{}}Figure
18: Temporal analysis of the lift coefficient component that has the
highest correlation to the original (normalised) cylinder displacement
signal, \(C_{Cl,\ y^{*}}\), and the normalised cylinder displacement
signal at \(U^{*} = 15.9\). The component was obtained by decomposing
the lift coefficient signal using EEMD. We show \(C_{Cl,y^{*}}\) and
\(y^{*}\) signal side by side in (a), present the temporal evolution of
the phase lag \(\phi\) of \(C_{Cl,y^{*}}\) in (b) and show the temporal
evolution of the instantaneous frequency of the \(C_{Cl,y^{*}}\) in (c).
The blue line in (a) represents the lift coefficient component signal,
while the black line represents the normalised cylinder displacement.

Inspecting the HHT spectrogram in Figure 18(c) reveals two dominant
bands in the frequency domain. The first one marked with a white
continuous rectangular box is the instantaneous frequency for the IMF
component of lift shown in Figure 18(a), and its mean frequency lies
close to the natural frequency of the system
(\(f_{\text{Cl}}^{*} \approx 1\)). There is however, a second band of
frequency with nearly similar amplitude around
\(f_{\text{Cl}}^{*} \approx 3.3\), marked with a white dashed
rectangular box. Computing the Strouhal number from this frequency
returns a value of \(\text{St} = 0.20\), which is very close to the
Strouhal number for Karman vortices as predicted by Eq.(20) at the
Reynolds number equivalent to \(U^{*} = 15.9\), which is
\(\text{Re} = 7.9 \times 10^{3}\). We thus attribute this second band of
frequency as being the footprint left by the shedding of Karman
vortices, and the first band (the component of Cl in Figure 18(a)) as
the result of streamwise vortex shedding.

The knowledge that Karman vortices continue to exist and shed from a
cruciform structure during SVIV is not new in the literature. However,
this is the first time that the lift signal from a cruciform structure
undergoing SVIV has been subjected to HHT, revealing the compositeness
of the lift signal and elucidating the footprint of the two dominant
vortical structures on it. Although the magnitude of the instantaneous
frequency due to Karman vortex is comparable to the streamwise vortex
(sometimes even bigger), the reason why the cylinder resists locking
into its frequency is perhaps because its frequency too distant from the
natural frequency of the system \(f_{n}\). The shedding frequency of the
streamwise vortex is much closer to \(f_{n}\) and is thus preferred by
the cylinder.

We consider the transition to SVIV to be complete at \(U^{*} = 18.2\),
when the mean phase lag \(\phi\) drops further to \(\approx 20\). Figure
17(a) and (b) documents this observation. The phase lag is observed to
slip through \(360\) at certain portions of the characteristic Cl
profile where there are slight distortions in the periodicity of the
IMF. The slipping through \(360\) was also observed by (Khalak and
Williamson, 1999) in their work on KVIV, which highlights the
quasi-periodic nature of the signal being analysed. The slip appeared in
(Khalak and Williamson, 1999) at the initial branch of KVIV. It may be
the case that the overall low value of \(\phi \approx 20\) at
\(U^{*} = 18.2\), coupled with the presence of \(\phi\) slippage is
suggesting the possibility of \(U^{*} = 18.2\) being the initial branch
for SVIV. We could not foresee this point brought up if the original Cl
signal is not decomposed beforehand, implying the utility of EEMD in
studying fluid-structure interactions with multiple dominant flow
structures.

\includegraphics[width=5.40208in,height=5.48542in]{media/image73.png}

\protect\hypertarget{_Ref41048365}{}{\protect\hypertarget{_Toc41048842}{}{}}Figure
19: Temporal analysis of the lift coefficient component that has the
highest correlation to the original (normalised) cylinder displacement
signal, \(C_{Cl,\ y^{*}}\), and the normalised cylinder displacement
signal at \(U^{*} = 18.2\). The component was obtained by decomposing
the lift coefficient signal using EEMD. We show \(C_{Cl,y^{*}}\) and
\(y^{*}\) signal side by side in (a), present the temporal evolution of
the phase lag \(\phi\) of \(C_{Cl,y^{*}}\) in (b) and show the temporal
evolution of the instantaneous frequency of the \(C_{Cl,y^{*}}\) in (c).
The blue line in (a) represents the lift coefficient component signal,
while the black line represents the normalised cylinder displacement.

\hypertarget{the-sviv-regime-mathbfumathbfmathbf-20}{\subsection{\texorpdfstring{The
SVIV Regime,
\(\mathbf{U}^{\mathbf{*}}\mathbf{> 20}\)}{The SVIV Regime, \textbackslash{}mathbf\{U\}\^{}\{\textbackslash{}mathbf\{*\}\}\textbackslash{}mathbf\{\textgreater{} 20\}}}\label{the-sviv-regime-mathbfumathbfmathbf-20}}

As \(U^{*}\) is increased to \(20.5\), we can see a jump in \(\phi\)
from a mean value of approximately \(20\) to about \(120\), shown in
Fig. 18(a). The phase slippage discussed previously is also observed in
this time series subset, indicating the quasi-periodic nature of the
lift coefficient signal at this \(U^{*}\). Incidentally, this
quasi-periodicity seems to be the norm for the lift signals up to
\(U^{*} = 27.3\), as suggested by the phase slippages evident in Figs.
18(b), (c), and (d). The slippage only stops once \(U^{*}\) reaches
\(29.5\), suggesting a more periodic behaviour of the lift coefficient
compared to its counterparts between \(20.5 \leq U^{*} \leq 27.3\).
Although the instantaneous phase between \(20.5 \leq U^{*} \leq 27.3\)
implies a quasi-periodic nature, their mean values at each \(U^{*}\) are
contained in the narrow region \(114 < \phi\ \left( \right) < 135\), as
is the value for \(\phi\) at \(U^{*} = 29.5\). This observation that the
value of \(\phi\) is only slowly varying with respect to \(U^{*}\), once
\(U^{*}\) increases past \(20.5\), can be interpreted as the dominant
flow structures settling into a stable form that becomes more resilient
against external excitations. Based on this feature, it seems
appropriate to classify \(20.5 \leq U^{*} \leq 29.5\) as the upper
branch of SVIV.

\includegraphics[width=5.40208in,height=4.40625in]{media/image79.png}

\protect\hypertarget{_Toc41048843}{}{}Figure 20: The instantaneous phase
lag \(\phi\) of the dominant component of the normalised cylinder
displacement signal (\(y^{*}\)) against \(C_{Cl,y^{*}}\) in the range
\(20 < U^{*} < 30\). See Figure 19 for the definition of
\(C_{Cl,y^{*}}\).

Figure 21 summarises our findings thus far, with respect to our analysis
of the Cl time series, specifically the ensemble average value of
\(\phi\), denoted as \(\phi_{\text{mean}}\). The region A denotes the
initial branch of (suppressed) KVIV, where \(\phi_{\text{mean}}\) is
close to zero. Region B denotes the upper/lower branch of (suppressed)
KVIV, where the system experiences a jump from
\(\phi_{\text{mean}} \approx 0\) to greater than \(110\). The value of
\(\phi_{\text{mean}}\) settles very close to \(180\) towards the end of
this upper/lower branch. The HHT spectrograms up to this \(U^{*}\) shows
only one dominant band of \(f_{\text{Cl}}^{*}\) which is close to the
Strouhal frequency of Karman vortex shedding.

\includegraphics[width=5.40208in,height=2.12500in]{media/image89.png}

\protect\hypertarget{_Ref41048465}{}{\protect\hypertarget{_Toc41048844}{}{}}Figure
21: Vibration regimes identified during analysis of \(\phi\). To capture
the evolution of \(\phi\) with respect to \(U^{*}\), a representative
value for \(\phi\) at each \(U^{*}\) must be selected. We chose to use
the mean \(\phi\) as the representative value.

Then, \(\phi_{\text{mean}}\) experiences a slight drop of about
one-sixth the value of \(\phi_{\text{mean}}\) at the preceding
upper/lower branch as we enter region C, marking the start of transition
to the SVIV regime. The emergence of two dominant instantaneous
frequency bands for \(f_{\text{Cl}}^{*}\) further supports this
demarcation. One of the dominant \(f_{\text{Cl}}^{*}\) band has a value
close to unity, while the other has a value close to the shedding
frequency of Karman vortex for a fixed, isolated circular cylinder at
the same Reynolds number. The system then undergoes a more sudden drop
to \(\phi_{\text{mean}} \approx 20\) at \(U^{*} = 18.2\). Inspecting the
temporal evolution of \(\phi\) revealed the quasi-periodic nature of Cl
at this \(U^{*}\), which is analogous to the KVIV initial branch studied
by (Khalak and Williamson, 1999), prompting us to assign the region up
to \(U^{*} = 20.5\) as the initial branch of SVIV (region D).

Finally, in region E we observe another jump in \(\phi_{\text{mean}}\)
from \(\phi_{\text{mean}} \approx 20\) to approximately \(120\) as
\(U^{*} > 20.5\). The Cl signal gradually loses its quasi-periodicity
with increasing \(U^{*}\), and the \(\phi_{\text{mean}}\) in this region
falls within the arguably narrow range of
\(114 < \phi\ \left( \right) < 135\), pointing to stabilisation of
dominant flow structures. We hence designate region E as the upper
branch of SVIV.

\hypertarget{mathematical-model-for-power-estimation}{\subsection{Mathematical
Model for Power
Estimation}\label{mathematical-model-for-power-estimation}}

The mathematical model for harnessable power estimation in this study
follows that which had been mentioned in (Raghavan, 2007; M. M. M.
Bernitsas \emph{et al.}, 2008; Bernitsas \emph{et al.}, 2009). In these
works, the authors mentioned that work done by the oscillating cylinder
\(W_{\text{cyl.}}\)during one cycle of oscillation \(T_{\text{osc.}}\)
is as follows.

\begin{longtable}[]{@{}lll@{}}
\toprule
&
\(W_{\text{cyl.}} = \int_{0}^{T_{\text{osc.}}}\left( F_{L} \cdot \dot{y} \right)dt,\)
& (21)\tabularnewline
\midrule
\endhead
\bottomrule
\end{longtable}

where both the lift \(F_{L}\) and cylinder velocity \(\dot{y}\) are both
functions of time. Through several manipulations and simplifying
assumptions (see (Sun \emph{et al.}, 2016) for updated derivation
steps), the power captured by the system can be written, using our
parameters, as the fluid power

\begin{longtable}[]{@{}lll@{}}
\toprule
&
\(P_{Fluid,RMS} = \frac{1}{2}\text{ρπ}C_{y,\text{RMS}}U^{2}f_{\text{osc.}}y_{\text{RMS}}\text{DL}\sin\left( \phi \right),\)
& \protect\hypertarget{_Ref41049025}{}{}(22)\tabularnewline
\midrule
\endhead
\bottomrule
\end{longtable}

or the mechanical power

\begin{longtable}[]{@{}lll@{}}
\toprule
&
\(P_{Mech.,RMS} = 8\pi^{3}m_{\text{eff.}}\zeta_{\text{Tot.}}\left( y_{\text{RMS}}f_{\text{osc.}} \right)^{2}f_{n}.\)
& \protect\hypertarget{_Ref41049030}{}{}(23)\tabularnewline
\midrule
\endhead
\bottomrule
\end{longtable}

Here, \(P_{Fluid,RMS}\), \(P_{Mech.,RMS}\), \(L\), \(C_{y,\text{RMS}}\),
\(\zeta_{\text{Tot.}}\) and \(m_{\text{eff.}}\) are the root mean square
of fluid power, root mean square of mechanical power, length of the
circular cylinder, root mean square of lift amplitude, total damping
coefficient, and the system effective mass respectively. We choose to
use root mean square (parameters subscript RMS) quantities in Eqs. (22)
and (23) instead of the maximum values like the original authors because
that may lead to a misunderstanding that the maximum value is sustained
throughout the observation window. This obviously is not always the case
in our study, especially once the flow transits to SVIV. Time series
analysis of \(y^{*}\left( t \right)\) and \(\text{Cl}\left( t \right)\)
in §\protect\hypertarget{currentWorkingPosition}{}{}4.1 revealed that
there is a degree of intermittency in both signals that cannot be
overlooked at certain ranges of \(U^{*}\), thus making it better to use
the root mean square values instead. Estimation of the root mean square
of harnessable power in our opinion makes more sense because it returns
a value that is continually approached by the system \emph{over time},
while the maximum, may be a one-off value.

Before presenting the results of our harnessable power estimation
following Eqs. (22) and (23), let us clarify our method of estimating
the root mean square of lift amplitude \(C_{y,\text{RMS}}\). Let
\(F_{L}\left( t \right)\) be the lift acting on the cylinder and
\(y\left( t \right)\) the cylinder displacement time series resulting
from that alternating lift. Decomposing \(F_{L}\left( t \right)\) via
EEMD yields a finite number \(N\) of IMFs which we can summarily write
as \(F_{L}\left( t \right) = \sum_{N}^{}{C_{i}\left( t \right)}\). The
IMF chosen as the component of lift driving \(y\left( t \right)\) is the
\(C_{i}\left( t \right)\) with the highest correlation with
\(y\left( t \right)\), the original cylinder displacement signal. This
automatically selects the component of lift with a mean frequency
closest to the mean frequency of \(y\left( t \right)\). We then compute
the root mean square value of that component of lift, giving us
\(C_{y,\text{RMS}}\).

Figure 21 shows the comparison between power estimated from our
experiment and numerical results, with the experimental results of
(Nguyen \emph{et al.}, 2012) and the direct power measurement of (Koide
\emph{et al.}, 2013). Only the value for \(P_{Mech.,RMS}\) is computed
from our experimental results due to the absence of lift data. Our
numerical results have both lift and cylinder displacement data and
hence, we computed both \(P_{Fluid,RMS}\) and \(P_{Mech.,RMS}\). We
estimated the power from the experimental results of (Nguyen \emph{et
al.}, 2012) by interpolating missing data points in both their amplitude
and frequency responses to compute the value of \(P_{Mech.,RMS}\) at a
given value of \(U^{*}\). The direct power measurement by (Koide
\emph{et al.}, 2013) was done by connecting the elastic support of the
cylinder to a coil. The coil moves with the cylinder, thus creating a
relative pistoning motion against a fixed magnet and produces an
alternating current.

\includegraphics[width=5.40208in,height=3.73958in]{media/image93.png}

\protect\hypertarget{_Ref41049462}{}{}Figure 22: Estimated root mean
square of mechanical power \(P_{Mech.,RMS}\), fluid power
\(P_{Fluid,RMS}\), or both, of our experimental and numerical results,
compared with results of similar studies in the literature. The fluid
power \(P_{Fluid,RMS}\) is calculated only from the results of our
numerical study as the others did not measure lift. The computation of
the instantaneous phase lag \(\phi\) requires both lift and cylinder
displacement signals.

We note that the evolution trend of estimated power with respect to
\(U^{*}\) is similar between \(P_{Mech.,RMS}\) from our experiment and
simulation, especially in the \(U^{*}\) region immediately after the
onset of SVIV. This makes sense since \(P_{Mech.,RMS}\) is basically a
single variable function, the variable being \(y_{\text{RMS}}\), with
the others fixed as we vary \(U^{*}\). The trend observed in
\(P_{Mech.,RMS}\) is thus a scaled version of the trend found in
\(y_{\text{RMS}}\). Nevertheless, besides this region of
\(18 < U^{*} < 23\) the trend between all data series compared in Figure
22 are relatively like each other. This is especially the case after
\(U^{*} > 23\), where we observe a fairly good agreement between
\(P_{Mech.,RMS}\) and \(P_{Fluid,RMS}\) computed from our experimental
and numerical results with the direct power measurements of (Koide
\emph{et al.}, 2013) and the estimated \(P_{Mech.,RMS}\) from the data
of (Nguyen \emph{et al.}, 2012). The estimated power in the KVIV regime
(\(U^{*} < 17\)) produces power only in the order of \(10\ \text{μW}\),
which is relatively insignificant in contrast to the magnitude of power
produced in the SVIV regime (mW).

\subsection{\texorpdfstring{Possibility for Increasing Fluid Power,
\(\mathbf{P}_{\mathbf{Fluid,RMS}}\)}{Possibility for Increasing Fluid Power, \textbackslash{}mathbf\{P\}\_\{\textbackslash{}mathbf\{Fluid,RMS\}\}}}\label{possibility-for-increasing-fluid-power-mathbfp_mathbffluidrms}

We have seen in Figure 22 the similarity in the evolution trend of
\(P_{Mech.,RMS}\) and \(P_{Fluid,RMS}\) against \(U^{*}\) of our
numerical results with those from (Nguyen \emph{et al.}, 2012) and
(Koide \emph{et al.}, 2013). However, recall that in order to represent
the amplitude of lift, we used the root mean square amplitude of the
component of lift that has the highest correlation with the normalised
cylinder displacement signal \(y^{*}\left( t \right)\). We did not use
the root mean square amplitude of the original lift signal, and yet we
obtained \(P_{Fluid,RMS}\) estimates that are in reasonable agreement
not only with its \(P_{Mech.,RMS}\) counterparts but with the actual
measured power of (Koide \emph{et al.}, 2013).

On one hand, this is an indication that the lift component selected for
use in computation is an arguably faithful representation of the force
driving the motion of the cylinder. On the other, this also suggests
that the motion of the cylinder, once it enters the SVIV regime, is
driven only by a subset of the total lift force. Another significant
subset of the lift force, once decomposed with EEMD, is the component
whose mean frequency is close to the Karman frequency of vortex
shedding, as explained in §4.5. This ``Karman'' component of lift has a
similar magnitude to the ``streamwise vortex'' component (or just
streamwise component for short) of lift, as evidenced inFigure 24, and
is therefore not negligible. The Karman components are marked with a
dashed, white box and the streamwise components are marked with a solid,
white box, following the convention in Figure 18 and Figure 19. However,
the Karman component fails to affect the cylinder vibration like the
streamwise component most probably due to the large difference between
the mean frequency of the Karman component and the natural frequency of
the system, \(f_{n}\). The streamwise component has a mean frequency
close to \(f_{n}\) and is hence able to synchronize with the vibration
of the cylinder, producing a sizeable amplitude response.

\includegraphics[width=5.40208in,height=2.99097in]{media/image94.png}

Figure 23: Evolution of the root mean square amplitude of two dominant
lift components: the Karman and streamwise components with respect to
\(U^{*}\). The region \(U^{*} < 23\) exhibits similar magnitude for both
the Karman and streamwise components of lift. On the other hand, the
magnitude of amplitude for the Karman component while the region
\(U^{*} > 23\) is almost always twice that of the streamwise component.

\includegraphics[width=5.40208in,height=5.64931in]{media/image95.png}

\protect\hypertarget{_Ref41049739}{}{}Figure 24: The instantaneous
frequency of the lift signal between \(20 < U^{*} < 30\). The white,
solid box encloses the region where the mean frequency is close to the
system natural frequency \(f_{n}\), while the dashed, white box encloses
the region where the mean frequency is close to the shedding frequency
of Karman vortex at the Reynolds number at which the simulation is
performed. Through visual inspection, we can see how the degree of
dispersion in the instantaneous frequency of the ``Karman component'' of
lift is about twice that of the ``streamwise component'' of lift.

Figure 24 shows the root mean square amplitude of the Karman and
streamwise components of lift in the SVIV regime \(U^{*} > 18\). Between
\(18 < U^{*} < 23\), the magnitude of the Karman and streamwise
components are nearly equal. However, once we exceed \(U^{*} = 23\),
Figure 24 shows that the contribution to the total lift by the Karman
component is on average twice the contribution of the streamwise
component. Let us assume a hypothetical situation where we can
consolidate the contribution by the Karman component under the
streamwise component of lift. Then, the value for \(C_{y,\text{RMS}}\)
in Eq. (22) will increase close to a factor of 2 when
\(18 < U^{*} < 23\), and close to a factor of
\protect\hypertarget{_Hlk24566759}{}{}3 when \(23 < U^{*} < 30\). This
increase in \(C_{y,\text{RMS}}\) will lead to a larger
\(P_{Fluid,RMS}\), if the value of the other parameters in Eq. (23) are
similar. The point of this exercise is to demonstrate a procedure to
estimate the room for improvement for future iterations of the cruciform
system with respect to
\(P_{Fluid,RMS}\).\protect\hypertarget{currentlyWorking}{}{}

Although not as pertinent in a vibrating system forced only by one
dominant mechanism, e.g. KVIV-based and galloping-based systems, we
think that it is advantageous to consider the contribution of different
vibration-driving mechanisms on the lift produced, in order to better
understand the phenomena, and tackle further improvements of the system
from a new perspective. This is especially true, in our opinion, in this
study where there exist simultaneous shedding of Karman and streamwise
vortices once we enter the SVIV regime. Once we have an estimate of how
much fluid energy is being taken up by the vortical structure that is
not the primary vibration forcing mechanism, we can then work to find
methods to redirect that energy towards the vortical structure that is
the primary vibration forcing mechanism, thus potentially harnessing an
even bigger power at the same \(U^{*}\).

.

\hypertarget{section-4}{\section{}\label{section-4}}

\textbf{THE MODIFIED CRUCIFORM}

\hypertarget{the-amplitude-frequency-response}{\subsection{The
amplitude-frequency response}\label{the-amplitude-frequency-response}}

\includegraphics[width=5.40208in,height=2.63577in]{media/image103.png}

\protect\hypertarget{_Toc41048845}{}{}Figure 25: The evolution of
\(y_{\text{RMS}}^{*}\) with respect to \(U^{*}\) and plate tilt angle.

\includegraphics[width=5.40208in,height=2.98333in]{media/image116.png}

\protect\hypertarget{_Toc41048846}{}{}Figure 26: The evolution of
\(F_{L,\text{RMS}}\) with respect to \(U^{*}\) and plate tilt angle.

\includegraphics[width=5.40208in,height=2.83194in]{media/image126.png}

\protect\hypertarget{_Toc41048847}{}{}Figure 27: Sketches of the
vortical structure in the moderate to high \(U^{*}\) region
(\(11 < U^{*} < 30\)). The sketches are made based on the flow
visualisation at maximum lift in a vibration cycle. This definition is
visualised in the lift/displacement versus time sketch at the top left
of the figure. The crucifix system is visualised from downstream. The
vortical structures at \(\theta_{\text{plate}} = 0\ \text{rad}\) and
\(\theta_{\text{plate}} = \frac{\pi}{8}\ \text{rad}\) are not streamwise
vortex in the usual sense as observed in
\(\theta_{\text{plate}} = \frac{\text{π~}}{2}\text{rad}\). Rather, they
are the streamwise undulation of Karman vortices. The vortical
structures driving the vibrations at
\(\theta_{\text{plate}} = 0\ \text{rad}\) and
\(\theta_{\text{plate}} = \frac{\pi}{8}\ \text{rad}\) are Karman
vortices with streamwise undulations.

\includegraphics[width=5.40208in,height=3.00000in]{media/image127.png}

\protect\hypertarget{_Toc41048848}{}{}Figure 28: Sketches of the
vortical structure in the low \(U^{*}\) region (\(2 < U^{*} < 10\)). The
sketches are made based on the flow visualisation at maximum lift in a
vibration cycle. This definition is visualised in the lift/displacement
versus time sketch at the top left of the figure. The crucifix system is
visualised from downstream. The vortical structures summarised are not
streamwise vortex in the usual sense as observed when
\(\theta_{\text{plate}} = \frac{\text{π~}}{2}\text{rad}\) at moderate to
high \(U^{*}\). Rather, they are the streamwise undulation of Karman
vortices. The vortical structures driving the vibrations (or lack
thereof) in the low \(U^{*}\) region are Karman vortices with streamwise
undulations.

\hypertarget{the-main-components-of-lift-driving-cylinder-vibration}{\subsection{The
main components of lift driving cylinder
vibration}\label{the-main-components-of-lift-driving-cylinder-vibration}}

\includegraphics[width=5.40208in,height=2.83118in]{media/image147.png}

\protect\hypertarget{_Ref39495423}{}{\protect\hypertarget{_Toc41048849}{}{}}Figure
29: Evolution of the lift force with varying \(U^{*}\). The results are
summarised for every tilt angle of the strip plate. The horizontal axis
denotes reduced velocity while the vertical axis denotes the
root-mean-square of the IMF component of lift that is most correlated to
the dominant IMF component of \(y^{*}\) (blue point) and the IMF
component of lift with the highest root-mean-square amplitude (grey
point).

\includegraphics[width=5.40208in,height=2.77778in]{media/image147.png}

\protect\hypertarget{_Toc41048850}{}{}Figure 30: Evolution of the lift
force with varying \(U^{*}\). Like Figure 29, but with enlarged vertical
axes to ease comparison between tilt angles. Refer to Figure 29 for
caption.

\includegraphics[width=5.40208in,height=2.91667in]{media/image167.png}

\protect\hypertarget{_Toc41048851}{}{}Figure 31: The ratio of the
root-mean-square amplitude of the IMF component of lift most correlated
to the \(y^{*}\) signal, to the sum between the root-mean-square
amplitude of the IMF component of lift most correlated to the \(y^{*}\)
signal and the IMF component of lift with maximum root-mean-square
amplitude.

\hypertarget{estimated-power-map-in-mathbfalphamathbfumathbf-space-and-its-interpretation}{\subsection{\texorpdfstring{Estimated
power map in \(\mathbf{\alpha}\)--\(\mathbf{U}^{\mathbf{*}}\) space and
its
interpretation}{Estimated power map in \textbackslash{}mathbf\{\textbackslash{}alpha\}--\textbackslash{}mathbf\{U\}\^{}\{\textbackslash{}mathbf\{*\}\} space and its interpretation}}\label{estimated-power-map-in-mathbfalphamathbfumathbf-space-and-its-interpretation}}

\includegraphics[width=5.40208in,height=2.73194in]{media/image177.png}

\protect\hypertarget{_Toc41048852}{}{}Figure 32: The estimated apparent
power computed using the following formula,
\(P_{a} = 8\pi^{3}m_{\text{eff.}}\zeta_{\text{tot.}}\left( Af_{\text{osc}} \right)^{2}f_{n,water}\),
based on the power estimation formula for mechanical power described in
(Sun \emph{et al.}, 2015).

\hypertarget{section-5}{\section{}\label{section-5}}

\textbf{CONCLUSIONS}

\hypertarget{conclusions}{\subsection{Conclusions}\label{conclusions}}

In this report, we have reviewed the methods and models used to compute
harnessable power from a circular cylinder in vibration due to the
shedding of Karman and streamwise vortices. We showed the method for
computing instantaneous power produced by the cylinder vibration while
discussing the model put forward by other researchers. The model,
dependent as it is on global quantities of the vibration and force
signal, gives a more in-depth insight into the variables that affect
harnessable power. In contrast, the instantaneous power formula gives us
value for power at any given instant without knowing much what is
happening behind the scenes.

We have analysed several regimes based on the cylinder vibration
characteristic. The pre-Karman VIV regime saw the flow dominated by
turbulent fluctuations, which is wholly overpowered by vortical
structures once we are in the KVIV regime. Transition to SVIV saw the
lift signal becoming more complex, with multiple flow structures leaving
their imprint on the lift signal. The phase lag shows that the SVIV
regime is shown to be a vortex, not galloping-induced phenomena.

Analysis of the streamwise and Karman vortex-based lift signals suggest
that a significant portion of fluid energy is dissipated as Karman
vortex lift in the SVIV regime. The reduction of this is critical, and
some evidence points towards the length of the circular cylinder as
holding the answer. We need to investigate this more in the future.

\protect\hypertarget{_Toc461037287}{}{\protect\hypertarget{_Toc461037470}{}{\protect\hypertarget{_Toc41048823}{}{}}}REFERENCES

Abdullah, K., 2011. \emph{Integrated River Basin Management Report},

Abdullah, K., 2004. Stormwater Management and Road Tunnel (SMART) a
Lateral Approach to Flood Mitigation Works. \emph{International
Conference on Bridge Engineering \& Hydraulic Structures}, pp.59--79.

Ahsan, N., 2015. Computational Analysis of Vortex-Induced Vibration and
its Potential in Energy Harvesting. , pp.437--443.

Ali, M.S.M., Doolan, C.J. and Wheatley, V., 2012. Low Reynolds number
flow over a square cylinder with a detached flat plate.
\emph{International Journal of Heat and Fluid Flow}, 36, pp.133--141.
Available at: http://dx.doi.org/10.1016/j.ijheatfluidflow.2012.03.011.

Arai, T. et al., 2011. Streamwise Vortices Introduced by ``Hyper-Mixer''
on Supersonic Mixing. \emph{17th AIAA International Space Planes and
Hypersonic Systems and Technologies Conference}, (April), pp.1--8.

Assi, G.R. da S. et al., 2013. The role of wake stiffness on the
wake-induced vibration of the downstream cylinder of a tandem pair.
\emph{Journal of Fluid Mechanics}, 718, pp.210--245.

Assi, G.R.S., Bearman, P.W. and Meneghini, J.R., 2010. On the
wake-induced vibration of tandem circular cylinders: the vortex
interaction excitation mechanism. \emph{Journal of Fluid Mechanics},
661, pp.365--401.

Bae, H.M. et al., 2001. Suppression of Karman Vortex Excitation of a
Circular Cylinder By a Second Cylinder set Downstream in Cruciform
Arrangement. \emph{Journal of Computational and Applied Mechanics},
2(2), pp.175--188. Available at:
http://www.ncbi.nlm.nih.gov/pubmed/15003161\%5Cnhttp://cid.oxfordjournals.org/lookup/doi/10.1093/cid/cir991\%5Cnhttp://www.scielo.cl/pdf/udecada/v15n26/art06.pdf\%5Cnhttp://www.scopus.com/inward/record.url?eid=2-s2.0-84861150233\&partnerID=tZOtx3y1.

Bae, H.M., Hirai, T., Sano, M. and Shirakashi, M., 1992. Excitation by
Vortices Shedding Periodically from Two Cylinders in Cruciform
Arrangement. \emph{Transactions of the Japan Society of Mechanical
Engineers Series B}, 58(551). Available at:
https://www.jstage.jst.go.jp/article/kikaib1979/58/551/58\_551\_2093/\_article.

Barrero-Gil, A., Alonso, G. and Sanz-Andres, A., 2010. Energy harvesting
from transverse galloping. \emph{Journal of Sound and Vibration},
329(14), pp.2873--2883. Available at:
http://www.sciencedirect.com/science/article/pii/S0022460X10000891
{[}Accessed: 18 January 2016{]}.

Barry, O., Oguamanam, D.C.D. and Lin, D.C., 2010. Free Vibration
Analysis of a Single Conductor With a Stockbridge Damper.
\emph{Proceedings of the 23rd CANCAM}. 2010 pp. 950--952.

Bearman, P.W., 2011. Circular cylinder wakes and vortex-induced
vibrations. \emph{Journal of Fluids and Structures}, 27(5--6),
pp.648--658. Available at:
http://www.sciencedirect.com/science/article/pii/S0889974611000600
{[}Accessed: 20 January 2016{]}.

Bernitsas, M.M., Ben-Simon, Y., Raghavan, K. and Garcia, E.M.H., 2009.
The VIVACE Converter: Model Tests at High Damping and Reynolds Number
Around 10{[}sup 5{]}. \emph{Journal of Offshore Mechanics and Arctic
Engineering}, 131(1), p.011102. Available at:
http://offshoremechanics.asmedigitalcollection.asme.org/article.aspx?articleid=1472649.

Bernitsas, M.M. and Raghavan, K., 2008. Reduction/suppression of VIV of
circular cylinders through roughness distribution at 8×103 \&lt; Re
\&lt; 1.5×105. \emph{Proceedings of the International Conference on
Offshore Mechanics and Arctic Engineering - OMAE}. 2008

Bernitsas, M.M., Raghavan, K., Ben-Simon, Y. and Garcia, E.M.H., 2008.
VIVACE (Vortex Induced Vibration Aquatic Clean Energy): A New Concept in
Generation of Clean and Renewable Energy From Fluid Flow. \emph{Journal
of Offshore Mechanics and Arctic Engineering}.

Bernitsas, M.M.M., Raghavan, K., Ben-Simon, Y. and Garcia, E.M.H.M.H.,
2008. VIVACE (Vortex Induced Vibration Aquatic Clean Energy): A New
Concept in Generation of Clean and Renewable Energy From Fluid Flow.
\emph{Journal of Offshore Mechanics and Arctic Engineering}, 130(4),
p.041101. Available at:
http://www.scopus.com/record/display.url?eid=2-s2.0-56749179917\&origin=resultslist\&sort=plf-f\&src=s\&st1=A+new+concept+in+generation+of+clean+and+renewable+energy+from+fluid+flow\&sid=620865A71FAF26768C42655E1E8BC194.aXczxbyuHHiXgaIW6Ho7g:230\&sot=b\&sdt=b\&
{[}Accessed: 2 February 2016{]}.

Billa, L., Mansor, S. and Mahmud, A.R., 2004. Spatial information
technology in flood early warning systems: an overview of theory,
application and latest developments in Malaysia. \emph{Disaster
Prevention and Management}, 13, pp.356--363.

Borazjani, I. and Daghooghi, M., 2013. The fish tail motion forms an
attached leading edge vortex. \emph{Proceedings. Biological sciences /
The Royal Society}, 280(February), p.20122071. Available at:
http://www.pubmedcentral.nih.gov/articlerender.fcgi?artid=3574357\&tool=pmcentrez\&rendertype=abstract.

Chen, Z.Q., Liu, M., Hua, X. and Mou, T., 2012. Flutter, Galloping, and
Vortex-Induced Vibrations of H-Section Hangers. \emph{Journal of Bridge
Engineering}, 17(3), pp.500--508.

Chia, C.W., 2004. Managing flood problems in Malaysia. \emph{Buletin
Ingeniur}.

Deng, J., Ren, A.-L. and Shao, X.-M., 2007. The flow between a
stationary cylinder and a downstream elastic cylinder in cruciform
arrangement. \emph{Journal of Fluids and Structures}, 23(5),
pp.715--731. Available at:
https://www.sciencedirect.com/science/article/pii/S0889974606001472
{[}Accessed: 12 September 2018{]}.

Department of Irrigation and Drainage Malaysia, 2010. Hydrological
Procedure No.27 (Estimation of Design Flood Hydrograph).

Department of Irrigation and Drainage Malaysia, Managing The Flood
Problem In Malaysia.
\emph{Http://Www.Water.Gov.My/Images/Pdf/Managing\_Flood.Pdf}, pp.1--11.

Derakhshandeh, J.F., Arjomandi, M., Dally, B. and Cazzolato, B., 2014.
The effect of arrangement of two circular cylinders on the maximum
efficiency of Vortex-Induced Vibration power using a Scale-Adaptive
Simulation model. \emph{Journal of Fluids and Structures}, 49,
pp.654--666. Available at:
http://dx.doi.org/10.1016/j.jfluidstructs.2014.06.005.

Diana, G., Cigada, A., Belloli, M. and Vanali, M., 2003.
Stockbridge-type damper effectiveness evaluation: Part I - Comparison
between tests on span and on the shaker. \emph{IEEE Transactions on
Power Delivery}, 18(4), pp.1462--1469.

Ding, L., Zhang, L., Wu, C., et al., 2015. Flow induced motion and
energy harvesting of bluff bodies with different cross sections.
\emph{Energy Conversion and Management}.

Ding, L., Zhang, L., Kim, E.S. and Bernitsas, M.M., 2015. URANS vs.
experiments of flow induced motions of multiple circular cylinders with
passive turbulence control. \emph{Journal of Fluids and Structures}, 54,
pp.612--628.

Doare, O. and Michelin, S., 2011. Piezoelectric coupling in
energy-harvesting fluttering flexible plates: Linear stability analysis
and conversion efficiency. \emph{Journal of Fluids and Structures},
27(8), pp.1357--1375.

Drucker, E.G. and Lauder, G. V, 1999. Locomotor forces on a swimming
fish: three-dimensional vortex wake dynamics quantified using digital
particle image velocimetry. \emph{The Journal of Experimental Biology},
202, pp.2393--2412. Available at:
http://www.ncbi.nlm.nih.gov/pubmed/10460729.

Gonçalves, R.T. et al., 2013. Two-degree-of-freedom vortex-induced
vibration of circular cylinders with very low aspect ratio and small
mass ratio. \emph{Journal of Fluids and Structures}, 39(March),
pp.237--257.

Govardhan, R.N. and Williamson, C.H.K., 2006. Defining the `modified
Griffin plot' in vortex-induced vibration: revealing the effect of
Reynolds number using controlled damping. \emph{Journal of Fluid
Mechanics}, 561, p.147. Available at:
http://journals.cambridge.org/abstract\_S0022112006000310 {[}Accessed: 8
March 2016{]}.

Harris, I., Jones, P.D., Osborn, T.J. and Lister, D.H., 2014. Updated
high-resolution grids of monthly climatic observations - the CRU TS3.10
Dataset. \emph{International Journal of Climatology}, 34(3),
pp.623--642. Available at: http://doi.wiley.com/10.1002/joc.3711
{[}Accessed: 11 July 2014{]}.

Hayashi, M., Sakurai, A. and Ohya, Y., 2006. Wake interference of a row
of normal flat plates arranged side by side in a uniform flow.
\emph{Journal of Fluid Mechanics}, 164(1), p.1. Available at:
http://www.journals.cambridge.org/abstract\_S0022112086002446.

Huang, N.E. and Wu, Z., 2008. A review on Hilbert-Huang transform:
Method and its applications to geophysical studies. \emph{Reviews of
Geophysics}, 46(2), p.RG2006. Available at:
http://doi.wiley.com/10.1029/2007RG000228 {[}Accessed: 27 February
2019{]}.

Kato, N., Koide, M., Takahashi, T. and Shirakashi, M., 2006. Influence
of Cross-Sectional Configuration on the Longitudinal Vortex Excitation
of the Upstream Cylinder in Cruciform Two-Cylinder System. \emph{Journal
of Fluid Science and Technology}, 1(2), pp.126--137. Available at:
http://joi.jlc.jst.go.jp/JST.JSTAGE/jfst/1.126?from=CrossRef
{[}Accessed: 17 December 2017{]}.

Kato, N., Koide, M., Takahashi, T. and Shirakashi, M., 2007. Vibration
Control for a Circular Cylinder by a Strip-plate Set Downstream in
Cruciform Arrangement (1st Report, Influence of a Downstream Strip-plate
on the Shedding of Longitudinal Vortices from Fixed System).
\emph{Transactions of the Japan Society of Mechanical Engineers Series
B}, 73(728), pp.957--964. Available at:
https://www.jstage.jst.go.jp/article/kikaib1979/73/728/73\_728\_957/\_article/-char/ja/
{[}Accessed: 13 March 2016{]}.

Kato, N., Koide, M., Takahashi, T. and Shirakashi, M., 2012. VIVs of a
circular cylinder with a downstream strip-plate in cruciform
arrangement. \emph{Journal of Fluids and Structures}, 30, pp.97--114.

Kawabata, Y., Takahashi, T., Haginoya, T. and Shirakashi, M., 2013.
Interference Effect of Downstream Strip-Plate on the Crossflow Vibration
of a Square Cylinder. \emph{Journal of Fluid Science and Technology},
8(3), pp.647--658.

Kawabata, Y., Takahashi, T. and Shirakashi, M., 2009. Influence of a
Downstream Strip-plate on Galloping of a Square Cylinder in Cruciform
Arrangement. \emph{日本機械学会論文集(B編)}, 75(754).

Kementerian Kewangan Malaysia, 2015. Bajet 2016. Available at:
http://www.treasury.gov.my/index.php?option=com\_content\&view=article\&id=6437:ucapan-bajet-2016\&catid=256\&Itemid=2472\&lang=ms.

Khalak, A. and Williamson, C.H.K., 1999. MOTIONS, FORCES AND MODE
TRANSITIONS IN VORTEX-INDUCED VIBRATIONS AT LOW MASS-DAMPING.
\emph{Journal of Fluids and Structures}, 13(7--8), pp.813--851.
Available at:
http://www.sciencedirect.com/science/article/pii/S0889974699902360
{[}Accessed: 10 March 2016{]}.

Kim, Y.-M. and You, K.-P., 2002. Dynamic responses of a tapered tall
building to wind loads. \emph{Journal of Wind Engineering and Industrial
Aerodynamics}, 90(12--15), pp.1771--1782. Available at:
http://www.sciencedirect.com/science/article/pii/S0167610502002866
{[}Accessed: 10 March 2016{]}.

Kluger, J.M., Moon, F.C. and Rand, R.H., 2013. Shape optimization of a
blunt body Vibro-wind galloping oscillator. \emph{Journal of Fluids and
Structures}, 40, pp.185--200. Available at:
http://www.sciencedirect.com/science/article/pii/S0889974613000868
{[}Accessed: 10 March 2016{]}.

Koide, M., Sekizaki, T., et al., 2009. A Novel Technique for
Hydroelectricity Utilizing Vortex Induced Vibration. \emph{Proceedings
of the ASME Pressure Vessels and Piping Division Conference,
PVP2009-77487}. 2009

Koide, M. et al., 2007. Influence Of A Cruciform Arrangement Downstream
Strip-Plate On Crossflow Vibration. \emph{Journal of Computational and
Applied Mechanics}, 8(2), pp.135--148.

Koide, M. et al., 2013. Prospect of Micro Power Generation Utilizing VIV
in Small Stream Based on Verification Experiments of Power Generation in
Water Tunnel. \emph{Journal of Fluid Science and Technology}, 8(3),
pp.294--308. Available at:
https://www.jstage.jst.go.jp/article/jfst/8/3/8\_294/\_article
{[}Accessed: 13 March 2016{]}.

Koide, M. et al., 2006. Vortex Excitation Caused by Longitudinal
Vortices Shedding from Cruciform Cylinder System in Water Flow.
\emph{JSME International Journal}, 49(4), pp.1043--1048.

Koide, M., Kato, N., Takahashi, T. and Shirakashi, M., 2009. Vibration
Control for a Circular Cylinder by a Strip-Plate Set Downstream in
Cruciform Arrangement (2nd Report, Generation and Suppression of Vortex
Excitation on Elastically Supported Cylinder).
\emph{日本機械学会論文集(B編)}, 75(752), pp.691--699.

Koide, M., Oogane, K., Takahashi, T. and Shirakashi, M., 2004.
Experimental Study on Universality of Longitudinal Vortices Shedding
Periodically from Crisscross Circular Cylinder System in Uniform Flow.
\emph{Transactions of the Visualization Society of Japan}, 24(4),
pp.15--22. Available at:
https://www.jstage.jst.go.jp/article/tvsj/24/4/24\_4\_15/\_article.

Koide, M., Takahashi, T., Shirakashi, M. and Salim, S.A.Z.B.S., 2017.
Three-dimensional structure of longitudinal vortices shedding from
cruciform two-cylinder systems with different geometries. \emph{Journal
of Visualization}, pp.1--11.

Kuala Lumpur Monsoon Activity Centre, 2015. \emph{Northeast Monsoon
Report},

Larsen, C.M. and Halse, K.H., 1997. Comparison of models for vortex
induced vibrations of slender marine structures. \emph{Marine
Structures}, 10(6), pp.413--441. Available at:
http://www.sciencedirect.com/science/article/pii/S0951833997000117
{[}Accessed: 22 January 2016{]}.

Liao, J.C., Beal, D.N., Lauder, G. V and Triantafyllou, M.S., 2003. Fish
Exploiting Vortices Decrease Muscle Activity. \emph{Science}, 302(2003),
pp.1566--1569.

Luo, S.C., Chew, Y.T. and Ng, Y.T., 2003. Hysteresis phenomenon in the
galloping oscillation of a square cylinder. \emph{Journal of Fluids and
Structures}, 18(1), pp.103--118. Available at:
http://www.sciencedirect.com/science/article/pii/S0889974603000847
{[}Accessed: 10 March 2016{]}.

Mackowski, A.W. and Williamson, C.H.K., 2013. An experimental
investigation of vortex-induced vibration with nonlinear restoring
forces. \emph{Physics of Fluids}, 25(8).

Maruai, N.M., Ali, M.S.M., Ismail, M.H. and Zaki, S.A., 2018.
Flow-induced vibration of a square cylinder and downstream flat plate
associated with micro-scale energy harvester. \emph{Journal of Wind
Engineering and Industrial Aerodynamics}, 175, pp.264--282. Available
at:
https://www.scopus.com/inward/record.uri?eid=2-s2.0-85042219159\&doi=10.1016\%2Fj.jweia.2018.01.010\&partnerID=40\&md5=2f3f62b94bb69ced3368b32e682aefc7.

Maruai, N.M., Mat Ali, M.S., Ismail, M.H. and Shaikh Salim, S.A.Z.,
2017. Downstream flat plate as the flow-induced vibration enhancer for
energy harvesting. \emph{Journal of Vibration and Control},
p.107754631770787. Available at:
http://journals.sagepub.com/doi/10.1177/1077546317707877 {[}Accessed: 2
January 2018{]}.

Mat Ali, M.S., Doolan, C.J. and Wheatley, V., 2011. Low Reynolds number
flow over a square cylinder with a splitter plate. \emph{Physics of
Fluids}, 23(3).

Nakamura, T. et al., 2013. \emph{Flow-Induced Vibrations:
Classifications and Lessons from Practical Experiences},
Butterworth-Heinemann.

Nakashima, M., 1986. Vortex excitation. , 163, pp.149--169.

Nguyen, T. et al., 2012. Influence of mass and damping ratios on VIVs of
a cylinder with a downstream counterpart in cruciform arrangement.
\emph{Journal of Fluids and Structures}, 28, pp.40--55.

Norman, J., 2012. VIVACE Hydrokinetic Power Generator System : The
Solution for Fish Protection and Debris Avoidance in Alaskan Rivers. ,
pp.2--6.

Ogink, R.H.M. and Metrikine, A.V., 2010. A wake oscillator with
frequency dependent coupling for the modeling of vortex-induced
vibration. \emph{Journal of Sound and Vibration}, 329(26),
pp.5452--5473. Available at:
http://www.sciencedirect.com/science/article/pii/S0022460X10004621
{[}Accessed: 10 March 2016{]}.

Pineirua, M., Doaré, O. and Michelin, S., 2015. Influence and
optimization of the electrodes position in a piezoelectric energy
harvesting flag. \emph{Journal of Sound and Vibration}, 346,
pp.200--215.

Quen, L.K. et al., 2014. Investigation on the effectiveness of helical
strakes in suppressing VIV of flexible riser. \emph{Applied Ocean
Research}, 44, pp.82--91. Available at:
http://www.sciencedirect.com/science/article/pii/S0141118713001028
{[}Accessed: 10 March 2016{]}.

Raghavan, K., 2007. \emph{Energy Extraction from a Steady Flow Using
Vortex Induced Vibration.} The University of Michigan.

Seyed-Aghazadeh, B., Carlson, D.W. and Modarres-Sadeghi, Y., 2015. The
influence of taper ratio on vortex-induced vibration of tapered
cylinders in the crossflow direction. \emph{Journal of Fluids and
Structures}, 53, pp.84--95. Available at:
http://www.sciencedirect.com/science/article/pii/S0889974614001765
{[}Accessed: 4 May 2017{]}.

Shiraishi, N., Matsumoto, M., Shirato, H. and Ishizaki, H., 1988. On
aerodynamic stability effects for bluff rectangular cylinders by their
corner-cut. \emph{Journal of Wind Engineering and Industrial
Aerodynamics}, 28(1--3), pp.371--380. Available at:
http://www.sciencedirect.com/science/article/pii/016761058890133X
{[}Accessed: 10 March 2016{]}.

Shirakashi, M., Mizuguchi, K. and Bae, H.M., 1989. Flow-induced
excitation of an elastically-supported cylinder caused by another
located downstream in cruciform arrangement. \emph{Journal of Fluids and
Structures}, 3(6), pp.595--607.

Su, H., Li, H., Chen, Z. and Wen, Z., 2016. An approach using ensemble
empirical mode decomposition to remove noise from prototypical
observations on dam safety. \emph{SpringerPlus}, 5, p.650. Available at:
http://www.ncbi.nlm.nih.gov/pubmed/27330916 {[}Accessed: 14 February
2019{]}.

Sukri Mat Ali, M., Doolan, C.J. and Wheatley, V., 2013. Aeolian Tones
Generated by a Square Cylinder with a Detached Flat Plate. \emph{AIAA
Journal}, 51(2), pp.291--301. Available at:
http://arc.aiaa.org/doi/abs/10.2514/1.J051378.

Sukri Mat Ali, M., Doolan, C.J. and Wheatley, V., 2011. The sound
generated by a square cylinder with a splitter plate at low Reynolds
number. \emph{Journal of Sound and Vibration}, 330(15), pp.3620--3635.
Available at: http://dx.doi.org/10.1016/j.jsv.2011.03.008.

Sun, H. et al., 2016. Effect of mass-ratio, damping, and stiffness on
optimal hydrokinetic energy conversion of a single, rough cylinder in
flow induced motions. \emph{Renewable Energy}, 99, pp.936--959.
Available at:
http://www.sciencedirect.com/science/article/pii/S0960148116306206
{[}Accessed: 12 February 2017{]}.

Sun, H., Soo Kim, E., Bernitsas, M.P. and Bernitsas, M.M., 2015. Virtual
Spring--Damping System for Flow-Induced Motion Experiments.
\emph{Journal of Offshore Mechanics and Arctic Engineering}.

Sunami, T. et al., 2002. Mixing and Combustion Control Strategies for
Effecient Scramjet Operation in Wide Range of Flight Mach Numbers.
\emph{11th AIAA/AAAF International Space Planes and Hypersonic Systems
and Technologies Conference}, (AIAA 2002-5116).

Takahashi, T., Baranyi, L. and Shirakashi, M., 1999. Configuration and
Frequency of Longitudinal Vortices Shedding from Circular Cylinders in
Cruciform Arrangement. \emph{Journal of the Visualization Society of
Japan}, 19(75), pp.64--72.

Takaijudin, H. et al., 2010. Implementation of Urban Stormwater
Management Practices in Malaysia. \emph{International Conference on
Sustainable Building and Infrastructure (ICSBI 2010)}, p.5.

Tanaka, H. et al., 2012. Experimental investigation of aerodynamic
forces and wind pressures acting on tall buildings with various
unconventional configurations. \emph{Journal of Wind Engineering and
Industrial Aerodynamics}, 107--108, pp.179--191. Available at:
http://www.sciencedirect.com/science/article/pii/S016761051200116X
{[}Accessed: 10 March 2016{]}.

Vandiver, J.K., Jaiswal, V. and Jhingran, V., 2009. Insights on
vortex-induced, traveling waves on long risers. \emph{Journal of Fluids
and Structures}, 25(4), pp.641--653. Available at:
http://www.sciencedirect.com/science/article/pii/S0889974608001321
{[}Accessed: 10 March 2016{]}.

Venugopal, A., Agrawal, A. and Prabhu, S. V, 2011. Review on vortex
flowmeter---Designer perspective. \emph{Sensors and Actuators A:
Physical}, 170(1), pp.8--23.

Williamson, C.H.K., 1996. Vortex Dynamics in the Cylinder Wake.
\emph{Annual Review of Fluid Mechanics}, 28(1), pp.477--539. Available
at:
http://www.annualreviews.org/doi/10.1146/annurev.fl.28.010196.002401.

Williamson, C.H.K. and Govardhan, R., 2004. Vortex-Induced Vibrations.
\emph{Annual Review of Fluid Mechanics}, 36(1), pp.413--455. Available
at:
http://arjournals.annualreviews.org/doi/abs/10.1146\%252Fannurev.fluid.36.050802.122128.

Wu, W., 2011. \emph{Two-Dimensional RANS Simulation of Flow Induced
Motion of Circular Cylinder with Passive Turbulence Control}. The
University of Michigan.

Wu, W., Bernitsas, M.M. and Maki, K., 2011. RANS Simulation vs.
Experiments of Flow Induced Motion of Circular Cylinder With Passive
Turbulence Control at 35,000\textless{}Re\textless{}130,000.
\emph{Volume 7: CFD and VIV; Offshore Geotechnics}. 1 January 2011 ASME,
pp. 733--744.

Xia, Y., Michelin, S. and Doaré, O., 2015. Fluid-solid-electric lock-in
of energy-harvesting piezoelectric flags. \emph{Physical Review
Applied}, 3(1), p.14009.

Xia, Y., Michelin, S.S., Doaré, O. and Doare, O., 2015.
Resonance-induced enhancement of the energy harvesting performance of
piezoelectric flags. \emph{Applied Physics Letters}, 107(26), pp.1--5.

Xiao, F., Chen, G.S., Zatar, W. and Hulsey, J.L., 2018. Quantification
of Dynamic Properties of Pile Using Ensemble Empirical Mode
Decomposition. \emph{Advances in Civil Engineering}, 2018, pp.1--6.
Available at: https://www.hindawi.com/journals/ace/2018/8379871/
{[}Accessed: 14 February 2019{]}.

Xu, J., He, M. and Bose, N., 2009. Vortex modes and vortex-induced
vibration of a long, flexible riser. \emph{Ocean Engineering}, 36(6--7),
pp.456--467. Available at:
http://www.sciencedirect.com/science/article/pii/S0029801809000146
{[}Accessed: 10 March 2016{]}.

Zdravkovich, M.M., 1981. Review and classification of various
aerodynamic and hydrodynamic means for suppressing vortex shedding.
\emph{Journal of Wind Engineering and Industrial Aerodynamics}, 7(2),
pp.145--189. Available at:
http://www.sciencedirect.com/science/article/pii/0167610581900362
{[}Accessed: 4 January 2016{]}.

Zhao, M. and Lu, L., 2018. Numerical simulation of flow past two
circular cylinders in cruciform arrangement. \emph{Journal of Fluid
Mechanics}, 848, pp.1013--1039. Available at:
https://www.cambridge.org/core/product/identifier/S0022112018003804/type/journal\_article
{[}Accessed: 12 September 2018{]}.

Abdullah, K., 2011. \emph{Integrated River Basin Management Report},

Abdullah, K., 2004. Stormwater Management and Road Tunnel (SMART) a
Lateral Approach to Flood Mitigation Works. \emph{International
Conference on Bridge Engineering \& Hydraulic Structures}, pp.59--79.

Ahsan, N., 2015. Computational Analysis of Vortex-Induced Vibration and
its Potential in Energy Harvesting. , pp.437--443.

Ali, M.S.M., Doolan, C.J. and Wheatley, V., 2012. Low Reynolds number
flow over a square cylinder with a detached flat plate.
\emph{International Journal of Heat and Fluid Flow}, 36, pp.133--141.
Available at: http://dx.doi.org/10.1016/j.ijheatfluidflow.2012.03.011.

Arai, T. et al., 2011. Streamwise Vortices Introduced by ``Hyper-Mixer''
on Supersonic Mixing. \emph{17th AIAA International Space Planes and
Hypersonic Systems and Technologies Conference}, (April), pp.1--8.

Assi, G.R. da S. et al., 2013. The role of wake stiffness on the
wake-induced vibration of the downstream cylinder of a tandem pair.
\emph{Journal of Fluid Mechanics}, 718, pp.210--245.

Assi, G.R.S., Bearman, P.W. and Meneghini, J.R., 2010. On the
wake-induced vibration of tandem circular cylinders: the vortex
interaction excitation mechanism. \emph{Journal of Fluid Mechanics},
661, pp.365--401.

Bae, H.M. et al., 2001. Suppression of Karman Vortex Excitation of a
Circular Cylinder By a Second Cylinder set Downstream in Cruciform
Arrangement. \emph{Journal of Computational and Applied Mechanics},
2(2), pp.175--188. Available at:
http://www.ncbi.nlm.nih.gov/pubmed/15003161\%5Cnhttp://cid.oxfordjournals.org/lookup/doi/10.1093/cid/cir991\%5Cnhttp://www.scielo.cl/pdf/udecada/v15n26/art06.pdf\%5Cnhttp://www.scopus.com/inward/record.url?eid=2-s2.0-84861150233\&partnerID=tZOtx3y1.

Bae, H.M., Hirai, T., Sano, M. and Shirakashi, M., 1992. Excitation by
Vortices Shedding Periodically from Two Cylinders in Cruciform
Arrangement. \emph{Transactions of the Japan Society of Mechanical
Engineers Series B}, 58(551). Available at:
https://www.jstage.jst.go.jp/article/kikaib1979/58/551/58\_551\_2093/\_article.

Barrero-Gil, A., Alonso, G. and Sanz-Andres, A., 2010. Energy harvesting
from transverse galloping. \emph{Journal of Sound and Vibration},
329(14), pp.2873--2883. Available at:
http://www.sciencedirect.com/science/article/pii/S0022460X10000891
{[}Accessed: 18 January 2016{]}.

Barry, O., Oguamanam, D.C.D. and Lin, D.C., 2010. Free Vibration
Analysis of a Single Conductor With a Stockbridge Damper.
\emph{Proceedings of the 23rd CANCAM}. 2010 pp. 950--952.

Bearman, P.W., 2011. Circular cylinder wakes and vortex-induced
vibrations. \emph{Journal of Fluids and Structures}, 27(5--6),
pp.648--658. Available at:
http://www.sciencedirect.com/science/article/pii/S0889974611000600
{[}Accessed: 20 January 2016{]}.

Bernitsas, M.M., Ben-Simon, Y., Raghavan, K. and Garcia, E.M.H., 2009.
The VIVACE Converter: Model Tests at High Damping and Reynolds Number
Around 10{[}sup 5{]}. \emph{Journal of Offshore Mechanics and Arctic
Engineering}, 131(1), p.011102. Available at:
http://offshoremechanics.asmedigitalcollection.asme.org/article.aspx?articleid=1472649.

Bernitsas, M.M. and Raghavan, K., 2008. Reduction/suppression of VIV of
circular cylinders through roughness distribution at 8×103 \&lt; Re
\&lt; 1.5×105. \emph{Proceedings of the International Conference on
Offshore Mechanics and Arctic Engineering - OMAE}. 2008

Bernitsas, M.M., Raghavan, K., Ben-Simon, Y. and Garcia, E.M.H., 2008.
VIVACE (Vortex Induced Vibration Aquatic Clean Energy): A New Concept in
Generation of Clean and Renewable Energy From Fluid Flow. \emph{Journal
of Offshore Mechanics and Arctic Engineering}.

Bernitsas, M.M.M., Raghavan, K., Ben-Simon, Y. and Garcia, E.M.H.M.H.,
2008. VIVACE (Vortex Induced Vibration Aquatic Clean Energy): A New
Concept in Generation of Clean and Renewable Energy From Fluid Flow.
\emph{Journal of Offshore Mechanics and Arctic Engineering}, 130(4),
p.041101. Available at:
http://www.scopus.com/record/display.url?eid=2-s2.0-56749179917\&origin=resultslist\&sort=plf-f\&src=s\&st1=A+new+concept+in+generation+of+clean+and+renewable+energy+from+fluid+flow\&sid=620865A71FAF26768C42655E1E8BC194.aXczxbyuHHiXgaIW6Ho7g:230\&sot=b\&sdt=b\&
{[}Accessed: 2 February 2016{]}.

Billa, L., Mansor, S. and Mahmud, A.R., 2004. Spatial information
technology in flood early warning systems: an overview of theory,
application and latest developments in Malaysia. \emph{Disaster
Prevention and Management}, 13, pp.356--363.

Borazjani, I. and Daghooghi, M., 2013. The fish tail motion forms an
attached leading edge vortex. \emph{Proceedings. Biological sciences /
The Royal Society}, 280(February), p.20122071. Available at:
http://www.pubmedcentral.nih.gov/articlerender.fcgi?artid=3574357\&tool=pmcentrez\&rendertype=abstract.

Chen, Z.Q., Liu, M., Hua, X. and Mou, T., 2012. Flutter, Galloping, and
Vortex-Induced Vibrations of H-Section Hangers. \emph{Journal of Bridge
Engineering}, 17(3), pp.500--508.

Chia, C.W., 2004. Managing flood problems in Malaysia. \emph{Buletin
Ingeniur}.

Deng, J., Ren, A.-L. and Shao, X.-M., 2007. The flow between a
stationary cylinder and a downstream elastic cylinder in cruciform
arrangement. \emph{Journal of Fluids and Structures}, 23(5),
pp.715--731. Available at:
https://www.sciencedirect.com/science/article/pii/S0889974606001472
{[}Accessed: 12 September 2018{]}.

Department of Irrigation and Drainage Malaysia, 2010. Hydrological
Procedure No.27 (Estimation of Design Flood Hydrograph).

Department of Irrigation and Drainage Malaysia, Managing The Flood
Problem In Malaysia.
\emph{Http://Www.Water.Gov.My/Images/Pdf/Managing\_Flood.Pdf}, pp.1--11.

Derakhshandeh, J.F., Arjomandi, M., Dally, B. and Cazzolato, B., 2014.
The effect of arrangement of two circular cylinders on the maximum
efficiency of Vortex-Induced Vibration power using a Scale-Adaptive
Simulation model. \emph{Journal of Fluids and Structures}, 49,
pp.654--666. Available at:
http://dx.doi.org/10.1016/j.jfluidstructs.2014.06.005.

Diana, G., Cigada, A., Belloli, M. and Vanali, M., 2003.
Stockbridge-type damper effectiveness evaluation: Part I - Comparison
between tests on span and on the shaker. \emph{IEEE Transactions on
Power Delivery}, 18(4), pp.1462--1469.

Ding, L., Zhang, L., Wu, C., et al., 2015. Flow induced motion and
energy harvesting of bluff bodies with different cross sections.
\emph{Energy Conversion and Management}.

Ding, L., Zhang, L., Kim, E.S. and Bernitsas, M.M., 2015. URANS vs.
experiments of flow induced motions of multiple circular cylinders with
passive turbulence control. \emph{Journal of Fluids and Structures}, 54,
pp.612--628.

Doare, O. and Michelin, S., 2011. Piezoelectric coupling in
energy-harvesting fluttering flexible plates: Linear stability analysis
and conversion efficiency. \emph{Journal of Fluids and Structures},
27(8), pp.1357--1375.

Drucker, E.G. and Lauder, G. V, 1999. Locomotor forces on a swimming
fish: three-dimensional vortex wake dynamics quantified using digital
particle image velocimetry. \emph{The Journal of Experimental Biology},
202, pp.2393--2412. Available at:
http://www.ncbi.nlm.nih.gov/pubmed/10460729.

Gonçalves, R.T. et al., 2013. Two-degree-of-freedom vortex-induced
vibration of circular cylinders with very low aspect ratio and small
mass ratio. \emph{Journal of Fluids and Structures}, 39(March),
pp.237--257.

Govardhan, R.N. and Williamson, C.H.K., 2006. Defining the `modified
Griffin plot' in vortex-induced vibration: revealing the effect of
Reynolds number using controlled damping. \emph{Journal of Fluid
Mechanics}, 561, p.147. Available at:
http://journals.cambridge.org/abstract\_S0022112006000310 {[}Accessed: 8
March 2016{]}.

Harris, I., Jones, P.D., Osborn, T.J. and Lister, D.H., 2014. Updated
high-resolution grids of monthly climatic observations - the CRU TS3.10
Dataset. \emph{International Journal of Climatology}, 34(3),
pp.623--642. Available at: http://doi.wiley.com/10.1002/joc.3711
{[}Accessed: 11 July 2014{]}.

Hayashi, M., Sakurai, A. and Ohya, Y., 2006. Wake interference of a row
of normal flat plates arranged side by side in a uniform flow.
\emph{Journal of Fluid Mechanics}, 164(1), p.1. Available at:
http://www.journals.cambridge.org/abstract\_S0022112086002446.

Huang, N.E. and Wu, Z., 2008. A review on Hilbert-Huang transform:
Method and its applications to geophysical studies. \emph{Reviews of
Geophysics}, 46(2), p.RG2006. Available at:
http://doi.wiley.com/10.1029/2007RG000228 {[}Accessed: 27 February
2019{]}.

Kato, N., Koide, M., Takahashi, T. and Shirakashi, M., 2006. Influence
of Cross-Sectional Configuration on the Longitudinal Vortex Excitation
of the Upstream Cylinder in Cruciform Two-Cylinder System. \emph{Journal
of Fluid Science and Technology}, 1(2), pp.126--137. Available at:
http://joi.jlc.jst.go.jp/JST.JSTAGE/jfst/1.126?from=CrossRef
{[}Accessed: 17 December 2017{]}.

Kato, N., Koide, M., Takahashi, T. and Shirakashi, M., 2007. Vibration
Control for a Circular Cylinder by a Strip-plate Set Downstream in
Cruciform Arrangement (1st Report, Influence of a Downstream Strip-plate
on the Shedding of Longitudinal Vortices from Fixed System).
\emph{Transactions of the Japan Society of Mechanical Engineers Series
B}, 73(728), pp.957--964. Available at:
https://www.jstage.jst.go.jp/article/kikaib1979/73/728/73\_728\_957/\_article/-char/ja/
{[}Accessed: 13 March 2016{]}.

Kato, N., Koide, M., Takahashi, T. and Shirakashi, M., 2012. VIVs of a
circular cylinder with a downstream strip-plate in cruciform
arrangement. \emph{Journal of Fluids and Structures}, 30, pp.97--114.

Kawabata, Y., Takahashi, T., Haginoya, T. and Shirakashi, M., 2013.
Interference Effect of Downstream Strip-Plate on the Crossflow Vibration
of a Square Cylinder. \emph{Journal of Fluid Science and Technology},
8(3), pp.647--658.

Kawabata, Y., Takahashi, T. and Shirakashi, M., 2009. Influence of a
Downstream Strip-plate on Galloping of a Square Cylinder in Cruciform
Arrangement. \emph{日本機械学会論文集(B編)}, 75(754).

Kementerian Kewangan Malaysia, 2015. Bajet 2016. Available at:
http://www.treasury.gov.my/index.php?option=com\_content\&view=article\&id=6437:ucapan-bajet-2016\&catid=256\&Itemid=2472\&lang=ms.

Khalak, A. and Williamson, C.H.K., 1999. MOTIONS, FORCES AND MODE
TRANSITIONS IN VORTEX-INDUCED VIBRATIONS AT LOW MASS-DAMPING.
\emph{Journal of Fluids and Structures}, 13(7--8), pp.813--851.
Available at:
http://www.sciencedirect.com/science/article/pii/S0889974699902360
{[}Accessed: 10 March 2016{]}.

Kim, Y.-M. and You, K.-P., 2002. Dynamic responses of a tapered tall
building to wind loads. \emph{Journal of Wind Engineering and Industrial
Aerodynamics}, 90(12--15), pp.1771--1782. Available at:
http://www.sciencedirect.com/science/article/pii/S0167610502002866
{[}Accessed: 10 March 2016{]}.

Kluger, J.M., Moon, F.C. and Rand, R.H., 2013. Shape optimization of a
blunt body Vibro-wind galloping oscillator. \emph{Journal of Fluids and
Structures}, 40, pp.185--200. Available at:
http://www.sciencedirect.com/science/article/pii/S0889974613000868
{[}Accessed: 10 March 2016{]}.

Koide, M., Sekizaki, T., et al., 2009. A Novel Technique for
Hydroelectricity Utilizing Vortex Induced Vibration. \emph{Proceedings
of the ASME Pressure Vessels and Piping Division Conference,
PVP2009-77487}. 2009

Koide, M. et al., 2007. Influence Of A Cruciform Arrangement Downstream
Strip-Plate On Crossflow Vibration. \emph{Journal of Computational and
Applied Mechanics}, 8(2), pp.135--148.

Koide, M. et al., 2013. Prospect of Micro Power Generation Utilizing VIV
in Small Stream Based on Verification Experiments of Power Generation in
Water Tunnel. \emph{Journal of Fluid Science and Technology}, 8(3),
pp.294--308. Available at:
https://www.jstage.jst.go.jp/article/jfst/8/3/8\_294/\_article
{[}Accessed: 13 March 2016{]}.

Koide, M. et al., 2006. Vortex Excitation Caused by Longitudinal
Vortices Shedding from Cruciform Cylinder System in Water Flow.
\emph{JSME International Journal}, 49(4), pp.1043--1048.

Koide, M., Kato, N., Takahashi, T. and Shirakashi, M., 2009. Vibration
Control for a Circular Cylinder by a Strip-Plate Set Downstream in
Cruciform Arrangement (2nd Report, Generation and Suppression of Vortex
Excitation on Elastically Supported Cylinder).
\emph{日本機械学会論文集(B編)}, 75(752), pp.691--699.

Koide, M., Oogane, K., Takahashi, T. and Shirakashi, M., 2004.
Experimental Study on Universality of Longitudinal Vortices Shedding
Periodically from Crisscross Circular Cylinder System in Uniform Flow.
\emph{Transactions of the Visualization Society of Japan}, 24(4),
pp.15--22. Available at:
https://www.jstage.jst.go.jp/article/tvsj/24/4/24\_4\_15/\_article.

Koide, M., Takahashi, T., Shirakashi, M. and Salim, S.A.Z.B.S., 2017.
Three-dimensional structure of longitudinal vortices shedding from
cruciform two-cylinder systems with different geometries. \emph{Journal
of Visualization}, pp.1--11.

Kuala Lumpur Monsoon Activity Centre, 2015. \emph{Northeast Monsoon
Report},

Larsen, C.M. and Halse, K.H., 1997. Comparison of models for vortex
induced vibrations of slender marine structures. \emph{Marine
Structures}, 10(6), pp.413--441. Available at:
http://www.sciencedirect.com/science/article/pii/S0951833997000117
{[}Accessed: 22 January 2016{]}.

Liao, J.C., Beal, D.N., Lauder, G. V and Triantafyllou, M.S., 2003. Fish
Exploiting Vortices Decrease Muscle Activity. \emph{Science}, 302(2003),
pp.1566--1569.

Luo, S.C., Chew, Y.T. and Ng, Y.T., 2003. Hysteresis phenomenon in the
galloping oscillation of a square cylinder. \emph{Journal of Fluids and
Structures}, 18(1), pp.103--118. Available at:
http://www.sciencedirect.com/science/article/pii/S0889974603000847
{[}Accessed: 10 March 2016{]}.

Mackowski, A.W. and Williamson, C.H.K., 2013. An experimental
investigation of vortex-induced vibration with nonlinear restoring
forces. \emph{Physics of Fluids}, 25(8).

Maruai, N.M., Ali, M.S.M., Ismail, M.H. and Zaki, S.A., 2018.
Flow-induced vibration of a square cylinder and downstream flat plate
associated with micro-scale energy harvester. \emph{Journal of Wind
Engineering and Industrial Aerodynamics}, 175, pp.264--282. Available
at:
https://www.scopus.com/inward/record.uri?eid=2-s2.0-85042219159\&doi=10.1016\%2Fj.jweia.2018.01.010\&partnerID=40\&md5=2f3f62b94bb69ced3368b32e682aefc7.

Maruai, N.M., Mat Ali, M.S., Ismail, M.H. and Shaikh Salim, S.A.Z.,
2017. Downstream flat plate as the flow-induced vibration enhancer for
energy harvesting. \emph{Journal of Vibration and Control},
p.107754631770787. Available at:
http://journals.sagepub.com/doi/10.1177/1077546317707877 {[}Accessed: 2
January 2018{]}.

Mat Ali, M.S., Doolan, C.J. and Wheatley, V., 2011. Low Reynolds number
flow over a square cylinder with a splitter plate. \emph{Physics of
Fluids}, 23(3).

Nakamura, T. et al., 2013. \emph{Flow-Induced Vibrations:
Classifications and Lessons from Practical Experiences},
Butterworth-Heinemann.

Nakashima, M., 1986. Vortex excitation. , 163, pp.149--169.

Nguyen, T. et al., 2012. Influence of mass and damping ratios on VIVs of
a cylinder with a downstream counterpart in cruciform arrangement.
\emph{Journal of Fluids and Structures}, 28, pp.40--55.

Norman, J., 2012. VIVACE Hydrokinetic Power Generator System : The
Solution for Fish Protection and Debris Avoidance in Alaskan Rivers. ,
pp.2--6.

Ogink, R.H.M. and Metrikine, A.V., 2010. A wake oscillator with
frequency dependent coupling for the modeling of vortex-induced
vibration. \emph{Journal of Sound and Vibration}, 329(26),
pp.5452--5473. Available at:
http://www.sciencedirect.com/science/article/pii/S0022460X10004621
{[}Accessed: 10 March 2016{]}.

Pineirua, M., Doaré, O. and Michelin, S., 2015. Influence and
optimization of the electrodes position in a piezoelectric energy
harvesting flag. \emph{Journal of Sound and Vibration}, 346,
pp.200--215.

Quen, L.K. et al., 2014. Investigation on the effectiveness of helical
strakes in suppressing VIV of flexible riser. \emph{Applied Ocean
Research}, 44, pp.82--91. Available at:
http://www.sciencedirect.com/science/article/pii/S0141118713001028
{[}Accessed: 10 March 2016{]}.

Raghavan, K., 2007. \emph{Energy Extraction from a Steady Flow Using
Vortex Induced Vibration.} The University of Michigan.

Seyed-Aghazadeh, B., Carlson, D.W. and Modarres-Sadeghi, Y., 2015. The
influence of taper ratio on vortex-induced vibration of tapered
cylinders in the crossflow direction. \emph{Journal of Fluids and
Structures}, 53, pp.84--95. Available at:
http://www.sciencedirect.com/science/article/pii/S0889974614001765
{[}Accessed: 4 May 2017{]}.

Shiraishi, N., Matsumoto, M., Shirato, H. and Ishizaki, H., 1988. On
aerodynamic stability effects for bluff rectangular cylinders by their
corner-cut. \emph{Journal of Wind Engineering and Industrial
Aerodynamics}, 28(1--3), pp.371--380. Available at:
http://www.sciencedirect.com/science/article/pii/016761058890133X
{[}Accessed: 10 March 2016{]}.

Shirakashi, M., Mizuguchi, K. and Bae, H.M., 1989. Flow-induced
excitation of an elastically-supported cylinder caused by another
located downstream in cruciform arrangement. \emph{Journal of Fluids and
Structures}, 3(6), pp.595--607.

Su, H., Li, H., Chen, Z. and Wen, Z., 2016. An approach using ensemble
empirical mode decomposition to remove noise from prototypical
observations on dam safety. \emph{SpringerPlus}, 5, p.650. Available at:
http://www.ncbi.nlm.nih.gov/pubmed/27330916 {[}Accessed: 14 February
2019{]}.

Sukri Mat Ali, M., Doolan, C.J. and Wheatley, V., 2013. Aeolian Tones
Generated by a Square Cylinder with a Detached Flat Plate. \emph{AIAA
Journal}, 51(2), pp.291--301. Available at:
http://arc.aiaa.org/doi/abs/10.2514/1.J051378.

Sukri Mat Ali, M., Doolan, C.J. and Wheatley, V., 2011. The sound
generated by a square cylinder with a splitter plate at low Reynolds
number. \emph{Journal of Sound and Vibration}, 330(15), pp.3620--3635.
Available at: http://dx.doi.org/10.1016/j.jsv.2011.03.008.

Sun, H. et al., 2016. Effect of mass-ratio, damping, and stiffness on
optimal hydrokinetic energy conversion of a single, rough cylinder in
flow induced motions. \emph{Renewable Energy}, 99, pp.936--959.
Available at:
http://www.sciencedirect.com/science/article/pii/S0960148116306206
{[}Accessed: 12 February 2017{]}.

Sun, H., Soo Kim, E., Bernitsas, M.P. and Bernitsas, M.M., 2015. Virtual
Spring--Damping System for Flow-Induced Motion Experiments.
\emph{Journal of Offshore Mechanics and Arctic Engineering}.

Sunami, T. et al., 2002. Mixing and Combustion Control Strategies for
Effecient Scramjet Operation in Wide Range of Flight Mach Numbers.
\emph{11th AIAA/AAAF International Space Planes and Hypersonic Systems
and Technologies Conference}, (AIAA 2002-5116).

Takahashi, T., Baranyi, L. and Shirakashi, M., 1999. Configuration and
Frequency of Longitudinal Vortices Shedding from Circular Cylinders in
Cruciform Arrangement. \emph{Journal of the Visualization Society of
Japan}, 19(75), pp.64--72.

Takaijudin, H. et al., 2010. Implementation of Urban Stormwater
Management Practices in Malaysia. \emph{International Conference on
Sustainable Building and Infrastructure (ICSBI 2010)}, p.5.

Tanaka, H. et al., 2012. Experimental investigation of aerodynamic
forces and wind pressures acting on tall buildings with various
unconventional configurations. \emph{Journal of Wind Engineering and
Industrial Aerodynamics}, 107--108, pp.179--191. Available at:
http://www.sciencedirect.com/science/article/pii/S016761051200116X
{[}Accessed: 10 March 2016{]}.

Vandiver, J.K., Jaiswal, V. and Jhingran, V., 2009. Insights on
vortex-induced, traveling waves on long risers. \emph{Journal of Fluids
and Structures}, 25(4), pp.641--653. Available at:
http://www.sciencedirect.com/science/article/pii/S0889974608001321
{[}Accessed: 10 March 2016{]}.

Venugopal, A., Agrawal, A. and Prabhu, S. V, 2011. Review on vortex
flowmeter---Designer perspective. \emph{Sensors and Actuators A:
Physical}, 170(1), pp.8--23.

Williamson, C.H.K., 1996. Vortex Dynamics in the Cylinder Wake.
\emph{Annual Review of Fluid Mechanics}, 28(1), pp.477--539. Available
at:
http://www.annualreviews.org/doi/10.1146/annurev.fl.28.010196.002401.

Williamson, C.H.K. and Govardhan, R., 2004. Vortex-Induced Vibrations.
\emph{Annual Review of Fluid Mechanics}, 36(1), pp.413--455. Available
at:
http://arjournals.annualreviews.org/doi/abs/10.1146\%252Fannurev.fluid.36.050802.122128.

Wu, W., 2011. \emph{Two-Dimensional RANS Simulation of Flow Induced
Motion of Circular Cylinder with Passive Turbulence Control}. The
University of Michigan.

Wu, W., Bernitsas, M.M. and Maki, K., 2011. RANS Simulation vs.
Experiments of Flow Induced Motion of Circular Cylinder With Passive
Turbulence Control at 35,000\textless{}Re\textless{}130,000.
\emph{Volume 7: CFD and VIV; Offshore Geotechnics}. 1 January 2011 ASME,
pp. 733--744.

Xia, Y., Michelin, S. and Doaré, O., 2015. Fluid-solid-electric lock-in
of energy-harvesting piezoelectric flags. \emph{Physical Review
Applied}, 3(1), p.14009.

Xia, Y., Michelin, S.S., Doaré, O. and Doare, O., 2015.
Resonance-induced enhancement of the energy harvesting performance of
piezoelectric flags. \emph{Applied Physics Letters}, 107(26), pp.1--5.

Xiao, F., Chen, G.S., Zatar, W. and Hulsey, J.L., 2018. Quantification
of Dynamic Properties of Pile Using Ensemble Empirical Mode
Decomposition. \emph{Advances in Civil Engineering}, 2018, pp.1--6.
Available at: https://www.hindawi.com/journals/ace/2018/8379871/
{[}Accessed: 14 February 2019{]}.

Xu, J., He, M. and Bose, N., 2009. Vortex modes and vortex-induced
vibration of a long, flexible riser. \emph{Ocean Engineering}, 36(6--7),
pp.456--467. Available at:
http://www.sciencedirect.com/science/article/pii/S0029801809000146
{[}Accessed: 10 March 2016{]}.

Zdravkovich, M.M., 1981. Review and classification of various
aerodynamic and hydrodynamic means for suppressing vortex shedding.
\emph{Journal of Wind Engineering and Industrial Aerodynamics}, 7(2),
pp.145--189. Available at:
http://www.sciencedirect.com/science/article/pii/0167610581900362
{[}Accessed: 4 January 2016{]}.

Zhao, M. and Lu, L., 2018. Numerical simulation of flow past two
circular cylinders in cruciform arrangement. \emph{Journal of Fluid
Mechanics}, 848, pp.1013--1039. Available at:
https://www.cambridge.org/core/product/identifier/S0022112018003804/type/journal\_article
{[}Accessed: 12 September 2018{]}.

\end{document}
